\documentclass[12pt,letterpaper]{article}
\usepackage[margin=1in]{geometry}
\usepackage{amsfonts}
\usepackage{amssymb}
\usepackage{amsthm}
\usepackage{amsmath}
\usepackage{enumerate}

%Here are some user-defined notations
\newcommand{\RR}{\mathbf R}  %bold R
\newcommand{\CC}{\mathbf C}  %bold C
\newcommand{\ZZ}{\mathbf Z}   %bold Z
\newcommand{\QQ}{\mathbf Q}   %bold Q
\newcommand{\rr}{\mathbb R}     %blackboard bold R
\newcommand{\cc}{\mathbb C}    %blackboard bold R
\newcommand{\zz}{\mathbb Z}    %blackboard bold R
\newcommand{\qq}{\mathbb Q}   %blackboard bold Q
\newcommand{\calM}{\mathcal M}  %calligraphic M
\newcommand{\sm}{\setminus} 
\newcommand{\bfa}{\mathbf a}
\newcommand{\bfb}{\mathbf b}
\newcommand{\bfc}{\mathbf c}




%Here are some user-defined operators
\newcommand{\re}{\operatorname {Re}}
\newcommand{\im}{\operatorname {Im}}


%These commands deal with theorem-like environments (i.e., italic)
\theoremstyle{plain}
\newtheorem{theorem}{Theorem}[section]
\newtheorem{corollary}[theorem]{Corollary}
\newtheorem{lemma}[theorem]{Lemma}
\newtheorem{conjecture}[theorem]{Conjecture}

%These deal with definition-like environments (i.e., non-italic)
\theoremstyle{definition}
\newtheorem{definition}[theorem]{Definition}
\newtheorem{example}[theorem]{Example}
\newtheorem{remark}[theorem]{Remark}

%your name and date in the header.
\usepackage[us]{datetime} 
\usepackage{fancyhdr}
\pagestyle{fancy}
\lhead{}
\chead{MATH 2710\\ Homework 1}
\rhead{ Your name \\ \today}
\lfoot{}
\cfoot{}
\rfoot{\thepage}
\renewcommand{\headrulewidth}{0 pt}
\renewcommand{\footrulewidth}{0 pt}
\begin{document}
\begin{enumerate}[1.]
\item Prove or give a counter example. 
\[(S\cap T) \cup U=S\cap(T\cup U)\text{ for any sets }S,\ T, \text{ and }U.\]
{\bf Solution:} We will provide a counter example. Let 
\[S=\{1,2,3\},\ T=\{2,4,5,6\}, \text{ and }U=\{8,9,10\}.\]
We have that
\[S\cap T=\{2\}\]
and so 
\[(S\cap T)\cup U=\{2,8,9,10\}.\]
Moreover, we have 
\[T\cup U=\{2,4,5,6,8,9,10\}\]
and so 
\[S\cap(T\cup U)=\{2\}.\]
\item Prove or give a counter example.
\[S\cup T = T \Longleftrightarrow S\subseteq T.\]
{\bf Solution: }We will show that the statement is true.
\begin{proof} We will first show that if $S\cup T=T$ then $S\subseteq T$. By definition of union we have that $S\subseteq S\cup T$. Since $S\cup T=T$ we have that $S\subseteq T$. Now suppose that $S\subseteq T$, we will show that this implies that $S\cup T=T$. To do this we show that $S\cup T\subseteq T$ and that $T\subseteq S\cup T$. However, we automatically will have that $T\subseteq S\cup T$. So let $x\in S\cup T$ be arbitrary. If $x\in S$ then since $S\subseteq T$ we have that $x\in T$. If $x\in T$ then obviously $x\in T$. In both cases we have that $x\in T$ and thus since $x\in S\cup T$ was arbitrary we have that $S\cup T\subseteq T$ if $S\subseteq T$.
\end{proof}
\item Prove the distributive law. 
\[A\cap(B\cup C) =(A\cap B) \cup (A\cap C)\]
\begin{proof}We will prove the proposition by show that $A\cap (B\cup C)\subseteq (A\cap B)\cup (A\cap C)$ and 
$(A\cap B)\cup (A\cap C)\subseteq A\cap(B\cup C)$. Let $x\in A\cap (B\cup C)$ be arbitrary. We have that $x\in A$ and $x\in B\cup C$. Since $x\in B\cup C$ then $x\in B$ or $x\in C$ (or both). If $x\in B$ then $x\in A\cap B$ and hence $x\in (A\cap B)\cup (A\cap C)$. Similarly if $x\in C$ we have that $x\in (A\cap B)\cup (A\cap C)$. Hence, since $x\in A\cap (B\cup C)$ was arbitrary we have that 
\[A\cap (B\cup C)\subseteq (A\cap B)\cup (A\cap C).\]
Now let $x\in (A\cap B)\cup (A\cap C)$ be arbitrary. If $x\in (A\cap B)\cup (A\cap C)$ then $x\in (A\cap B)$ or $x\in (A\cap C)$ (or both). If $x\in A\cap B$ then $x\in A$ and $x\in B$ and therefore $x\in B\cup C$ as well. So if $x\in A\cap B$ then $x\in A\cap (B\cup C)$. Similarly, if $x\in A\cap C$ then $x\in A\cap (B\cup C)$. Hence since $x\in (A\cap B)\cup (A\cap C)$ was arbitrary we have that 
\[(A\cap B)\cup (A\cap C)\subseteq A\cap (B\cup C).\]
\end{proof}
\end{enumerate}

\end{document}









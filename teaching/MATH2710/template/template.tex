\documentclass[12pt,letterpaper]{article}
\usepackage[margin=1in]{geometry}
\usepackage{amsfonts}
\usepackage{amssymb}
\usepackage{amsthm}

%Here are some user-defined notations
\newcommand{\RR}{\mathbf R}  %bold R
\newcommand{\CC}{\mathbf C}  %bold C
\newcommand{\ZZ}{\mathbf Z}   %bold Z
\newcommand{\QQ}{\mathbf Q}   %bold Q
\newcommand{\rr}{\mathbb R}     %blackboard bold R
\newcommand{\cc}{\mathbb C}    %blackboard bold R
\newcommand{\zz}{\mathbb Z}    %blackboard bold R
\newcommand{\qq}{\mathbb Q}   %blackboard bold Q
\newcommand{\calM}{\mathcal M}  %calligraphic M
\newcommand{\sm}{\setminus} 
\newcommand{\bfa}{\mathbf a}
\newcommand{\bfb}{\mathbf b}
\newcommand{\bfc}{\mathbf c}




%Here are some user-defined operators
\newcommand{\re}{\operatorname {Re}}
\newcommand{\im}{\operatorname {Im}}


%These commands deal with theorem-like environments (i.e., italic)
\theoremstyle{plain}
\newtheorem{theorem}{Theorem}[section]
\newtheorem{corollary}[theorem]{Corollary}
\newtheorem{lemma}[theorem]{Lemma}
\newtheorem{conjecture}[theorem]{Conjecture}

%These deal with definition-like environments (i.e., non-italic)
\theoremstyle{definition}
\newtheorem{definition}[theorem]{Definition}
\newtheorem{example}[theorem]{Example}
\newtheorem{remark}[theorem]{Remark}

%your name and date in the header.
\usepackage[us]{datetime} 
\usepackage{fancyhdr}
\pagestyle{fancy}
\lhead{}
\chead{MATH 2710\\ Homework \#}
\rhead{ Your name \\ \today}
\lfoot{}
\cfoot{}
\rfoot{\thepage}
\renewcommand{\headrulewidth}{0 pt}
\renewcommand{\footrulewidth}{0 pt}
\begin{document}


Here is some inline math $f(x)=e^{2x^2}$. Here is a displayed equation,
\[e^x=\sum_{n=0}^\infty \frac{x^n}{n!}\]
Notice that it does not have a label. \\

{\bf \noindent Here's some more examples:}\\
The function $\sin x$ can be defined as an infinite series
\begin{equation}\label{sineseries}
\sin x = x - \frac{x^3}{3!} + \frac{x^5}{5!} - \frac{x^7}{7!} + \cdots = \sum_{k \geq 0} \frac{x^{2k+1}}{(2k+1)!}.
\end{equation}
Here is another way to characterize it, using differential equations and initial conditions.

\begin{theorem}\label{diffthm}
The function $\sin x$ is the \underline{unique} solution of the differential equation
\begin{equation}\label{sine-eqn}
\frac{d^2y}{dx^2} + y = 0
\end{equation}
satisfying the initial conditions $y(0) = 0$ and $y'(0) = 1$.
\end{theorem}

Notice in the code for this file that the number for the theorem, \ref{diffthm}, is {\it not} hard-coded, and that 
if you need to manually enter parentheses if you want the equation number to appear in text as (\ref{sine-eqn}).



There are four references below: \cite{irros}, \cite{unabomber}, and \cite{roquette}.

\begin{thebibliography}{4}


\bibitem{irros}
K. Ireland and M. Rosen, ``A Classical Introduction to Modern 
Number Theory,'' 2nd ed., Springer-Verlag, New York, 1990.

\bibitem{unabomber}
T. J. Kaczynski, Another proof of Wedderburn's theorem, 
{\it Amer. Math. Monthly} {\bf 71} (1964), 652--653.


\bibitem{roquette}
P. Roquette, Class field theory in characteristic $p$, its origin 
and development, pp.~549--631 in: ``Class field theory -- its centenary 
and prospect,'' Math. Soc. Japan, Tokyo, 2001.

\end{thebibliography}

\end{document}









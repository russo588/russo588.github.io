\documentclass[11pt]{exam}
\RequirePackage{amssymb, amsfonts, amsmath, latexsym, verbatim, xspace, setspace}
\RequirePackage{tikz, pgflibraryplotmarks}
\usepackage[margin=1in]{geometry}
\usepackage{amsmath, amsthm, amssymb}
\newtheorem*{thm}{{\bf Theorem}}
\newtheorem{lemma}{{\bf Lemma}}
\newcommand{\A}{\mathfrak{A}}
\theoremstyle{definition}
\newtheorem{define}{Definition}
\newtheorem{claim}{Claim}
\newtheorem*{method}{Method}
\newtheorem{ex}{Example}
\newcommand{\dydx}{\dfrac{dy}{dx}}
\newcommand{\dydt}{\dfrac{dy}{dt}}
\newcommand{\dxdt}{\dfrac{dx}{dt}}
\newcommand{\dxdy}{\dfrac{dx}{dy}}
\newcommand{\pp}{\prime\prime}
\newcommand{\p}{\prime}
\renewcommand{\d}[2]{\dfrac{d#1}{d#2}}
\newcommand{\dd}[2]{\dfrac{d^2#1}{d#2^2}}
\newcommand{\ypp}{y^{\prime\prime}}
\newcommand{\yp}{y^{\prime}}
\renewcommand\thesection{2.5}
\usepackage{xcolor}
\usepackage{graphicx}
\usepackage{lipsum}% Used for dummy text.
\usepackage{enumerate}

% Here's where you edit the Class, Exam, Date, etc.
\newcommand{\class}{MATH2710}
\newcommand{\term}{Fall 2016}
\newcommand{\examnum}{Practice Worksheet 3}
\newcommand{\examdate}{9/30/16}
\newcommand{\timelimit}{50 Minutes}

% For an exam, single spacing is most appropriate
\singlespacing
% \onehalfspacing
% \doublespacing

% For an exam, we generally want to turn off paragraph indentation
\parindent 0ex

\begin{document} 

% These commands set up the running header on the top of the exam pages
\pagestyle{head}
\chead{Worksheet 3}
%\firstpageheader{}{}{}
%\runningheader{\class}{\examnum\ - Page \thepage\ of \numpages}{\examdate}
%\runningheadrule
% {\bf \class} \hfill  \textbf{Name:\underline{\hspace{2in}}}\\
%\ \\
% {\bf \examnum} \hfill  \textbf{Date:\underline{\hspace{2in}}}\\


%This exam contains \numpages\ pages (including this cover page) and
%\numquestions\ problems.  Check to see if any pages are missing.  Enter
%all requested information on the top of this page, and put your initials
%on the top of every page, in case the pages become separated.\\
%
%You may \textit{not} use your books, notes, or any unapproved calculator on this exam.\\
%
%You are required to show your work on each problem on this exam.  The following rules apply:\\
%
%\begin{minipage}[t]{3.7in}
%
%\begin{itemize}
%\item \textbf{Organize your work}, in a reasonably neat and coherent way, in
%the space provided. Work scattered all over the page without a clear ordering will 
%receive very little credit.  
%
%\item \textbf{Mysterious or unsupported answers will not receive full
%credit}.  A correct answer, unsupported by calculations, explanation,
%or algebraic work will receive no credit; an incorrect answer supported
%by substantially correct calculations and explanations might still receive
%partial credit.
%
%\item If you need more space, ask for an extra sheet of paper to continue the problem on; clearly indicate when you have done this.
%\end{itemize}
%
%{\bf Do not write in the table to the right.}
%\end{minipage}
%\hfill
%\begin{minipage}[t]{2.3in}
%\vspace{0pt}
%\cellwidth{3em} %cellwidth
%\gradetablestretch{2}%some sort of stretching of the table
%\vqword{Problem}
%\addpoints % required here by exam.cls, even though questions haven't started yet.	
%\gradetable[v]%[pages]  % Use [pages] to have grading table by page instead of question
%\end{minipage}
%\newpage % End of cover page


\begin{questions}
\question Complete the following definitions. 
\begin{enumerate}[(a)]
\item We say for two sets $A$ and $B$ that $|A|=|B|\ldots $
\item We say a sequence $x_n\rightarrow x\ldots$ 
\item We say a sequence $x_n\rightarrow \infty\ldots $
\end{enumerate}
\vfill
\question True or False. 
\begin{enumerate}[(a)]
\item $|\mathbb{Q}|=|\mathbb{R}|$
\item $|\mathbb{N}|=|\mathbb{Q}|$
\item For any set $A$ we have $|A|\neq |P(A)|$ where $P(A)$ is the power set of $A$. 
\item $|x-y|\geq ||x|-|y||$
\item $|x-y|\geq |x|+|y|$
\end{enumerate}
\vspace{1.5in}
\newpage 
\question Show that the even numbers have the same cardinality as the odd numbers.  
\newpage
\question Show that $\{\frac{1}{n^2+1}\}_{n=1}^\infty$ converges by monotone sequence theorem. 
\newpage
\question Show that the sequence $\{\frac{n^2+1}{n}\}_{n=1}^\infty$ diverges to infinity.
\end{questions}

\end{document}


%%%%%%%%%%%%%%%%%%%%%%%%%%%%%%%%%%%%%%%%%%%%%%%
% Basic question
\addpoints
\question[10] Differentiate $f(x)=x^2$ with respect to $x$.

% Question with parts
\newpage
\addpoints
\question Consider the function $f(x)=x^2$.
\begin{parts}
\part[5] Find $f'(x)$ using the limit definition of derivative.
\vfill
\part[5] Find the line tangent to the graph of $y=f(x)$ at the point where $x=2$.
\vfill
\end{parts}

% If you want the total number of points for a question displayed at the top,
% as well as the number of points for each part, then you must turn off the point-counter
% or they will be double counted.
\newpage
\addpoints
\question[10] Consider the function $f(x)=x^3$.
\noaddpoints % If you remove this line, the grading table will show 20 points for this problem.
\begin{parts}
\part[5] Find $f'(x)$ using the limit definition of derivative.
\vspace{4.5in}
\part[5] Find the line tangent to the graph of $y=f(x)$ at the point where $x=2$.
\end{parts}
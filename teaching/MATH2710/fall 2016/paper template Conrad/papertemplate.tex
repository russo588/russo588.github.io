\documentclass[12pt,letterpaper]{amsart}
\setlength{\oddsidemargin}{.0in}
\setlength{\evensidemargin}{.0in}
\setlength{\textwidth}{6.5in}
\setlength{\topmargin}{-.3in}
\setlength{\headsep}{.20in}
\setlength{\textheight}{9.in}
\usepackage[leqno]{amsmath}
\usepackage{amsfonts}
\usepackage{amssymb}
\usepackage{amsthm}
\usepackage{amssymb}
\usepackage[all]{xy}
\usepackage{graphicx}

%Here are some user-defined notations
\newcommand{\RR}{\mathbf R}  %bold R
\newcommand{\CC}{\mathbf C}  %bold C
\newcommand{\ZZ}{\mathbf Z}   %bold Z
\newcommand{\QQ}{\mathbf Q}   %bold Q
\newcommand{\rr}{\mathbb R}     %blackboard bold R
\newcommand{\cc}{\mathbb C}    %blackboard bold R
\newcommand{\zz}{\mathbb Z}    %blackboard bold R
\newcommand{\qq}{\mathbb Q}   %blackboard bold Q
\newcommand{\ZZn}[1]{\ZZ/{#1}\ZZ}
\newcommand{\zzn}[1]{\zz/{#1}\zz}
\newcommand{\calM}{\mathcal M}  %calligraphic M
\newcommand{\sm}{\setminus} 
\newcommand{\bfa}{\mathbf a}
\newcommand{\bfb}{\mathbf b}
\newcommand{\bfc}{\mathbf c}


%improving spacing in tables (space above and below characters in a row)
\newcommand{\tfix}{\rule{0pt}{2.6ex}}
\newcommand{\bfix}{\rule[-1.2ex]{0pt}{0pt}}



%Here are commands with variable inputs 
\newcommand{\intf}[1]{\int_a^b{#1}\,dx}
\newcommand{\intfb}[3]{\int_{#1}^{#2}{#3}\,dx}
\newcommand{\marginalfootnote}[1]{%
        \footnote{#1}
        \marginpar[\hfill{\sf\thefootnote}]{{\sf\thefootnote}}}
\newcommand{\edit}[1]{\marginalfootnote{#1}}


%Here are some user-defined operators
\newcommand{\Tr}{\operatorname {Tr}}
\newcommand{\GL}{\operatorname {GL}}
\newcommand{\SL}{\operatorname {SL}}
\newcommand{\Prob}{\operatorname {Prob}}
\newcommand{\re}{\operatorname {Re}}
\newcommand{\im}{\operatorname {Im}}


%These commands deal with theorem-like environments (i.e., italic)
\theoremstyle{plain}
\newtheorem{theorem}{Theorem}[section]
\newtheorem{corollary}[theorem]{Corollary}
\newtheorem{lemma}[theorem]{Lemma}
\newtheorem{conjecture}[theorem]{Conjecture}

%These deal with definition-like environments (i.e., non-italic)
\theoremstyle{definition}
\newtheorem{definition}[theorem]{Definition}
\newtheorem{example}[theorem]{Example}
\newtheorem{remark}[theorem]{Remark}

%This numbers equations by section
\numberwithin{equation}{section}



%This is for hypertext references
\usepackage{color}
\usepackage{hyperref}


\begin{document}


\begin{titlepage}
\title{Title Here}
\author{Your Name}
\date{Date Here}
\maketitle

\centerline{\Large Math 2710}


\thispagestyle{empty}
\end{titlepage}


\pagebreak



%%%Start your work here. 

\section{Introduction}\label{intro}


In this file, edit the information between  
\verb1\begin{titlepage}1 and \verb1\end{titlepage}1.
Do {\it not} change the typesetting commands such as 
\verb1\setlength1 at the top of the file, which 
affect the size of the output.

Write your paper between  
\verb1\section{Introduction}\label{intro}1 and \verb1\end{document}1.
In this section you should put an introduction.  
Tell us what your topic is about, roughly, 
and what you are going to do with it. 



If you need any LaTeX command, see if you can find a similar one in one of the LaTeX files you have and then 
copy, paste and edit.  Or ask a math professor (most know LaTeX).


\section{The Next Section}\label{sec1}





The function $\sin x$ can be defined as an infinite series
\begin{equation}\label{sineseries}
\sin x = x - \frac{x^3}{3!} + \frac{x^5}{5!} - \frac{x^7}{7!} + \cdots = \sum_{k \geq 0} \frac{x^{2k+1}}{(2k+1)!}.
\end{equation}
Here is another way to characterize it, using differential equations and initial conditions.

\begin{theorem}\label{diffthm}
The function $\sin x$ is the \underline{unique} solution of the differential equation
\begin{equation}\label{sine-eqn}
\frac{d^2y}{dx^2} + y = 0
\end{equation}
satisfying the initial conditions $y(0) = 0$ and $y'(0) = 1$.
\end{theorem}

Notice in the code for this file that the number for the theorem, \ref{diffthm}, is {\it not} hard-coded, and that 
if you need to manually enter parentheses if you want the equation number to appear in text as (\ref{sine-eqn}).


\section{The Section After That}\label{sec2}

There is nothing here.

\appendix

\section{Some More Stuff}


There are four references below: \cite{irros}, \cite{unabomber}, \cite{roquette}, and \cite{wiki}. 



\begin{thebibliography}{4}


\bibitem{irros}
K. Ireland and M. Rosen, ``A Classical Introduction to Modern 
Number Theory,'' 2nd ed., Springer-Verlag, New York, 1990.

\bibitem{unabomber}
T. J. Kaczynski, Another proof of Wedderburn's theorem, 
{\it Amer. Math. Monthly} {\bf 71} (1964), 652--653.


\bibitem{roquette}
P. Roquette, Class field theory in characteristic $p$, its origin 
and development, pp.~549--631 in: ``Class field theory -- its centenary 
and prospect,'' Math. Soc. Japan, Tokyo, 2001.


\bibitem{wiki}
Wikipedia, {\color{blue}\href{http://en.wikipedia.org/wiki/Spectral_theorem}{\tt http://en.wikipedia.org/wiki/Spectral\underline{ }theorem}}.




\end{thebibliography}

\end{document}









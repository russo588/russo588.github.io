\documentclass[12pt,letterpaper]{article}
\usepackage[margin=1in]{geometry}
\usepackage{amsfonts}
\usepackage{amssymb}
\usepackage{amsthm}
\usepackage{amsmath}
\usepackage{enumerate}

%Here are some user-defined notations
\newcommand{\RR}{\mathbf R}  %bold R
\newcommand{\CC}{\mathbf C}  %bold C
\newcommand{\ZZ}{\mathbf Z}   %bold Z
\newcommand{\QQ}{\mathbf Q}   %bold Q
\newcommand{\rr}{\mathbb R}     %blackboard bold R
\newcommand{\cc}{\mathbb C}    %blackboard bold R
\newcommand{\zz}{\mathbb Z}    %blackboard bold R
\newcommand{\qq}{\mathbb Q}   %blackboard bold Q
\newcommand{\calM}{\mathcal M}  %calligraphic M
\newcommand{\sm}{\setminus} 
\newcommand{\bfa}{\mathbf a}
\newcommand{\bfb}{\mathbf b}
\newcommand{\bfc}{\mathbf c}




%Here are some user-defined operators
\newcommand{\re}{\operatorname {Re}}
\newcommand{\im}{\operatorname {Im}}


%These commands deal with theorem-like environments (i.e., italic)
\theoremstyle{plain}
\newtheorem{theorem}{Theorem}[section]
\newtheorem{corollary}[theorem]{Corollary}
\newtheorem{lemma}[theorem]{Lemma}
\newtheorem{conjecture}[theorem]{Conjecture}
\newtheorem{prop}[theorem]{Proposition}

%These deal with definition-like environments (i.e., non-italic)
\theoremstyle{definition}
\newtheorem{definition}[theorem]{Definition}
\newtheorem{example}[theorem]{Example}
\newtheorem{remark}[theorem]{Remark}

%your name and date in the header.
\usepackage[us]{datetime} 
\usepackage{fancyhdr}
\pagestyle{fancy}
\lhead{}
\chead{MATH 2710\\ Homework 2}
\rhead{ Your name \\ \today}
\lfoot{}
\cfoot{}
\rfoot{\thepage}
\renewcommand{\headrulewidth}{0 pt}
\renewcommand{\footrulewidth}{0 pt}
\begin{document}
\begin{enumerate}[{\bf1.}]
\item If $ac\mid bc$ and $c\neq 0$, prove that $a\mid b$. 
\begin{proof}
Since $ac\mid bc$ there exists a $q\in \zz$ such that $acq=bc$. Since $c\neq 0$ we can divide both sides by $c$ and get $aq=b$. Therefore $a\mid b$. 
\end{proof}
\item Prove that $\gcd(ad,bd)=|d|\gcd(a,b)$
\begin{proof}
Since $|d|\mid d$ we have that $|d|\gcd(a,b)\mid ad$ and $|d|\gcd(a,b)\mid bd$. Hence $|d|\gcd(a,b)$ is a common divisor of $ad$ and $bd$. We will now show that $|d|\gcd(a,b)$ is the largest common divisor. Since $\gcd(a,b)\geq 1$ we have that 
\[|d\gcd(a,b)|= |d|\gcd(a,b).\] 
By the characterization of the greatest common divisor there exist integers $x$ and $y$ such that $ax+by=\gcd(a,b)$. Hence,
\[adx+bdy=d\gcd(a,b).\]
If $c$ is a common divisor of $ad$ and $bd$ then $c\mid adx+bdy$. Hence, $c\leq |c|\leq |d|\gcd(a,b)$ by Proposition 2.11(iv).
\end{proof}
\item Prove that $\gcd(a,c)=\gcd(b,c)=1$ if and only if $\gcd(ab,c)=1$. 
\begin{proof} Suppose that $\gcd(a,c)=\gcd(b,c)=1$. There exist some $x_0, y_0, x_1, y_1\in \zz$ such that 
\[ax_0+cy_0=1\] and \[bx_1+cy_1=1.\]
Therefore,
\[1=bx_1+cy_1=b(ax_0+cy_0)x_1+cy_1=ab(x_0x_1)+c(by_0x_1+y_1)\]
Hence, we have that $\gcd(ab,c)=1$ by Proposition 2.27 (i).\\

Conversely, suppose that $\gcd(ab,c)=1$. There exist some $x,y\in \zz$ such that 
\[abx+cy=1.\]
However, this implies that $\gcd(a,c)=1$ and $\gcd(b,c)=1$ by Proposition 2.27 (i) since $ax, bx\in \zz$. 

\end{proof}
\newpage
\item Prove that any two consecutive integers are relatively prime. 
\begin{proof}Let $n\in \zz$ be an arbitrary integer. Suppose for the sake of contradiction that there exists a $q\in \zz$ such that $q\neq 1$ and $q=\gcd(n,n+1)$. Then $q\mid n$ and $q\mid (n+1)$ so $q\mid (n+1)-n$ i.e. $q\mid 1$. Since $1\mid q$ we have $q=\pm 1$ by 2.11 (iii). This is a contradiction and thus $\gcd(n,n+1)=1$ for all integers $n$.  
\end{proof}
\emph{Alternatively,} we can do the following.
\begin{proof} Since for every integer $n$ we have $(n+1)(1)+(n)(-1)=1$ we have the $\gcd(n,n+1)=1$ by Proposition 2.27(i). 
\end{proof}
\item Prove that $\{ax+by\mid x,y\in \mathbb{Z}\}=\{n\cdot \gcd(a,b) \mid n\in \mathbb{Z}\}$
\begin{proof}We will show the following set inclusions  \begin{equation}\label{one}\{ax+by\mid x,y\in \mathbb{Z}\}\subseteq\{n\cdot \gcd(a,b) \mid n\in \mathbb{Z}\}\end{equation} and  \begin{equation}\label{two}\{n\cdot \gcd(a,b) \mid n\in~ \mathbb{Z}\}\subseteq  \{ax+by\mid x,y\in \mathbb{Z}\}.\end{equation}
To show (\ref{one}) let $z\in \{ax+by\mid x,y\in \mathbb{Z}\}$ be an arbitrary element. Then 
\[z=ax_0+by_0\ \ \ \text{ for some }x_0,y_0\in \zz.\] By Theorem 2.31 the equation 
\[z=ax+by\] has integer solutions if and only if $\gcd(a,b)\mid z$ i.e. $z=n\cdot \gcd(a,b)$ for some $n\in \zz$. Hence we have that $z\in \{n\cdot \gcd(a,b) \mid n\in \mathbb{Z}\}$. Since $z$ was arbitrary we have 
\[\{ax+by\mid x,y\in \mathbb{Z}\}\subseteq\{n\cdot \gcd(a,b) \mid n\in \mathbb{Z}\}.\]
To show (\ref{two}), let $n$ be an arbitrary integer. By the characterization of the greatest common divisor there exist $x_0, y_0\in \zz$ such that 
\[\gcd(a,b)=ax_0+by_0.\] 
Then, 
\[n\gcd(a,b)=anx_0+bny_0\in \{ax+by\mid x,y\in \mathbb{Z}\}.\]
Since $n$ was arbitrary we have that 
\[\{n\cdot \gcd(a,b) \mid n\in~ \mathbb{Z}\}\subseteq  \{ax+by\mid x,y\in \mathbb{Z}\}.\]
\end{proof}
\end{enumerate}

\end{document}









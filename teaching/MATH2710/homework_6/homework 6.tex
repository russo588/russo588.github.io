\documentclass[12pt,letterpaper]{article}
\usepackage[margin=1in]{geometry}
\usepackage{amsfonts}
\usepackage{amssymb}
\usepackage{amsthm}
\usepackage{amsmath}
\usepackage{enumerate}

%Here are some user-defined notations
\newcommand{\RR}{\mathbf R}  %bold R
\newcommand{\CC}{\mathbf C}  %bold C
\newcommand{\ZZ}{\mathbf Z}   %bold Z
\newcommand{\QQ}{\mathbf Q}   %bold Q
\newcommand{\rr}{\mathbb R}     %blackboard bold R
\newcommand{\cc}{\mathbb C}    %blackboard bold R
\newcommand{\zz}{\mathbb Z}    %blackboard bold R
\newcommand{\qq}{\mathbb Q}   %blackboard bold Q
\newcommand{\calM}{\mathcal M}  %calligraphic M
\newcommand{\sm}{\setminus} 
\newcommand{\bfa}{\mathbf a}
\newcommand{\bfb}{\mathbf b}
\newcommand{\bfc}{\mathbf c}
\newcommand{\lub}{\text{lub}}

\usepackage{tikz}
\usetikzlibrary{positioning}
\usepackage{graphicx}


%Here are some user-defined operators
\newcommand{\re}{\operatorname {Re}}
\newcommand{\im}{\operatorname {Im}}


%These commands deal with theorem-like environments (i.e., italic)
\theoremstyle{plain}
\newtheorem{theorem}{Theorem}[section]
\newtheorem{corollary}[theorem]{Corollary}
\newtheorem{lemma}[theorem]{Lemma}
\newtheorem{conjecture}[theorem]{Conjecture}

%These deal with definition-like environments (i.e., non-italic)
\theoremstyle{definition}
\newtheorem{definition}[theorem]{Definition}
\newtheorem{example}[theorem]{Example}
\newtheorem{remark}[theorem]{Remark}

%your name and date in the header.
\usepackage[us]{datetime} 
\usepackage{fancyhdr}
\pagestyle{fancy}
\lhead{}
\chead{MATH 2710\\ Homework 5}
\rhead{ Your name \\ Dec. 9th}
\lfoot{}
\cfoot{}
\rfoot{\thepage}
\renewcommand{\headrulewidth}{0 pt}
\renewcommand{\footrulewidth}{0 pt}
\begin{document}
\ \\
\begin{enumerate}[1.]
\item Let $X$ and $Y$ be sets of positive real numbers which are bounded above. Define \[XY=\{xy\, |\ x\in X,\, y\in Y\}.\] Show that $\lub(XY)=\lub(X)\cdot \lub(Y)$.  \\
\ \\

\item Show, that the sequence 
\[a_n=\frac{2n-3}{n+5}\ \ \ \ n\geq 1\] 
converges.
\ \\
\item Prove that $\left \{n^2+2\right \}_{n=1}^\infty$ diverges to infinity.       
\ \\
\item Let $\{x_n\}$ and $\{y_n\}$ be convergent sequences with limits $x$ and $y$ respectively. Prove
\begin{enumerate}[(a)]
\item $\{cx_n\}$ converges to $cx$ where $c\in \mathbb{R}$.
\item $\{x_n+y_n\}$ converges to $x+y$. 
\end{enumerate}
\ \\
\item Use the monotone convergence theorem to show the sequence $\{x_n\}$ defined by
\[x_1=\sqrt{2},\ \ x_{n+1}=\sqrt{2+x_{n}}\ \ \text{ for }n>1\]
converges.
\vspace{.15in}\\
Hint: Show by induction that the sequence is increasing and bounded above by 2.

\end{enumerate}

\end{document}









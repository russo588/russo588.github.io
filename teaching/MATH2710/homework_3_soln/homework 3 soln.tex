\documentclass[12pt,letterpaper]{article}
\usepackage[margin=1in]{geometry}
\usepackage{amsfonts}
\usepackage{amssymb}
\usepackage{amsthm}
\usepackage{amsmath}
\usepackage{enumerate}

%Here are some user-defined notations
\newcommand{\RR}{\mathbf R}  %bold R
\newcommand{\CC}{\mathbf C}  %bold C
\newcommand{\ZZ}{\mathbf Z}   %bold Z
\newcommand{\QQ}{\mathbf Q}   %bold Q
\newcommand{\rr}{\mathbb R}     %blackboard bold R
\newcommand{\cc}{\mathbb C}    %blackboard bold R
\newcommand{\zz}{\mathbb Z}    %blackboard bold R
\newcommand{\qq}{\mathbb Q}   %blackboard bold Q
\newcommand{\calM}{\mathcal M}  %calligraphic M
\newcommand{\sm}{\setminus} 
\newcommand{\bfa}{\mathbf a}
\newcommand{\bfb}{\mathbf b}
\newcommand{\bfc}{\mathbf c}




%Here are some user-defined operators
\newcommand{\re}{\operatorname {Re}}
\newcommand{\im}{\operatorname {Im}}


%These commands deal with theorem-like environments (i.e., italic)
\theoremstyle{plain}
\newtheorem{theorem}{Theorem}[section]
\newtheorem{corollary}[theorem]{Corollary}
\newtheorem{lemma}[theorem]{Lemma}
\newtheorem{conjecture}[theorem]{Conjecture}

%These deal with definition-like environments (i.e., non-italic)
\theoremstyle{definition}
\newtheorem{definition}[theorem]{Definition}
\newtheorem{example}[theorem]{Example}
\newtheorem{remark}[theorem]{Remark}

%your name and date in the header.
\usepackage[us]{datetime} 
\usepackage{fancyhdr}
\pagestyle{fancy}
\lhead{}
\chead{MATH 2710\\ Homework 3}
\rhead{ Your name \\ \today}
\lfoot{}
\cfoot{}
\rfoot{\thepage}
\renewcommand{\headrulewidth}{0 pt}
\renewcommand{\footrulewidth}{0 pt}
\begin{document}
\begin{enumerate}[1.]
\item Let $A$ be a set and define $P(A)$ to be the set of all subsets of $A$. Let $C$ be a fixed subset of the set $A$ and define relation $R$ on the set $P(A)$ by $X R Y$ if and only if $X\cap C=Y\cap C$. Prove that this is an equivalence relation.\\
\ \\
{\bf Solution: }
We will show that this is an equivalence relation.\\ 
\begin{proof} We must show that for all $X, Y, Z\in P(A)$ that 
\begin{enumerate}
\item\label{a} $X R X$ 
\item\label{b} If $X R Y$ then $Y R X$
\item\label{c} If $X R Y$ and $Y R Z$ then $X R Z$.
\end{enumerate}
To show (a), notice that $X\cap C=X\cap C$ and thus $XRX$.  To show (b) suppose that $X R Y$, then $X\cap C=Y\cap C$. Obviously, $Y\cap C=X\cap C$ so $Y R X$. To show (c), suppose that $ X R Y$ and $Y R Z$. Thus $X\cap C=Y\cap C$ and $Y\cap C=Z\cap C$. By transitivity of set equality we have that $X\cap C= Z\cap C$. Hence, $XRZ$. 
\end{proof}
\ \\
\item Let $A$ be a set and let $P$ be a partition of the set $A$ i.e. $P=\{A_1, A_2, \ldots A_n\}$ where 
\begin{enumerate}[i)]
\item $A_i\subset A$, 
\item $\emptyset \not \in P$
\item $A_1\cup A_2\cup \ldots \cup A_n=A$ 
\item $A_i\cap A_j =\emptyset $ when $i\neq j$. 
\end{enumerate}
For $x,y \in A$ we say that $x R y$ if and only if $x\in A_i$ and $y\in A_i$ for the same $i$. Prove this is an equivalence relation. \\
\ \\
{\bf Solution: }
We will show that this is an equivalence relation.\\ 
\begin{proof} We must show that for all $x, y, z\in A$ that 
\begin{enumerate}
\item\label{a} $x R x$ 
\item\label{b} If $x R y$ then $y R x$
\item\label{c} If $x R y$ and $y R z$ then $x R z$.
\end{enumerate}
To show (a) notice that if $x\in A_i$ for some $i$ that $x\in A_i$, i.e. $x R x$. To show (b) suppose that $x R y$, then $x\in A_i$ and $y\in A_i$ for the same $i$. Hence $y R x$. To show (c), suppose that $ x R y$ and $y R z$. Then $x\in A_i$ and $y\in A_i$ for some $i$ and $y\in A_j$ and $z\in A_j$ for some $j$. We must have that $A_i=A_j$ since $y\in A_i$, $y\in A_j$ and $A_i\cap A_j=\emptyset$ if $i\neq j$. Hence $x$ and $z$ lie in the same $A_i$ and thus $x R z$. 
\end{proof}
\ \\
\item Prove or disprove: The relation $R$ defined on the set $\mathbb{Z}$ by $x R y$ if and only if $xy>0$ is an equivalence relation. \\
\ \\
{\bf Solution: }
We will show that $R$ is not an equivalence relation by providing a counterexample. For $R$ to be an equivalence relation on $\mathbb{Z}$ we need that $x R x$ for all $x\in \mathbb{Z}$, i.e. 
\[x^2>0 \text{ for all }x\in \mathbb{Z}\]
However, if $x=0$ then $0^2\not > 0$ and we have a counterexample.\\
\ \\
\item Find all the $x$ that satisfy the following equation. (Hint: Use Fermat's Little theorem and notice that if $x_0$ is a solution then it's entire residue class is a solution.) 
\[x^{86} \equiv 2\ \ \  \text{(mod 7)}\]
\ \\
{\bf Solution: }
We note that $7\nmid x$ since if $x=7\cdot n$ for some $n\in \mathbb{Z}$ then 
\[x^{86}\equiv 0 \mod 7.\]
Hence, by Fermat's Little theorem 
\[x^6\equiv 1 \mod 7\]
and 
\[x^{86}=x^{6(14)+2}\equiv x^2.\]
Thus, we can reduce the problem to solving 
\[x^2\equiv 2 \mod 7.\]
Upon inspection we see that the solution sets are $[3]$ and $[4]$. \\
\ \\
\item Prove that every integer of the form $5n+3$ for $n\in \mathbb{Z}$, $n\geq 1$, cannot be a perfect square.  \\
\ \\
{\bf Solution: }
\begin{proof}
Suppose for the sake of contradiction that there exists an integer $q$ such that 
\[q^2=5n+3 \text{ for some }n\in \mathbb{Z}\]
Thus, 
\[[q^2]=[q]^2=3 \mod 5\]
However, 
\[[0]^2=0 \mod 5\]
\[[1]^2=1 \mod 5\]
\[[2]^2=4 \mod 5\]
\[[3]^2=4 \mod 5\]
\[[4]^2=1 \mod 5\]
Thus, we arrive at a contradiction. 
\end{proof}
\end{enumerate}
{\bf Bonus Question: (+3 Points added to exam)}\\
\ \\
Jim is looking to have a easy life and make a lot of money. Jim goes looking for employment and finds a mysterious man. The man points to a bridge and says the following to Jim: ``The work I have for you is light and you will get rich. Do you see the bridge? Each time you cross it I will double the money in your pocket. But since I am so generous you must give me back \$ 24 after each crossing.'' Jim accepts and walks across the bridge. Miraculously the money in his pocket doubled! He threw \$ 24 dollars to the mystery man for the first crossing and crossed again.  Amazingly his money doubled! He paid the mystery man \$ 24 again for the second crossing. He crossed a third time, again his money doubles. He goes to pay the mystery man, but the mystery man laughs because Jim only had \$ 24 dollars in his pocket and had to give it all away. How much money did Jim start with ?\\
\ \\
{\bf Solution: } \$21
\end{document}









\documentclass[12pt]{article}
\usepackage[margin=1in]{geometry}
\usepackage{amsfonts}
\usepackage{amssymb}
\usepackage{amsthm}
\usepackage{amsmath}
\usepackage{enumerate}

%Here are some user-defined notations
\newcommand{\RR}{\mathbf R}  %bold R
\newcommand{\CC}{\mathbf C}  %bold C
\newcommand{\ZZ}{\mathbf Z}   %bold Z
\newcommand{\QQ}{\mathbf Q}   %bold Q
\newcommand{\rr}{\mathbb R}     %blackboard bold R
\newcommand{\cc}{\mathbb C}    %blackboard bold R
\newcommand{\zz}{\mathbb Z}    %blackboard bold R
\newcommand{\qq}{\mathbb Q}   %blackboard bold Q
\newcommand{\calM}{\mathcal M}  %calligraphic M
\newcommand{\sm}{\setminus} 
\newcommand{\bfa}{\mathbf a}
\newcommand{\bfb}{\mathbf b}
\newcommand{\bfc}{\mathbf c}




%Here are some user-defined operators
\newcommand{\re}{\operatorname {Re}}
\newcommand{\im}{\operatorname {Im}}
\newcommand{\lub}{\operatorname {lub}}
\newcommand{\glb}{\operatorname {glb}}


%These commands deal with theorem-like environments (i.e., italic)
\theoremstyle{plain}
\newtheorem{theorem}{Theorem}[section]
\newtheorem{corollary}[theorem]{Corollary}
\newtheorem{proposition}[theorem]{Proposition}
\newtheorem{lemma}[theorem]{Lemma}
\newtheorem{conjecture}[theorem]{Conjecture}

%These deal with definition-like environments (i.e., non-italic)
\theoremstyle{definition}
\newtheorem{definition}[theorem]{Definition}
\newtheorem{example}[theorem]{Example}
\newtheorem{remark}[theorem]{Remark}

%your name and date in the header.
\usepackage[us]{datetime} 
\usepackage{fancyhdr}
\pagestyle{fancy}
\fancyhf{}
\lhead{Homework 1}
\chead{MTH 354}
\rhead{Your name here}
\lfoot{\ }
\cfoot{\ }
\rfoot{\thepage}
%\renewcommand{\headrulewidth}{0 pt}
%\renewcommand{\footrulewidth}{0 pt}
\begin{document}
\begin{enumerate}[1.]
\item In this question I will guide you through the proof of the following proposition:
\begin{proposition}
 If $y\geq 0$ and $n\in \mathbb{N}^+$ then there is a unique non-negative real number $s$ such that $s^n=y$. In particular, the function $f:[0,\infty)\rightarrow [0,\infty)$ defined by $f(x)=x^n$ is one to one and onto. 
\end{proposition}
\begin{enumerate}[a)]
\item Prove the following two lemmas that are used throughout the rest of the proof.
\begin{lemma}
If $x<y$ then $xz<yz$ for all $x,y,z\in \mathbb{R}^+$
\end{lemma}
\begin{lemma}
If $0<x<1$ then $x^n<x$ for all $n\in \mathbb{N}^+$ and if $x>1$ then $x^n>x$ for all $n\in \mathbb{N}^+$.\\
\end{lemma}
\item Let $S=\{x\in \mathbb{R}: x>0 \text{ and }x^n<y\}$, $\alpha=\min\{1,y\}$ and $\beta=\max\{1,y\}$. Show that if $x<\alpha$ then $x\in S$ and that $\beta$ is an upper-bound for $S$. From this information show that $S$ has a least upper bound, call it $s$. We will show that this is the root for $y$.
\ \\
\item Show that if $t>0$ and $y<t^n$, then $t$ is an upper bound for $S$. 
\ \\
\item Show that if $t>0$ and $y<t^n$, then there is a $v\in (0,t)$ such that $y<v^n$. Hence, $v<t$ and $v$ is an upper bound for $S$. In particular, $t$ is not the least upper bound for $S$ and therefore $s^n\leq y$. \\
{\noindent\bf Hint:} Suppose for the sake of contradiction that $v^n<y$ for all $v\in (0,t)$. Thus, $v^n<y<t^n$ and hence $0<|t^n-y|<|t^n-v^n|$ (why?). Then conclude that this is a contradiction based on the fact that for a fixed $t$ and $v\in (0,t)$ 
\[|t^n-v^n|\leq |t-v|\cdot M \text{ for some }M\in\mathbb{R}^+\]
(state why this is contradiction). 
\ \\
\item Show that if $t>0$ and $t^n<y$ then there exists a $v$ such that $0<t<v$ and $v^n<y$. Use this to conclude that $t$ is not an upper bound for $S$ and thus $s^n\geq y$. In combination with the above parts conclude that $s^n=y$
\end{enumerate}
 \end{enumerate}


\end{document}



\documentclass[11pt]{exam}
\RequirePackage{amssymb, amsfonts, amsmath, latexsym, verbatim, xspace, setspace}
\RequirePackage{tikz, pgflibraryplotmarks}
\usepackage[margin=1in]{geometry}
\usepackage{amsmath, amsthm, amssymb}
\newtheorem*{thm}{{\bf Theorem}}
\newtheorem{lemma}{{\bf Lemma}}
\newcommand{\A}{\mathfrak{A}}
\theoremstyle{definition}
\newtheorem{define}{Definition}
\newtheorem{claim}{Claim}
\newtheorem*{method}{Method}
\newtheorem{ex}{Example}
\newcommand{\dydx}{\dfrac{dy}{dx}}
\newcommand{\dydt}{\dfrac{dy}{dt}}
\newcommand{\dxdt}{\dfrac{dx}{dt}}
\newcommand{\dxdy}{\dfrac{dx}{dy}}
\newcommand{\pp}{\prime\prime}
\newcommand{\p}{\prime}
\renewcommand{\d}[2]{\dfrac{d#1}{d#2}}
\newcommand{\dd}[2]{\dfrac{d^2#1}{d#2^2}}
\newcommand{\ypp}{y^{\prime\prime}}
\newcommand{\yp}{y^{\prime}}
\newcommand{\tr}{\text{tr}}
\renewcommand\thesection{2.5}
\usepackage{xcolor}
\usepackage{graphicx}
\usepackage{lipsum}% Used for dummy text.
\usepackage{enumerate}
\DeclareMathOperator{\dimension}{dim}
\DeclareMathOperator{\ran}{ran}
\DeclareMathOperator{\nullspc}{null}

% Here's where you edit the Class, Exam, Date, etc.
\newcommand{\class}{MATH3210}
\newcommand{\term}{Spring 2017}
\newcommand{\examnum}{Exam 2}
\newcommand{\examdate}{Due: 2/20/17}


% For an exam, single spacing is most appropriate
\singlespacing
% \onehalfspacing
% \doublespacing

% For an exam, we generally want to turn off paragraph indentation
\parindent 0ex

\begin{document} 

% These commands set up the running header on the top of the exam pages
\pagestyle{head}
\firstpageheader{}{}{}
\runningheader{\class}{\examnum\ - Page \thepage\ of \numpages}{\examdate}
\runningheadrule
 {\bf \class} \\
\ \\
 {\bf \examnum} \\\

\rule[1ex]{\textwidth}{.1pt}
\ \\
The following rules apply:\\

\begin{itemize}
\item \textbf{Exam must be typed}. Please organize your proofs in a reasonably neat and coherent way. Write in complete sentences.  

\item \textbf{Mysterious or unsupported claims will not receive full
credit}.  Unreasonably large gaps in logic or an argument will receive little credit. You may quote theorems from class or the book.

\item \textbf{Your solutions must be your own.} You may use outside sources but your submitted solution must be in your own words. 
\end{itemize}

\newpage % End of cover page


\begin{questions}
\question Show that the determinant of a matrix is the product of its eigenvalues.\vspace{-.1in}\\

{\bf Hint: }You can use Corollary 6 in the determinant notes. 
\vfill
\question For $u\in V$, let $\varphi_u$ denote the linear functional on the inner product space $V$ defined by 
\[\varphi_u(v)=\langle v,u\rangle\] 
for $v\in V$. 
\begin{enumerate}[(a)]
\item Show that if $\mathbb{F}=\mathbb{R}$ then the map $\Phi: V(\mathbb{R})\rightarrow V(\mathbb{R})^\prime$ defined by 
\[\Phi(u)=\varphi_u\]
is a linear map. 
\item Show that if $\mathbb{F}=\mathbb{C}$ and $V(\mathbb{C})\neq \{0\}$, then $\Phi$ is not linear. 
\item Suppose that $\mathbb{F}=\mathbb{R}$ and $V(\mathbb{R})$ is finite dimensional. Show $\Phi$ is an isomorphism. 
\end{enumerate}
\vfill
\question Let $P(\mathbb{R})$ be the vector space of polynomials and $\mathbb{R}^\infty$ be the vector space of sequences of real numbers (page 13). Show that $P(\mathbb{R})^\prime$ and $\mathbb{R}^\infty$ are isomorphic. \vspace{-.1in}\\

{\bf Hint: }Construct an explicit isomorphism between $P_n(\mathbb{R})^\prime$ and $\mathbb{R}^{n+1}$ and extend this in the natural way to $P(\mathbb{R})$ and $\mathbb{R}^\infty$. 
\vfill
\question Consider the space $\mathbb{C}^\infty$. Let $B:\mathbb{C}^\infty\rightarrow \mathbb{C}^\infty$ be defined by the following:
\[B(x_1, x_2, x_3,\ldots) =(0, x_1, x_2, \ldots).\]
Does $B$ have eigenvalues? Rectify your answer with Theorem 5.21 (page 145).
\vfill
\question Let $M_{2\times 2}(\mathbb{R})$ denote the vector space of $2\times 2$ matrices with real entries.  Define the trace of a matrix $A$  as the sum of the diagonal entries, i.e. 
\[\tr(A)=a_{1,1}+a_{2,2}\]
Show that 
\[\langle A, B\rangle=\tr(B^\top A)\]
is an inner product on this space. 
\vfill
\question Show via Cauchy-Schwarz that 
\[\left(\frac{a_1+\ldots +a_n}{n}\right)^2\leq \frac{a_1^2+\ldots+a_n^2}{n}\]
i.e. the square of an average is less than or equal to the average of the squares. 
\question Suppose that $V$ is finite dimensional and $U$ is a subspace of $V$. Show that 
\[P_{U^\bot}=I-P_U,\]
where $I$ is the identity operator on $V$. 
\vfill
\end{questions}

\end{document}


%%%%%%%%%%%%%%%%%%%%%%%%%%%%%%%%%%%%%%%%%%%%%%%
% Basic question
\addpoints
\question[10] Differentiate $f(x)=x^2$ with respect to $x$.

% Question with parts
\newpage
\addpoints
\question Consider the function $f(x)=x^2$.
\begin{parts}
\part[5] Find $f'(x)$ using the limit definition of derivative.
\vfill
\part[5] Find the line tangent to the graph of $y=f(x)$ at the point where $x=2$.
\vfill
\end{parts}

% If you want the total number of points for a question displayed at the top,
% as well as the number of points for each part, then you must turn off the point-counter
% or they will be double counted.
\newpage
\addpoints
\question[10] Consider the function $f(x)=x^3$.
\noaddpoints % If you remove this line, the grading table will show 20 points for this problem.
\begin{parts}
\part[5] Find $f'(x)$ using the limit definition of derivative.
\vspace{4.5in}
\part[5] Find the line tangent to the graph of $y=f(x)$ at the point where $x=2$.
\end{parts}
\documentclass[11pt]{exam}
\RequirePackage{amssymb, amsfonts, amsmath, latexsym, verbatim, xspace, setspace}
\RequirePackage{tikz, pgflibraryplotmarks}
\usepackage[margin=1in]{geometry}
\usepackage{amsmath, amsthm, amssymb}
\newtheorem*{thm}{{\bf Theorem}}
\newtheorem{lemma}{{\bf Lemma}}
\newcommand{\A}{\mathfrak{A}}
\theoremstyle{definition}
\newtheorem{define}{Definition}
\newtheorem{claim}{Claim}
\newtheorem*{method}{Method}
\newtheorem{ex}{Example}
\newcommand{\dydx}{\dfrac{dy}{dx}}
\newcommand{\dydt}{\dfrac{dy}{dt}}
\newcommand{\dxdt}{\dfrac{dx}{dt}}
\newcommand{\dxdy}{\dfrac{dx}{dy}}
\newcommand{\pp}{\prime\prime}
\newcommand{\p}{\prime}
\renewcommand{\d}[2]{\dfrac{d#1}{d#2}}
\newcommand{\dd}[2]{\dfrac{d^2#1}{d#2^2}}
\newcommand{\ypp}{y^{\prime\prime}}
\newcommand{\yp}{y^{\prime}}
\newcommand{\tr}{\text{tr}}
\renewcommand\thesection{2.5}
\usepackage{xcolor}
\usepackage{graphicx}
\usepackage{lipsum}% Used for dummy text.
\usepackage{enumerate}
\DeclareMathOperator{\dimension}{dim}
\DeclareMathOperator{\ran}{ran}
\DeclareMathOperator{\nullspc}{null}

% Here's where you edit the Class, Exam, Date, etc.
\newcommand{\class}{MATH3210}
\newcommand{\term}{Spring 2017}
\newcommand{\examnum}{Exam 2}
\newcommand{\examdate}{Due: 2/20/17}


% For an exam, single spacing is most appropriate
\singlespacing
% \onehalfspacing
% \doublespacing

% For an exam, we generally want to turn off paragraph indentation
\parindent 0ex

\begin{document} 

% These commands set up the running header on the top of the exam pages
\pagestyle{head}
\firstpageheader{}{}{}
\runningheader{\class}{\examnum\ - Page \thepage\ of \numpages}{\examdate}
\runningheadrule
 {\bf \class} \\
\ \\
 {\bf \examnum} \\\

\rule[1ex]{\textwidth}{.1pt}
\ \\
The following rules apply:\\

\begin{itemize}
\item \textbf{Exam must be typed}. Please organize your proofs in a reasonably neat and coherent way. Write in complete sentences.  

\item \textbf{Mysterious or unsupported claims will not receive full
credit}.  Unreasonably large gaps in logic or an argument will receive little credit. You may quote theorems from class or the book.

\item \textbf{Your solutions must be your own.} You may use outside sources but your submitted solution must be in your own words. 
\end{itemize}

\newpage % End of cover page


\begin{questions}
\question Show that the determinant of a matrix is the product of its eigenvalues.\vspace{-.1in}\\

{\bf Hint: }You can use Corollary 6 in the determinant notes. 
\begin{proof} Let $A$ be a a square $n\times n$ matrix. Let $L_A:\mathbb{C}^n\rightarrow \mathbb{C}^n$ denote the operator given by multiplication by $A$. We note under the standard basis $\mathcal{E}=\{e_1,\ldots, e_n\}$ of $\mathbb{C}^n$ that 
\[[L_A]_{\mathcal{E}}=A.\]
We note that $L_A\in \mathcal{L}(\mathbb{C}^n)$ has an upper triangular matrix with respect to some basis $\mathcal{B}=\{b_1,\ldots, b_n\}$ of $\mathbb{C}^n$. We note that the eigenvalues of $L_A$ (and hence $A$) are the diagonal entries of $[L_A]_\mathcal{B}$. For convenience let $B=[L_A]_\mathcal{B}$.  By corollary 6, we have that 
\[A=P^{-1}BP\] for some invertible matrix $P$. Moreover, 
\[\det(A)=\det(P^{-1}BP)=\det(P^{-1})\det(B)\det(P)=\det(B).\]
Since the eigenvalues of $L_A$ (and thus $A$) are the diagonal entries of $B$ we note that 
\[\det(A)=\det(B)=\prod_{i=1}^n\lambda_n\]
since $B$ is upper triangular and where $\lambda_i$, $i\in \{1,\ldots, n\}$, are the eigenvalues of $A$. 
\end{proof}

\question For $u\in V$, let $\varphi_u$ denote the linear functional on the inner product space $V$ defined by 
\[\varphi_u(v)=\langle v,u\rangle\] 
for $v\in V$. 
\begin{enumerate}[(a)]
\item Show that if $\mathbb{F}=\mathbb{R}$ then the map $\Phi: V(\mathbb{R})\rightarrow V(\mathbb{R})^\prime$ defined by 
\[\Phi(u)=\varphi_u\]
is a linear map. 
\item Show that if $\mathbb{F}=\mathbb{C}$ and $V(\mathbb{C})\neq \{0\}$, then $\Phi$ is not linear. 
\item Suppose that $\mathbb{F}=\mathbb{R}$ and $V(\mathbb{R})$ is finite dimensional. Show $\Phi$ is an isomorphism. 
\end{enumerate}
\begin{proof}\ \\
\begin{enumerate}[(a)] 
\item Let $v\in V$ and consider $\Phi(\lambda u+w)$ where $u, w\in V$ and $\lambda\in \mathbb{R}$. 
\begin{align*}(\Phi(\lambda u+w))(v)&=\varphi_{\lambda u+w}(v)\\
&=\langle v, \lambda u+w\rangle\\
&=\lambda \langle u,v\rangle +\langle w,v\rangle\\
&=\lambda \varphi_u(v)+\varphi_w(v)\\
&=(\lambda \Phi(u)+\Phi(w))(v)
\end{align*}
\item Suppose $\mathbb{F}=\mathbb{C}$ and $V(\mathbb{C})\neq \{0\}$. Since $V\neq \{0\}$, let $u,v\in V$ and consider $(\Phi(iu))(v)$. We have that 
\[(\Phi(iu))(v)=\varphi_{iu}(v)=\langle v, iu\rangle=-\i\langle v,u\rangle=-i(\varphi_u)(v)=-i(\Phi(u))(v).\]
Hence $\Phi$ is not linear. 
\item We need only show that the map is injective since $\text{dim}(V)=\text{dim}(V^\prime)$ by application of the fundamental theorem of linear maps the map is automatically surjective. Suppose $u,w\in V$ and $\Phi(u)=\Phi(w)$. We have that 
$(\Phi(u))(v)=(\Phi(w))(v)$ for all $v\in V$. Hence, 
\[\langle v,u-w\rangle=0\] for all $v\in V$ and in particular for $v=u-w$. Therefore $\|u-w\|=0$ and we have that $u=w$. 
\end{enumerate}
\end{proof}
\question Let $P(\mathbb{R})$ be the vector space of polynomials and $\mathbb{R}^\infty$ be the vector space of sequences of real numbers (page 13). Show that $P(\mathbb{R})^\prime$ and $\mathbb{R}^\infty$ are isomorphic. \vspace{-.1in}\\

{\bf Hint: }Construct an explicit isomorphism between $P_n(\mathbb{R})^\prime$ and $\mathbb{R}^{n+1}$ and extend this in the natural way to $P(\mathbb{R})^\prime$ and $\mathbb{R}^\infty$. 
\begin{proof} We construct the following isomorphism. Let $\Gamma: P(\mathbb{R})^\prime\rightarrow \mathbb{R}^\infty$ be defined by 
\[\varphi\mapsto (\varphi(1), \varphi(x), \varphi(x^2), \ldots).\]
We will show that $\Gamma$ is linear and a bijection. Let $\varphi, \psi\in P(\mathbb{R})^\prime$ and $\lambda \in \mathbb{R}$. Consider 
\begin{align*}
\Gamma(\lambda \varphi+\psi)&=((\lambda \varphi+\psi)(1),(\lambda \varphi+\psi)(x),(\lambda \varphi+\psi)(x^2),\ldots)\\
&= ((\lambda \varphi)(1),(\lambda \varphi)(x),(\lambda \varphi)(x^2),\ldots)+(\psi(1),\psi(x),\psi(x^2),\ldots)\\
&= \lambda(\varphi(1),\varphi(x),\varphi(x^2),\ldots)+(\psi(1),\psi(x),\psi(x^2),\ldots)\\
&=\lambda \Gamma(\varphi)+\Gamma(\psi)
\end{align*}
This shows that $\Gamma$ is linear. We now show surjectivity. Suppose $(a_0, a_1,a_2,\ldots)\in \mathbb{R}^\infty$ and define $\hat{\varphi}\in P(\mathbb{R})^\prime$ by 
\[\hat{\varphi}(c_0+c_1x+\ldots+c_nx^n)=c_0a_0+c_1a_1+\ldots +c_na_n.\]
Hence we have that, 
\[\hat{\varphi}(x^m)=a_m\]
by definition and $\hat{\varphi}\in P(\mathbb{R})^\prime$. Moreover, 
\[\Gamma(\hat{\varphi})=(a_0,a_1,a_2,\ldots).\] We now show that $\Gamma$ is injective. Suppose that $\Gamma(\varphi)=(0,0,\ldots)$ for some $\varphi\in P(\mathbb{R})^\prime$. Hence $\varphi(x^m)=0$ for all $m\in \mathbb{N}$. By linearity of $\varphi$ we have that $\varphi(p)=0$ for all $p\in P(\mathbb{R})$. Thus $\varphi$ is the zero functional on $P(\mathbb{R})$ and $\Gamma$ is injective.
\end{proof}
\question Consider the space $\mathbb{C}^\infty$. Let $B:\mathbb{C}^\infty\rightarrow \mathbb{C}^\infty$ be defined by the following:
\[B(x_1, x_2, x_3,\ldots) =(0, x_1, x_2, \ldots).\]
Does $B$ have eigenvalues? Rectify your answer with Theorem 5.21 (page 145).
\begin{proof} The eigenvalue condition implies the following: 
\[\lambda x_1, \lambda x_2, \ldots)=(0, x_1, x_2,\ldots).\]
Which gives us the following set of equations. 
\[\lambda x_1=0\]
\[\lambda x_2=x_1\]
\[\vdots\]
\[\lambda x_n=x_{n-1}\]
\[\vdots\]
If $\lambda =0$ then $0=x_1=x_2=\ldots$ and $(x_1,x_2,\ldots)=(0,0,\ldots)$ which is not possible since the zero vector cannot be an eigenvector. If $\lambda\neq 0$ then $x_1=\frac{0}{\lambda}$, $x_n=\frac{x_{n-1}}{\lambda}$ and $(x_1,x_2,\ldots)=(0,0,\ldots)$. To rectify this with our prior knowledge we need only to note that these are infinite dimensional vector spaces and the theorem does not apply. 
\end{proof}
\question Let $M_{2\times 2}(\mathbb{R})$ denote the vector space of $2\times 2$ matrices with real entries.  Define the trace of a matrix $A$  as the sum of the diagonal entries, i.e. 
\[\tr(A)=a_{1,1}+a_{2,2}\]
Show that 
\[\langle A, B\rangle=\tr(B^\top A)\]
is an inner product on this space. 
\begin{proof}
We must show the following:
\begin{enumerate}[(a) ]
\item $\langle A,A\rangle\geq 0$ for all $A\in M_{2\times 2}(\mathbb{R})$
\item $\langle A,A\rangle=0$ if and only if $A=0$
\item $\langle A+B, C\rangle =\langle A, C\rangle +\langle B, C\rangle$ for all $A, B, C\in M_{2\times 2}(\mathbb{R})$
\item $\langle \lambda A, B\rangle =\lambda \langle A, B\rangle$  for all $\lambda \in \mathbb{R}$ and $A,B \in M_{2\times 2}(\mathbb{R})$
\item $\langle A,B\rangle=\langle B,A\rangle$ for all $A,B\in M_{2\times 2}(\mathbb{R})$. 
\end{enumerate}
Let 
\[A=\begin{pmatrix}a_{1,1} & a_{1,2}\\a_{2,1}& a_{2,2}\end{pmatrix} \quad \text{ and }\quad B=\begin{pmatrix}b_{1,1} & b_{1,2}\\b_{2,1}& b_{2,2}\end{pmatrix}.\]
We have that 
\[B^\top A=\begin{pmatrix}b_{1,1}a_{1,1} +b_{2,1}a_{2,1}& b_{1,1}a_{1,2}+b_{2,1}a_{2,2}\\b_{12}a_{1,1}+b_{2,2}a_{2,1}& b_{1,2}a_{1,2}+b_{2,2}a_{2,2}\end{pmatrix}\]
and 
\[\tr(B^\top A)=b_{1,1}a_{1,1}+b_{2,1}a_{2,1}+b_{1,2}a_{1,2}+b_{2,2}a_{2,2}.\]
To prove (a) and (b) note that $\tr(A^\top A)=a_{1,1}^2+a_{2,1}^2+a_{1,2}^2+a_{2,2}^2\geq 0$ and that $\tr(A^\top A)=0$ if and only if $a_{i,j}=0$ for $i,j\in\{1,2\}$.
To prove (c) note
\begin{align*}
\tr (C^\top (A+B))&=c_{1,1}(a_{1,1}+b_{1,1})+c_{2,1}(a_{2,1}+b_{2,1})+c_{1,2}(a_{1,2}+b_{1,2})+c_{2,2}(a_{2,2}+b_{2,2})\\
& =c_{1,1}a_{1,1}+c_{1,1}b_{1,1}+c_{2,1}a_{2,1}+c_{2,1}b_{2,1}+c_{1,2}a_{1,2}+c_{1,2}b_{1,2}+c_{2,2}a_{2,2}+c_{2,2}b_{2,2}\\
&=c_{1,1}a_{1,1}+c_{2,1}a_{2,1}+c_{1,2}a_{1,2}+c_{2,2}a_{2,2}+c_{1,1}b_{1,1}+c_{2,1}b_{2,1}+c_{1,2}b_{1,2}+c_{2,2}b_{2,2}\\
&=\tr (C^\top A)+\tr (C^\top B).
\end{align*}
To prove (d) note 
\begin{align*}
\tr(B^\top (\lambda A))&=b_{1,1}\lambda a_{1,1}+b_{2,1}\lambda a_{2,1}+b_{1,2}\lambda a_{1,2}+b_{2,2}\lambda a_{2,2}\\
&=\lambda(b_{1,1}a_{1,1}+b_{2,1}a_{2,1}+b_{1,2}a_{1,2}+b_{2,2}a_{2,2})\\
&=\lambda \tr(B^\top A).
\end{align*}
Finally to prove (e) note
\[\tr(B^\top A)=b_{1,1}a_{1,1}+b_{2,1}a_{2,1}+b_{1,2}a_{1,2}+b_{2,2}a_{2,2}=\tr(A^\top B).\]
\end{proof}
\question Show via Cauchy-Schwarz that 
\[\left(\frac{a_1+\ldots +a_n}{n}\right)^2\leq \frac{a_1^2+\ldots+a_n^2}{n}\]
i.e. the square of an average is less than or equal to the average of the squares. 
\begin{proof}
Let ${\bf a}=(a_1,\ldots, a_n)$ and ${\bf n}=(\frac{1}{n},\frac{1}{n},\ldots \frac{1}{n})$.
We note 
\[\left|\langle {\bf a},{\bf n}\rangle\right|^2=\left(\frac{a_1+\ldots +a_n}{n}\right)^2\]
\[\|{\bf a}\|^2=a_1^2+\ldots +a_n^2,\]
and
\[\|{\bf n}\|^2=\left(\frac{1}{n}\right)^2+\ldots +\left(\frac{1}{n}\right)^2=\frac{n}{n^2}=\frac{1}{n}.\]
 By Cauchy-Schwarz
\[\left|\langle {\bf a},{\bf n}\rangle\right|^2\leq \|{\bf a}\|^2\|{\bf n}\|^2\]
and 
\[\left(\frac{a_1+\ldots +a_n}{n}\right)^2\leq \frac{a_1^2+\ldots+a_n^2}{n}.\]
\end{proof}
\question Suppose that $V$ is finite dimensional and $U$ is a subspace of $V$. Show that 
\[P_{U^\bot}=I-P_U,\]
where $I$ is the identity operator on $V$. 
\begin{proof} Suppose that $v\in V$. We note that $v=u+w$ where $u\in U$ and $w\in U^\bot$. By definition of $P_U$ we have that 
\[P_U(v)=u.\]
Since $U^{\bot\bot}=U$ we have that 
\[P_{U^\bot}(v)=w=v-u=v-P_u(v)=(I-P_U)v\]
as desired. 
\end{proof}
\end{questions}

\end{document}


%%%%%%%%%%%%%%%%%%%%%%%%%%%%%%%%%%%%%%%%%%%%%%%
% Basic question
\addpoints
\question[10] Differentiate $f(x)=x^2$ with respect to $x$.

% Question with parts
\newpage
\addpoints
\question Consider the function $f(x)=x^2$.
\begin{parts}
\part[5] Find $f'(x)$ using the limit definition of derivative.
\vfill
\part[5] Find the line tangent to the graph of $y=f(x)$ at the point where $x=2$.
\vfill
\end{parts}

% If you want the total number of points for a question displayed at the top,
% as well as the number of points for each part, then you must turn off the point-counter
% or they will be double counted.
\newpage
\addpoints
\question[10] Consider the function $f(x)=x^3$.
\noaddpoints % If you remove this line, the grading table will show 20 points for this problem.
\begin{parts}
\part[5] Find $f'(x)$ using the limit definition of derivative.
\vspace{4.5in}
\part[5] Find the line tangent to the graph of $y=f(x)$ at the point where $x=2$.
\end{parts}
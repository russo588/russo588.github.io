\documentclass[12pt,letterpaper]{article}
\usepackage[margin=1in]{geometry}
\usepackage{amsfonts}
\usepackage{amssymb}
\usepackage{amsthm}
\usepackage{amsmath}
\usepackage{enumerate}

%Here are some user-defined notations
\newcommand{\RR}{\mathbf R}  %bold R
\newcommand{\CC}{\mathbf C}  %bold C
\newcommand{\ZZ}{\mathbf Z}   %bold Z
\newcommand{\QQ}{\mathbf Q}   %bold Q
\newcommand{\rr}{\mathbb R}     %blackboard bold R
\newcommand{\cc}{\mathbb C}    %blackboard bold R
\newcommand{\zz}{\mathbb Z}    %blackboard bold R
\newcommand{\qq}{\mathbb Q}   %blackboard bold Q
\newcommand{\calM}{\mathcal M}  %calligraphic M
\newcommand{\sm}{\setminus} 
\newcommand{\bfa}{\mathbf a}
\newcommand{\bfb}{\mathbf b}
\newcommand{\bfc}{\mathbf c}




%Here are some user-defined operators
\newcommand{\re}{\operatorname {Re}}
\newcommand{\im}{\operatorname {Im}}


%These commands deal with theorem-like environments (i.e., italic)
\theoremstyle{plain}
\newtheorem{theorem}{Theorem}[section]
\newtheorem{corollary}[theorem]{Corollary}
\newtheorem{lemma}[theorem]{Lemma}
\newtheorem{conjecture}[theorem]{Conjecture}

%These deal with definition-like environments (i.e., non-italic)
\theoremstyle{definition}
\newtheorem{definition}[theorem]{Definition}
\newtheorem{example}[theorem]{Example}
\newtheorem{remark}[theorem]{Remark}

%your name and date in the header.
\usepackage[us]{datetime} 
\usepackage{fancyhdr}
\pagestyle{fancy}
\lhead{}
\chead{MATH 3210\\ Homework 1}
\rhead{ Your name \\ \today}
\lfoot{}
\cfoot{}
\rfoot{\thepage}
\renewcommand{\headrulewidth}{0 pt}
\renewcommand{\footrulewidth}{0 pt}
\begin{document}
\begin{enumerate}[1.]
 \item Let $\qq(\sqrt{2})=\{a+b\sqrt{2} : a,b\in \mathbb{Q}\}$.  Note that $\qq(\sqrt{2})$ is field and more specifically it is known as an algebraic number field. The binary operations on $\qq(\sqrt{2})$ are the standard addition and multiplication of numbers. Verify for all $\alpha\neq 0$ in $\qq(\sqrt{2})$ that there exists a $\beta\in \qq(\sqrt{2})$ such that $\alpha \cdot \beta = 1$. \\
 \ \\
 {\bf Solution:} Consider $\alpha=a+b\sqrt{2}\neq0$, where $a,b\in \mathbb{Q}$. Let 
 \[\beta=\frac{1}{\alpha}=\frac{1}{a+b\sqrt{2}}\cdot \left(\frac{a-b\sqrt{2}}{a-b\sqrt{2}}\right)=\left(\frac{a}{a^2-2b^2}\right)-\left(\frac{b}{a^2-2b^2}\right)\sqrt{2}\in \qq(\sqrt{2}).\]
 Clearly, 
 \[\alpha\cdot \beta = \frac{a^2+ab\sqrt{2}-ab\sqrt{2}-2b^2}{a^2-2b^2}=1.\]
 Note that if $a+b\sqrt{2}\neq 0$ where $b\neq 0$ then $a-b\sqrt{2}\neq 0$ (otherwise this implies $\sqrt{2}=\frac{a}{b}$) and \[a^2-2b^2=\left(a+b\sqrt{2}\right)\cdot \left(a-b\sqrt{2}\neq 0\right)\neq 0.\]
 \ \\
 \hrule
 For the next two problems let $\mathbb{F}$ be an arbitrary field. We define the following vector space over $\mathbb{F}$. Let
\[\mathbb{F}^n=\{(x_1,x_2,\ldots, x_n) : x_j\in \mathbb{F},\ j=1, \ldots n\}\] where scalar multiplication and vector addition is defined thusly, 
\[\lambda \cdot (x_1, \ldots , x_n)=(\lambda x_1, \ldots , \lambda x_n)\]
\[(x_1,\ldots ,x_n)+(y_1,\ldots ,y_n)=(x_1+y_1, \ldots x_n+y_n).\]
\hrule
\ \\
\item (\#13 \S 1.A) Show that $(ab)x=a(bx)$ for all $x\in \mathbb{F}^n$ and all $a,b\in \mathbb{F}$. 
\begin{proof} Note,
\begin{align*}(ab)x&=(ab)\cdot (x_1,x_2,\ldots, x_n)\\
&=((ab)x_1,(ab)x_2,\ldots, (ab)x_n)\\
&=(a(b x_1),a(b x_2),\ldots, a(b x_n))\\
&=a(bx_1,bx_2,\ldots, bx_n)\\
&=a(bx).
\end{align*}
The above calculation relies on the definition of the scalar multiplication and from the associativity of the field multiplication. 
\end{proof}
\newpage
\item (\# 15\S 1.A) Show that $\lambda \cdot (x+y)=\lambda x+\lambda y$ for all $\lambda \in \mathbb{F}$ and all $x,y\in \mathbb{F}^n$. 
\begin{proof} Note,
\begin{align*}
\lambda \cdot (x+y)&=\lambda \cdot \left((x_1,\ldots ,x_n)+(y_1,\ldots ,y_n)\right)\\
&=\lambda \cdot (x_1+y_1, \ldots x_n+y_n)\\
&=(\lambda(x_1+y_1), \ldots \lambda(x_n+y_n))\\
&=(\lambda x_1+\lambda y_1, \ldots \lambda x_n+\lambda y_n)\\
&=(\lambda x_1, \ldots \lambda x_n)+(\lambda y_1, \ldots \lambda y_n)\\
&=\lambda x +\lambda y.
\end{align*}
The above calculation relies on the definition of the binary operations and from the distribution property of the field multiplication.
\end{proof}
\ \\
\hrule 
\ \\
For the next two problems let $\mathbb{F}$ be an arbitrary field and $V$ a vector space over $\mathbb{F}$.
\ \\
\hrule
\ \\ 
\item (\#1 \S 1.B) Prove that $-(-v)=v$ for every $v\in V$. 
\begin{proof}Note, 
\begin{align*}
-(-v)=-((-1)\cdot v)&=(-1)\cdot((-1)\cdot v)\\
&=(-1)(-1)\cdot (v)\\
&=1\cdot v\\
&=v
\end{align*}
The above calculation is done by two applications on Proposition 1.31 (pg 17). As an aside, if $1$ is the multiplicative identity in the field and $-1$ is the additive inverse of 1, then 
\[(-1)(-1+1)=(-1)(0)=0.\]
So 
\[((-1)(-1)+(-1))=0.\]
This would show that $(-1)(-1)=1$ if we also showed that the additive inverses in a field are unique and that $a0=0$ for all $a\in \mathbb{F}$. However, this was not necessary for the problem. 
\end{proof}
\item (\#2 \S 1.B) Suppose $a\in \mathbb{F}$, $v\in V$, and $av=0$. Prove $a=0$ or $v=0$. 
\begin{proof}
Suppose $a\neq 0$ and show that $v=0$. If $a\neq 0$ then there exists a unique multiplicative inverse element in the field, call it $a^{-1}$. If 
\[av=0\]
then 
\[a^{-1} (av)=a^{-1}0=0,\]
and thus
\[v=0. \]

\end{proof}
\end{enumerate}


\end{document}









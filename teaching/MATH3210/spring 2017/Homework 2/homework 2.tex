\documentclass[12pt,letterpaper]{article}
\usepackage[margin=1in]{geometry}
\usepackage{amsfonts}
\usepackage{amssymb}
\usepackage{amsthm}
\usepackage{amsmath}
\usepackage{enumerate}

%Here are some user-defined notations
\newcommand{\RR}{\mathbf R}  %bold R
\newcommand{\CC}{\mathbf C}  %bold C
\newcommand{\ZZ}{\mathbf Z}   %bold Z
\newcommand{\QQ}{\mathbf Q}   %bold Q
\newcommand{\rr}{\mathbb R}     %blackboard bold R
\newcommand{\cc}{\mathbb C}    %blackboard bold R
\newcommand{\zz}{\mathbb Z}    %blackboard bold R
\newcommand{\qq}{\mathbb Q}   %blackboard bold Q
\newcommand{\calM}{\mathcal M}  %calligraphic M
\newcommand{\sm}{\setminus} 
\newcommand{\bfa}{\mathbf a}
\newcommand{\bfb}{\mathbf b}
\newcommand{\bfc}{\mathbf c}




%Here are some user-defined operators
\newcommand{\re}{\operatorname {Re}}
\newcommand{\im}{\operatorname {Im}}


%These commands deal with theorem-like environments (i.e., italic)
\theoremstyle{plain}
\newtheorem{theorem}{Theorem}[section]
\newtheorem{corollary}[theorem]{Corollary}
\newtheorem{lemma}[theorem]{Lemma}
\newtheorem{conjecture}[theorem]{Conjecture}

%These deal with definition-like environments (i.e., non-italic)
\theoremstyle{definition}
\newtheorem{definition}[theorem]{Definition}
\newtheorem{example}[theorem]{Example}
\newtheorem{remark}[theorem]{Remark}

%your name and date in the header.
\usepackage[us]{datetime} 
\usepackage{fancyhdr}
\pagestyle{fancy}
\lhead{}
\chead{MATH 3210\\ Homework 2}
\rhead{ Your name \\ \today}
\lfoot{}
\cfoot{}
\rfoot{\thepage}
\renewcommand{\headrulewidth}{0 pt}
\renewcommand{\footrulewidth}{0 pt}
\begin{document}
\hrule
\vspace{.1in}
 {\bf Note:} Let $\Gamma$ be an arbitrary indexing set (possibly infinite and possibly uncountable).\\ \indent A collection of subspaces indexed by $\Gamma$ is $\{U_\gamma \mid \gamma\in \Gamma, U_\gamma\text{ is a subspace of } V\}$.\\
\hrule
\ \\
\begin{enumerate}[1.]
\item (\S 1.C \#11) Prove that the intersection of every collection of subspaces of $V$ is a subspace of $V$. \\
\ \\
\hrule 
{\bf Definition:}\vspace{.1in}\\
We say that a vector space $V$ is the direct sum of subspaces $U_1, \ldots ,U_n$ if the following hold true:
\begin{enumerate}[(a)]
\item $U_i\neq \{0\}$ for each $i=1, \ldots n$. 
\item $U_1\cap (U_1+\ldots U_{i-1}+U_{i+1}+\ldots U_n)=\{0\}$ for $i=1,\ldots n$. 
\item $V=U_1+\ldots +U_n$. 
\end{enumerate}
Denote this by $V=U_1\oplus\ldots \oplus U_n$. \\

\hrule
\item Prove the following theorem. \\
\begin{theorem} If $U_1, \ldots U_n$ are subspaces of $V$, then 
\[V=U_1\oplus \ldots \oplus U_n\] if and only if every $v\in V$ has a unique representation of the form 
\[v=u_1+\ldots +u_n\] 
where $u_i\in U_i$ for each $i=1,\ldots, n$. 
\end{theorem}
\ \\
\item (\S 2.A \# 14) Prove that $V$ is infinite dimensional if and only if there is a sequence $v_1, v_2, \ldots $ of vectors in $V$ such that $v_1, \ldots ,v_m$ is linearly independent for every positive integer $m$. \\
\ \\
\item (\S 2.A \# 16) Prove that the real vector space of all continuous real-valued functions on $[0,1]$ is infinite dimensional. \\
\ \\
\item (\S 2.B \# 8) Suppose that $U$ and $W$ are subspaces of $V$ such that $V=U\oplus W$. Suppose also that $u_1, \ldots ,u_m$ is a basis of $U$ and $w_1, \ldots , w_n$ is a basis of $W$. Prove that 
\[u_1, \ldots u_m, w_1, \ldots, w_n\]
is a basis of $V$. 
\end{enumerate}


\end{document}









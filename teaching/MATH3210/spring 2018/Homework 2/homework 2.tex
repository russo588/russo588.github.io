\documentclass[12pt,letterpaper]{article}
\usepackage[margin=1in]{geometry}
\usepackage{amsfonts}
\usepackage{amssymb}
\usepackage{amsthm}
\usepackage{amsmath}
\usepackage{enumerate}

%Here are some user-defined notations
\newcommand{\RR}{\mathbf R}  %bold R
\newcommand{\CC}{\mathbf C}  %bold C
\newcommand{\ZZ}{\mathbf Z}   %bold Z
\newcommand{\QQ}{\mathbf Q}   %bold Q
\newcommand{\rr}{\mathbb R}     %blackboard bold R
\newcommand{\cc}{\mathbb C}    %blackboard bold R
\newcommand{\zz}{\mathbb Z}    %blackboard bold R
\newcommand{\qq}{\mathbb Q}   %blackboard bold Q
\newcommand{\calM}{\mathcal M}  %calligraphic M
\newcommand{\sm}{\setminus} 
\newcommand{\bfa}{\mathbf a}
\newcommand{\bfb}{\mathbf b}
\newcommand{\bfc}{\mathbf c}





%Here are some user-defined operators
\newcommand{\re}{\operatorname {Re}}
\newcommand{\im}{\operatorname {Im}}


%These commands deal with theorem-like environments (i.e., italic)
\theoremstyle{plain}
\newtheorem{theorem}{Theorem}[section]
\newtheorem{corollary}[theorem]{Corollary}
\newtheorem{lemma}[theorem]{Lemma}
\newtheorem{conjecture}[theorem]{Conjecture}

%These deal with definition-like environments (i.e., non-italic)
\theoremstyle{definition}
\newtheorem{definition}[theorem]{Definition}
\newtheorem{example}[theorem]{Example}
\newtheorem{remark}[theorem]{Remark}

%your name and date in the header.
\usepackage[us]{datetime} 
\usepackage{fancyhdr}
\pagestyle{fancy}
\lhead{}
\chead{MATH 3210\\ Homework 2}
\rhead{ Your name \\ \today}
\lfoot{}
\cfoot{}
\rfoot{\thepage}
\renewcommand{\headrulewidth}{0 pt}
\renewcommand{\footrulewidth}{0 pt}
\begin{document}
\begin{enumerate}[1.]
\item Prove that the intersection of every collection of subspaces of $V$ is a subspace of $V$. The following definition maybe helpful.
\begin{definition}Let $\Gamma$ be an arbitrary indexing set (possibly infinite and possibly uncountable). A collection of subspaces indexed by $\Gamma$ is $\{U_\gamma \mid \gamma\in \Gamma, U_\gamma\text{ is a subspace of } V\}$.\\
\end{definition}
\item Prove that the real vector space of all continuous real-valued functions on $[0,1]$ is infinite dimensional. \\
\item This exercise will walk you through a basic scheme for polynomial interpolation.\\
\ \\
{\bf Polynomial Interpolation:}\\
\ \\
\noindent Given data
\begin{center}
\def\arraystretch{1.2}
\begin{tabular}{|c|c|c|c|}
\hline
$x_1$& $x_2$& $\cdots$ & $x_n$\\ \hline
$a_1$& $a_2$& $\cdots$ & $a_n$\\
\hline
\end{tabular}
\end{center}
We want to compute a \emph{interpolating polynomial} $p$, i.e. a polynomial of degree at most $n-1$ such that 
\[p(x_i)=f_i\]

Suppose you have a basis for the space of polynomials of $\text{deg}(p)\leq n-1$, $P_{n-1}(x)$, say $\{p_1, p_2, \ldots, p_n\}$. If our interpolating polynomial $p$ exists then 
\[p(x)=c_1p_1(x)+c_2p_2(x)+\ldots+c_np_n(x)\]
If $p$ interpolates the data, then 
\[p(x_1)=c_1p(x_1)+c_2p(x_1)+\ldots+c_np(x_1)= a_1\]
\[p(x_2)=c_1p(x_2)+c_2p(x_2)+\ldots+c_np(x_2)= a_2\]
\[\vdots\]
\[p(x_n)=c_1p(x_n)+c_2p(x_n)+\ldots+c_np(x_n)= a_n\]
Thus we have to solve the linear system:
\[\begin{pmatrix}
p_1(x_1) & p_2(x_1) & \cdots & p_n(x_1)\\
p_1(x_2) & p_2(x_2) & \cdots & p_n(x_2)\\
\vdots & \vdots & \cdots & \vdots\\
p_1(x_n) & p_2(x_n) & \cdots & p_n(x_n)\\
\end{pmatrix}
\begin{pmatrix}
c_1\\
c_2\\
\vdots \\
c_n
\end{pmatrix}=
\begin{pmatrix}
a_1\\
a_2\\
\vdots \\
a_n
\end{pmatrix}
\]
\newpage
{\noindent \bf Questions:}
\begin{enumerate}[(a)]
\item Find the matrix corresponding to the data points 
\begin{center}
\begin{tabular}{|c|c|c|c|}
\hline
$x_1=0$& $x_2=-1$& $x_3=1$\\ \hline
$2$& $3$& $3$\\
\hline
\end{tabular}
\end{center}
and using the basis $\{p_1(x)=1,\, p_2(x)=x,\, p_3(x)=x^2\}$

\item A more convenient basis for this problem is the Lagrange basis $\{L_1(x), \ldots, L_n(x)\}$ where the $i$-th Lagrange polynomial is given by 
\[L_i(x)=\prod_{j=1, j\neq i}^n \frac{x-x_j}{x_i-x_j}\]

\begin{enumerate}[b.1)]
\item Find the Lagrange polynomials for the above data. Show that 
\[L_i(x_k)=\left\{\begin{array}{cl}1 & \text{ if }k=i\\ 0 & \text{ if }k\neq i\end{array}\right..\]
\item Use the above fact to show that the Lagrange polynomials are indeed a basis for $P_2(x)$. 
\item Compute the corresponding matrix to the above data and using the Lagrange polynomials as a basis. 
\end{enumerate}

\end{enumerate}

\end{enumerate}
\end{document}









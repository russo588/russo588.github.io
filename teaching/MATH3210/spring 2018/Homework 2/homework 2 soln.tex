\documentclass[12pt,letterpaper]{article}
\usepackage[margin=1in]{geometry}
\usepackage{amsfonts}
\usepackage{amssymb}
\usepackage{amsthm}
\usepackage{amsmath}
\usepackage{enumerate}

%Here are some user-defined notations
\newcommand{\RR}{\mathbf R}  %bold R
\newcommand{\CC}{\mathbf C}  %bold C
\newcommand{\ZZ}{\mathbf Z}   %bold Z
\newcommand{\QQ}{\mathbf Q}   %bold Q
\newcommand{\rr}{\mathbb R}     %blackboard bold R
\newcommand{\cc}{\mathbb C}    %blackboard bold R
\newcommand{\zz}{\mathbb Z}    %blackboard bold R
\newcommand{\qq}{\mathbb Q}   %blackboard bold Q
\newcommand{\calM}{\mathcal M}  %calligraphic M
\newcommand{\sm}{\setminus} 
\newcommand{\bfa}{\mathbf a}
\newcommand{\bfb}{\mathbf b}
\newcommand{\bfc}{\mathbf c}

\newcommand{\ds}{\displaystyle}


%Here are some user-defined operators
\newcommand{\re}{\operatorname {Re}}
\newcommand{\im}{\operatorname {Im}}


%These commands deal with theorem-like environments (i.e., italic)
\theoremstyle{plain}
\newtheorem{theorem}{Theorem}[section]
\newtheorem{corollary}[theorem]{Corollary}
\newtheorem{lemma}[theorem]{Lemma}
\newtheorem{conjecture}[theorem]{Conjecture}

%These deal with definition-like environments (i.e., non-italic)
\theoremstyle{definition}
\newtheorem{definition}[theorem]{Definition}
\newtheorem{example}[theorem]{Example}
\newtheorem{remark}[theorem]{Remark}

%your name and date in the header.
\usepackage[us]{datetime} 
\usepackage{fancyhdr}
\pagestyle{fancy}
\lhead{}
\chead{MATH 3210\\ Homework 2}
\rhead{ Your name \\ \today}
\lfoot{}
\cfoot{}
\rfoot{\thepage}
\renewcommand{\headrulewidth}{0 pt}
\renewcommand{\footrulewidth}{0 pt}
\begin{document}
\hrule
\vspace{.1in}
 {\bf Note:} Let $\Gamma$ be an arbitrary indexing set (possibly infinite and possibly uncountable).\\ \indent A collection of subspaces indexed by $\Gamma$ is $\{U_\gamma \mid \gamma\in \Gamma, U_\gamma\text{ is a subspace of } V\}$.\\
\hrule
\ \\
\begin{enumerate}[1.]
\item (\S 1.C \#11) Prove that the intersection of every collection of subspaces of $V$ is a subspace of $V$. 
\begin{proof} A vector $u\in V$ is in $\bigcap_{\gamma \in \Gamma}U_\gamma$ if and only if $u\in U_\gamma$ for every $\gamma\in \Gamma$. To prove that $\bigcap_{\gamma \in \Gamma}U_\gamma$ is a subspace we will show that $0\in \bigcap_{\gamma \in \Gamma}U_\gamma$ and that $\bigcap_{\gamma \in \Gamma}U_\gamma$ is closed under addition and scalar multiplication. Since each $U_\gamma$ is a subspace then $0\in U_\gamma$ for all $\gamma \in \Gamma$. Hence $0\in  \bigcap_{\gamma\in \Gamma} U_\gamma$. Likewise, let $x$ and $y$ be arbitrary vectors in $ \bigcap_{\gamma\in \Gamma}U_\gamma$. Then $x\in U_\gamma$ and $y\in U_\gamma$ for all $\gamma \in \Gamma$. Since each $U_\gamma$ is a subspace we have $x+y\in U_\gamma $ for all $\gamma \in \Gamma$. Hence $x+y\in \bigcap_{\gamma\in \Gamma}U_\gamma$. Similarly, since each $U_\gamma$ is a subspace we have that $\lambda x\in U_\gamma$ for each $\lambda \in \mathbb{F}$, $x\in U_\gamma$ and each $\gamma \in \Gamma$. Thus $\lambda x\in \bigcap_{\gamma\in \Gamma} U_\gamma$. \\
\end{proof}
\ \\
\hrule 
{\bf Definition:}\vspace{.1in}\\
We say that a vector space $V$ is the direct sum of subspaces $U_1, \ldots ,U_n$ if the following hold true:
\begin{enumerate}[(a)]
\item $U_i\neq \{0\}$ for each $i=1, \ldots n$. 
\item $U_i\cap (U_1+\ldots U_{i-1}+U_{i+1}+\ldots U_n)=\{0\}$ for $i=1,\ldots n$. 
\item $V=U_1+\ldots +U_n$. 
\end{enumerate}
Denote this by $V=U_1\oplus\ldots \oplus U_n$. \\

\hrule
\item Prove the following theorem. \\
\begin{theorem} If $U_1, \ldots U_n$ are non-trivial subspaces of $V$, then 
\[V=U_1\oplus \ldots \oplus U_n\] if and only if every $v\in V$ has a unique representation of the form 
\[v=u_1+\ldots +u_n\] 
where $u_i\in U_i$ for each $i=1,\ldots, n$. 
\end{theorem}
\begin{proof} First assume that $V=U_1\oplus \ldots \oplus U_n$ for some non-trivial subspaces $U_1, \ldots U_n$ as defined above. Let $v\in V$ and suppose that 
\[v=v_1+\ldots +v_n\]
and 
\[v=u_1+\ldots +u_n\]
where $v_i, u_i\in U_i$ for $1\leq i\leq n$. Let $j\in \{1, \ldots ,n\}$ and note that 
\[-(v_j-u_j)=(v_1-u_1)+\ldots (v_{j-1}-u_{j-1})+(v_{j+1}-u_{j+1})+\ldots +(v_n-u_n).\]
Hence, 
\[v_j-u_j\in U_j\]
and 
\[v_j-u_j\in (U_1+\ldots +U_{j-1}+U_{j+1}+\ldots U_n)\]
for each $j\in \{1,\ldots ,n\}$. 
Hence $u_j=v_j$ for each $j\in \{1,\ldots, n\}$. 

Conversely, suppose each $v\in V$ has a unique representation in the form 
\[v=u_1+\ldots +u_n\quad \text{where each}\quad u_i\in U_i\neq \{0\}.\]
Parts (a) and (c) are automatically satisfied. We need to show part (b) of the the definition holds. If
\[w\in U_i\]
and 
\[w\in U_1+\ldots +U_{i-1}+U_{i+1}+\ldots U_n\]
for some $i\in \{1,\ldots, n\}$ then, 
\[0=w_1+\ldots w_{i-1}+w+w_{i+1}+\ldots +w_n\]
where each $w_j\in U_j$ for $j\in \{1,\ldots i-1\}\cup \{i+1, \ldots n\}$. By the unique representation of $0$ we have that 
\[w_1=\ldots w_{i-1}=w=w_{i+1}=\ldots =w_n=0.\]
Hence (b) is satisfied.
\end{proof}
\ \\
\item (\S 2.A \# 14) Prove that $V$ is infinite dimensional if and only if there is a sequence $v_1, v_2, \ldots $ of vectors in $V$ such that $v_1, \ldots ,v_m$ is linearly independent for every positive integer $m$. 
\begin{proof} Suppose that $V$ is infinite dimensional. Let $v_1\neq 0$ and choose $v_2, v_3, \ldots $ by the following procedure:
Suppose $v_1, \ldots v_{m-1}$ is chosen, and choose $v_m\in V$ such that $v_m\not\in \text{span}\{v_1, \ldots v_{m-1}\}$. Since $V$ is infinite dimensional this is always possible and $\{v_1, \ldots ,v_m\}$ is linearly independent for each $m\in \mathbb{N}$. Conversely, suppose $V$ is finite dimensional. Thus $V$ has a finite spanning list. Since the length of every linearly independent list must be less than or equal to the length of any spanning list there does not exist a sequence of vectors such that $v_1, \ldots v_m$ is linearly independent for all $m\in \mathbb{N}$.
\end{proof}
\ \\
\item (\S 2.A \# 16) Prove that the real vector space of all continuous real-valued functions on $[0,1]$ is infinite dimensional. 
\begin{proof} For each $m\in \mathbb{N}$ we have that $\{1, x, \ldots , x^m\}$ is a linearly independent list of vectors in $C[0,1]$ (continuous functions over [0,1]) since if 
\[a_0\cdot 1+\ldots+a_m\cdot x^m=0 \quad\text{ for all }\quad x\in [0,1]\]
then $a_0=\ldots =a_m=0$ (the only polynomial with an infinite number of zeros in $[0,1]$ is the zero polynomial). Hence, by the above problem, polynomials over $[0,1]$ form a infinite dimensional subspace and hence $C[0,1]$ is infinite dimensional.
\end{proof}
\ \\
\item (\S 2.B \# 8) Suppose that $U$ and $W$ are subspaces of $V$ such that $V=U\oplus W$. Suppose also that $u_1, \ldots ,u_m$ is a basis of $U$ and $w_1, \ldots , w_n$ is a basis of $W$. Prove that 
\[u_1, \ldots u_m, w_1, \ldots, w_n\]
is a basis of $V$.
\begin{proof} Clearly, $V\subseteq \text{span}\{u_1,\ldots u_m, w_1,\ldots w_n\}$ since $V=U\oplus W$. We need to show that $\{u_1, \ldots u_m, w_1, \ldots w_n\}$ is linearly independent. Suppose that 
\[a_1u_1+\ldots a_mu_m+b_1w_1+\ldots b_nw_n=0.\] Note that 
\[a_1u_1+\ldots a_mu_m=-(b_1w_1+\ldots +b_nw_n).\] 
Hence,
\[a_1u_1+\ldots a_mu_m\in U\cap W\]
and 
\[b_1w_1+\ldots b_nw_n\in U\cap W.\] 
Thus, 
\[a_1u_1+\ldots a_mu_m=b_1w_1+\ldots b_nw_n=0\] and by linear independence of $\{u_1, \ldots ,u_m\}$  and $\{w_1, \ldots , w_n\}$  we have 
\[a_1=\ldots a_m=b_1=\ldots =b_n=0.\]
\end{proof} 
\end{enumerate}


\end{document}









\documentclass[12pt,letterpaper]{article}
\usepackage[margin=1in]{geometry}
\usepackage{amsfonts}
\usepackage{amssymb}
\usepackage{amsthm}
\usepackage{amsmath}
\usepackage{enumerate}

%Here are some user-defined notations
\newcommand{\RR}{\mathbf R}  %bold R
\newcommand{\CC}{\mathbf C}  %bold C
\newcommand{\ZZ}{\mathbf Z}   %bold Z
\newcommand{\QQ}{\mathbf Q}   %bold Q
\newcommand{\rr}{\mathbb R}     %blackboard bold R
\newcommand{\cc}{\mathbb C}    %blackboard bold R
\newcommand{\zz}{\mathbb Z}    %blackboard bold R
\newcommand{\qq}{\mathbb Q}   %blackboard bold Q
\newcommand{\calM}{\mathcal M}  %calligraphic M
\newcommand{\sm}{\setminus} 
\newcommand{\bfa}{\mathbf a}
\newcommand{\bfb}{\mathbf b}
\newcommand{\bfc}{\mathbf c}




%Here are some user-defined operators
\newcommand{\re}{\operatorname {Re}}
\newcommand{\im}{\operatorname {Im}}


%These commands deal with theorem-like environments (i.e., italic)
\theoremstyle{plain}
\newtheorem{theorem}{Theorem}[section]
\newtheorem{corollary}[theorem]{Corollary}
\newtheorem{lemma}[theorem]{Lemma}
\newtheorem{conjecture}[theorem]{Conjecture}

%These deal with definition-like environments (i.e., non-italic)
\theoremstyle{definition}
\newtheorem{definition}[theorem]{Definition}
\newtheorem{example}[theorem]{Example}
\newtheorem{remark}[theorem]{Remark}

%your name and date in the header.
\usepackage[us]{datetime} 
\usepackage{fancyhdr}
\pagestyle{fancy}
\lhead{}
\chead{MATH 3210\\ Homework 4}
\rhead{ Your name \\ date}
\lfoot{}
\cfoot{}
\rfoot{\thepage}
\renewcommand{\headrulewidth}{0 pt}
\renewcommand{\footrulewidth}{0 pt}
\begin{document}
\begin{enumerate}[1.]
\item Suppose $S,T\in \mathcal{L}(V)$ are such that $ST=TS$. Prove that $\text{ran}(S)$ is invariant under $T$
\item Let $V=\left(\mathbb{Z}/5\mathbb{Z}\right)^3$. Define, $\langle \cdot, \cdot \rangle: V\times V\rightarrow \mathbb{Z}/5\mathbb{Z}$ by 
\[\langle\vec{x},\vec{y}\rangle =x_1y_1+x_2y_2+x_3y_3 \quad \quad \text{ for all }\vec{x},\vec{y}\in V.\] 
Is $\langle \cdot, \cdot \rangle$ an inner product?
\item Let $V=\left(\mathbb{Z}/5\mathbb{Z}\right)^2$ and $T:V\rightarrow V$ be the transformation $T(\vec{x})=A\cdot \vec{x}$ where $A$ is given by 
\[A=\begin{bmatrix}2 & 3 \\ 0 & 4\end{bmatrix}\in M_{2\times 2}(\mathbb{Z}/5\mathbb{Z}).\]
Does $T$ have eigenvalues and eigenvectors? If so, find them and determine if $T$ has a diagonal matrix with respect to a basis of eigen-vectors. 
\item In class we defined for a polynomial $p(x)=a_nx^n+\ldots+a_1x +a_0$ and an operator $T\in \mathcal{L}(V)$ the operator $p(T)$ as 
\[p(T)=a_nT^n+\ldots +a_1T+a_0I\in \mathcal{L}(V).\]
Let $e^x=\sum_{n=0}^\infty \frac{x^n}{n!}$ and define for a operator $T\in \mathcal{L}(V)$ 
\[e^{T}=\sum_{n=0}^\infty \frac{T^n}{n!}.\]
[If you have taken analysis do not worry about convergence, the power series has an infinite radius of convergence.]

Let $A=\begin{bmatrix}1 & 1 \\ 0 & 1\end{bmatrix}$. 

\begin{enumerate}[(a)]
\item Find a formula for $A^n$ and prove it by induction. 
\item Find $e^A$.
\end{enumerate}

\item The Fibonacci sequence $F_1, F_2, \ldots$ is defined by 
\[F_1=1, F_2=1, \quad \text{ and }\quad F_n=F_{n-2}+F_{n-1} \text{ for }n\geq 3\]
Define $T\in \mathcal{L}(\mathbb{R}^2)$ by 
\[T \left( \left[\begin{array}{c}x\\y\end{array}\right]\right)=\left[\begin{array}{cc}y\\ x+y\end{array}\right].\]
\begin{enumerate}[(a)]
\item Show that $T^n\left(\left[\begin{array}{c}0\\1\end{array}\right]\right)=\left[\begin{array}{c}F_n\\F_{n+1}\end{array}\right]$
\item Find the eigenvalues of $T$.
\item Find a basis of $\mathbb{R}^2$ consisting of eigenvectors of $T$. 
\item Use the solution to part (c) to compute $T^n\left(\left[\begin{array}{c}0\\1\end{array}\right]\right)$. Conclude that 
\[F_n=\frac{1}{\sqrt{5}}\left[\left(\frac{1+\sqrt{5}}{2}\right)^n-\left(\frac{1-\sqrt{5}}{2}\right)^n\right]\]
for each positive integer $n$. 
\end{enumerate}
\end{enumerate}

\end{document}









\documentclass[12pt,letterpaper]{article}
\usepackage[margin=1in]{geometry}
\usepackage{amsfonts}
\usepackage{amssymb}
\usepackage{amsthm}
\usepackage{amsmath}
\usepackage{enumerate}

%Here are some user-defined notations
\newcommand{\RR}{\mathbf R}  %bold R
\newcommand{\CC}{\mathbf C}  %bold C
\newcommand{\ZZ}{\mathbf Z}   %bold Z
\newcommand{\QQ}{\mathbf Q}   %bold Q
\newcommand{\rr}{\mathbb R}     %blackboard bold R
\newcommand{\cc}{\mathbb C}    %blackboard bold R
\newcommand{\zz}{\mathbb Z}    %blackboard bold R
\newcommand{\qq}{\mathbb Q}   %blackboard bold Q
\newcommand{\calM}{\mathcal M}  %calligraphic M
\newcommand{\sm}{\setminus} 
\newcommand{\bfa}{\mathbf a}
\newcommand{\bfb}{\mathbf b}
\newcommand{\bfc}{\mathbf c}




%Here are some user-defined operators
\newcommand{\re}{\operatorname {Re}}
\newcommand{\im}{\operatorname {Im}}


%These commands deal with theorem-like environments (i.e., italic)
\theoremstyle{plain}
\newtheorem{theorem}{Theorem}[section]
\newtheorem{corollary}[theorem]{Corollary}
\newtheorem{lemma}[theorem]{Lemma}
\newtheorem{conjecture}[theorem]{Conjecture}

%These deal with definition-like environments (i.e., non-italic)
\theoremstyle{definition}
\newtheorem{definition}[theorem]{Definition}
\newtheorem{example}[theorem]{Example}
\newtheorem{remark}[theorem]{Remark}

%your name and date in the header.
\usepackage[us]{datetime} 
\usepackage{fancyhdr}
\pagestyle{fancy}
\lhead{}
\chead{MATH 3210\\ Homework 4}
\rhead{ Your name \\ date}
\lfoot{}
\cfoot{}
\rfoot{\thepage}
\renewcommand{\headrulewidth}{0 pt}
\renewcommand{\footrulewidth}{0 pt}
\begin{document}
\begin{enumerate}[1.]
\item Suppose that $A$ and $B$ are monoids and that $\phi:A\rightarrow B$ is a monoid homomorphism. Show that $\phi$ sends the invertible elements of $A$ to the invertible elements of $B$. Use this to show that the determinant of an invertible matrix is non-zero. \\
\ \\
\item Show by induction that the determinant of an upper triangular matrix is the product of the diagonal entries. \\
\ \\
\item Call a matrix $A$ nilpotent if $A^k=0$ for some positive integer $k$. Show that every square nilpotent matrix has determinant zero. \\
\ \\
\item Suppose $A$ is square non-invertible. Note that there exists a sequence of elementry row operations $e_1,\ldots ,e_n$ such that 
$B$ the matrix resulting from applying $e_1,\ldots, e_n$ to $A$ is upper triangular and contains a $0$ along the diagonal. Use this to prove that the determinant of a square non-invertible matrix is zero. 
\ \\
\item Use concepts in Example 3.104 on page 105 of your text to prove Theorem $3.106$.\\
\ \\
\item (\S 3.F \# 12) Show that the dual map of the identity map on $V$ is the identity map on $V^\prime$. \\
\ \\
\item (\S 3.F \# 34) The \emph{double dual} of $V$ denoted $V^{\prime\prime}$, is defined to be the dual space of $V^\prime$. In other words $V^{\prime}=(V^\prime)^\prime$. Define $\Lambda:V\rightarrow V^{\prime\prime}$ by 
\[(\Lambda v) (\varphi)=\varphi(v)\]
for $v\in V$ and $\varphi \in V^\prime$. 
\begin{enumerate}[(a)]
\item Show that $\Lambda$ is a linear map from $V$ to $V^{\prime\prime}$. 
\item Show that if $T\in \mathcal{L}(V)$ then $T^{\prime\prime}\circ \Lambda=\Lambda\circ T$ where $T^{\prime\prime}=(T^\prime)^\prime$. 
\item Show that if $V$ is finite dimensional, then $\Lambda$ is an isomorphism from $V$ onto $V^{\prime\prime}$.
\end{enumerate}
\end{enumerate}


\end{document}









\documentclass[12pt,letterpaper]{article}
\usepackage[margin=1in]{geometry}
\usepackage{amsfonts}
\usepackage{amssymb}
\usepackage{amsthm}
\usepackage{amsmath}
\usepackage{enumerate}

%Here are some user-defined notations
\newcommand{\RR}{\mathbf R}  %bold R
\newcommand{\CC}{\mathbf C}  %bold C
\newcommand{\ZZ}{\mathbf Z}   %bold Z
\newcommand{\QQ}{\mathbf Q}   %bold Q
\newcommand{\rr}{\mathbb R}     %blackboard bold R
\newcommand{\cc}{\mathbb C}    %blackboard bold R
\newcommand{\zz}{\mathbb Z}    %blackboard bold R
\newcommand{\qq}{\mathbb Q}   %blackboard bold Q
\newcommand{\calM}{\mathcal M}  %calligraphic M
\newcommand{\sm}{\setminus} 
\newcommand{\bfa}{\mathbf a}
\newcommand{\bfb}{\mathbf b}
\newcommand{\bfc}{\mathbf c}




%Here are some user-defined operators
\newcommand{\re}{\operatorname {Re}}
\newcommand{\im}{\operatorname {Im}}


%These commands deal with theorem-like environments (i.e., italic)
\theoremstyle{plain}
\newtheorem{theorem}{Theorem}[section]
\newtheorem{corollary}[theorem]{Corollary}
\newtheorem{lemma}[theorem]{Lemma}
\newtheorem{conjecture}[theorem]{Conjecture}

%These deal with definition-like environments (i.e., non-italic)
\theoremstyle{definition}
\newtheorem{definition}[theorem]{Definition}
\newtheorem{example}[theorem]{Example}
\newtheorem{remark}[theorem]{Remark}

%your name and date in the header.
\usepackage[us]{datetime} 
\usepackage{fancyhdr}
\pagestyle{fancy}
\lhead{}
\chead{MATH 3210\\ Homework 5}
\rhead{ Your name \\ date}
\lfoot{}
\cfoot{}
\rfoot{\thepage}
\renewcommand{\headrulewidth}{0 pt}
\renewcommand{\footrulewidth}{0 pt}
\begin{document}
\begin{enumerate}[1.]
\item Let $C_0^1[0,1]=\{f:[0,1]\rightarrow \mathbb{C} \mid f\text{ is continuously differentiable and } f(0)=0=f(1)\}$, i.e. it is the vector-space of functions with a continuous derivative which are zero at the end points. Let 
\[\langle f, g\rangle =\int_{0}^1f(x)\bar{g}(x)\, dx\]
be an inner-product on this space. Define a map $T: C_0^1[0,1]\rightarrow C[0,1]$ by $T(f)=-i\frac{df}{dx}$. 
Show that, 
\[\langle Tf, g\rangle =\langle f, Tg\rangle.\]
{\bf Hint: }Use integration by parts.

\item Show that a normal operator is self adjoint if and only if its eigenvalues are real. 

\item Let $U\in \mathcal{L}(V)$ be called a unitary operator if $U^*U=UU^*=I$. 
\begin{enumerate}
\item Show that for all $v\in V$ that $\|v\|=\|Uv\|$. 
\item Show that if $\lambda$ is an eigenvalue of $U$ then $|\lambda|=1$. 
\item Show that if $\{e_1, e_2, \ldots, e_n\}$ is an orthonormal basis then $\{Ue_1, Ue_2, \ldots, Ue_n\}$ is an orthonormal basis.
\item Show that if  $S$ is an operator such that if $\{e_1,\ldots, e_n\}$ is an orthonormal basis then $\{Se_1, \ldots, Se_n\}$ is an orthonormal basis then $S$ is unitary.
\end{enumerate}

\item Call a matrix $U$ unitary if the operator $S(x)=Ux$ is a unitary operator. Let $T:\mathbb{C}^n\rightarrow \mathbb{C}^n$ be a normal operator given by $T(x)=Ax$ where $A$ is an $n\times n$ matrix ($A$ is the matrix for $T$ with respect to the standard basis.) Show that there exists a unitary matrix $U$ such that $U^{-1}AU=D$ where $D$ is a diagonal matrix. 

\end{enumerate}


\end{document}









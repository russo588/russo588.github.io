\documentclass[12pt,letterpaper]{article}
\usepackage[margin=1in]{geometry}
\usepackage{amsfonts}
\usepackage{amssymb}
\usepackage{amsthm}
\usepackage{amsmath}
\usepackage{enumerate}

%Here are some user-defined notations
\newcommand{\RR}{\mathbf R}  %bold R
\newcommand{\CC}{\mathbf C}  %bold C
\newcommand{\ZZ}{\mathbf Z}   %bold Z
\newcommand{\QQ}{\mathbf Q}   %bold Q
\newcommand{\rr}{\mathbb R}     %blackboard bold R
\newcommand{\cc}{\mathbb C}    %blackboard bold R
\newcommand{\zz}{\mathbb Z}    %blackboard bold R
\newcommand{\qq}{\mathbb Q}   %blackboard bold Q
\newcommand{\calM}{\mathcal M}  %calligraphic M
\newcommand{\sm}{\setminus} 
\newcommand{\bfa}{\mathbf a}
\newcommand{\bfb}{\mathbf b}
\newcommand{\bfc}{\mathbf c}




%Here are some user-defined operators
\newcommand{\re}{\operatorname {Re}}
\newcommand{\im}{\operatorname {Im}}


%These commands deal with theorem-like environments (i.e., italic)
\theoremstyle{plain}
\newtheorem{theorem}{Theorem}[section]
\newtheorem{corollary}[theorem]{Corollary}
\newtheorem{lemma}[theorem]{Lemma}
\newtheorem{conjecture}[theorem]{Conjecture}

%These deal with definition-like environments (i.e., non-italic)
\theoremstyle{definition}
\newtheorem{definition}[theorem]{Definition}
\newtheorem{example}[theorem]{Example}
\newtheorem{remark}[theorem]{Remark}

%your name and date in the header.
\usepackage[us]{datetime} 
\usepackage{fancyhdr}
\pagestyle{fancy}
\lhead{}
\chead{MATH 3210\\ Homework 5}
\rhead{ Your name \\ date}
\lfoot{}
\cfoot{}
\rfoot{\thepage}
\renewcommand{\headrulewidth}{0 pt}
\renewcommand{\footrulewidth}{0 pt}
\begin{document}
\begin{enumerate}[1.]
\item (\S 5.A \#3) Suppose $S,T\in \mathcal{L}(V)$ are such that $ST=TS$. Prove that $\text{ran}(S)$ is invariant under $T$. 
\item (\S5.B \#1) Suppose that $T\in \mathcal{L}(V)$ and there exists a positive integer $n$ such that $T^n=0$. Prove that $(I-T)$ is invertible and that 
\[(I-T)^{-1}=I+T+\cdots +T^{n-1}\]
\item Suppose that $S,T\in \mathcal{L}(V)$ and $S$ is invertible. Suppose that $p\in \mathcal{P}(\mathbb{F})$ is a polynomial. Prove that 
\[p(STS^{-1})=Sp(T)S^{-1}.\]
\item (\S 5.C \# 16) The Fibonacci sequence $F_1, F_2, \ldots$ is defined by 
\[F_1=1, F_2=1, \quad \text{ and }\quad F_n=F_{n-2}+F_{n-1} \text{ for }n\geq 3\]
Define $T\in \mathcal{L}(\mathbb{R}^2)$ by 
\[T \left( \left[\begin{array}{c}x\\y\end{array}\right]\right)=\left[\begin{array}{cc}y\\ x+y\end{array}\right].\]
\begin{enumerate}[(a)]
\item Show that $T^n\left(\left[\begin{array}{c}0\\1\end{array}\right]\right)=\left[\begin{array}{c}F_n\\F_{n+1}\end{array}\right]$
\item Find the eigenvalues of $T$.
\item Find a basis of $\mathbb{R}^2$ consisting of eigenvectors of $T$. 
\item Use the solution to part (c) to compute $T^n\left(\left[\begin{array}{c}0\\1\end{array}\right]\right)$. Conclude that 
\[F_n=\frac{1}{\sqrt{5}}\left[\left(\frac{1+\sqrt{5}}{2}\right)^n-\left(\frac{1-\sqrt{5}}{2}\right)^n\right]\]
for each positive integer $n$. 
\end{enumerate}
\end{enumerate}


\end{document}









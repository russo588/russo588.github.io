\documentclass[12pt,letterpaper]{article}
\usepackage[margin=1in]{geometry}
\usepackage{amsfonts}
\usepackage{amssymb}
\usepackage{amsthm}
\usepackage{amsmath}
\usepackage{enumerate}

%Here are some user-defined notations
\newcommand{\RR}{\mathbf R}  %bold R
\newcommand{\CC}{\mathbf C}  %bold C
\newcommand{\ZZ}{\mathbf Z}   %bold Z
\newcommand{\QQ}{\mathbf Q}   %bold Q
\newcommand{\rr}{\mathbb R}     %blackboard bold R
\newcommand{\cc}{\mathbb C}    %blackboard bold R
\newcommand{\zz}{\mathbb Z}    %blackboard bold R
\newcommand{\qq}{\mathbb Q}   %blackboard bold Q
\newcommand{\calM}{\mathcal M}  %calligraphic M
\newcommand{\sm}{\setminus} 
\newcommand{\bfa}{\mathbf a}
\newcommand{\bfb}{\mathbf b}
\newcommand{\bfc}{\mathbf c}




%Here are some user-defined operators
\newcommand{\re}{\operatorname {Re}}
\newcommand{\im}{\operatorname {Im}}


%These commands deal with theorem-like environments (i.e., italic)
\theoremstyle{plain}
\newtheorem{theorem}{Theorem}[section]
\newtheorem{corollary}[theorem]{Corollary}
\newtheorem{lemma}[theorem]{Lemma}
\newtheorem{conjecture}[theorem]{Conjecture}

%These deal with definition-like environments (i.e., non-italic)
\theoremstyle{definition}
\newtheorem{definition}[theorem]{Definition}
\newtheorem{example}[theorem]{Example}
\newtheorem{remark}[theorem]{Remark}

%your name and date in the header.
\usepackage[us]{datetime} 
\usepackage{fancyhdr}
\pagestyle{fancy}
\lhead{}
\chead{MATH 3210\\ Homework 1}
\rhead{ Your name \\ \today}
\lfoot{}
\cfoot{}
\rfoot{\thepage}
\renewcommand{\headrulewidth}{0 pt}
\renewcommand{\footrulewidth}{0 pt}
\begin{document}
\begin{enumerate}[1.]
 \item Let $\qq(\sqrt{2})=\{a+b\sqrt{2} : a,b\in \mathbb{Q}\}$.  Note that $\qq(\sqrt{2})$ is field and more specifically it is known as an algebraic number field. The binary operations on $\qq(\sqrt{2})$ are the standard addition and multiplication of numbers. Verify for all $\alpha\neq 0$ in $\qq(\sqrt{2})$ that there exists a $\beta\in \qq(\sqrt{2})$ such that $\alpha \cdot \beta = 1$. \\
 \ \\
 \hrule
 For the next two problems let $\mathbb{F}$ be an arbitrary field. We define the following vector space over $\mathbb{F}$. Let
\[\mathbb{F}^n=\{(x_1,x_2,\ldots, x_n) : x_j\in \mathbb{F},\ j=1, \ldots n\}\] where scalar multiplication and vector addition is defined thusly, 
\[\lambda \cdot (x_1, \ldots , x_n)=(\lambda x_1, \ldots , \lambda x_n)\]
\[(x_1,\ldots ,x_n)+(y_1,\ldots ,y_n)=(x_1+y_1, \ldots x_n+y_n).\]
\hrule

\item (\#13 \S 1.A) Show that $(ab)x=a(bx)$ for all $x\in \mathbb{F}^n$ and all $a,b\in \mathbb{F}$. \\
\item (\# 15\S 1.A) Show that $\lambda \cdot (x+y)=\lambda x+\lambda y$ for all $\lambda \in \mathbb{F}$ and all $x,y\in \mathbb{F}^n$. \\
\hrule 
\ \\
For the next two problems let $\mathbb{F}$ be an arbitrary field and $V$ a vector space over $\mathbb{F}$.
\ \\
\hrule 
\item (\#1 \S 1.B) Prove that $-(-v)=v$ for every $v\in V$. 
\item (\#2 \S 1.B) Suppose $a\in \mathbb{F}$, $v\in V$, and $av=0$. Prove $a=0$ or $v=0$. 
\end{enumerate}


\end{document}









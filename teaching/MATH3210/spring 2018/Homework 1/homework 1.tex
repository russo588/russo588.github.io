\documentclass[12pt,letterpaper]{article}
\usepackage[margin=1in]{geometry}
\usepackage{amsfonts}
\usepackage{amssymb}
\usepackage{amsthm}
\usepackage{amsmath}
\usepackage{enumerate}

%Here are some user-defined notations
\newcommand{\RR}{\mathbf R}  %bold R
\newcommand{\CC}{\mathbf C}  %bold C
\newcommand{\ZZ}{\mathbf Z}   %bold Z
\newcommand{\QQ}{\mathbf Q}   %bold Q
\newcommand{\rr}{\mathbb R}     %blackboard bold R
\newcommand{\cc}{\mathbb C}    %blackboard bold R
\newcommand{\zz}{\mathbb Z}    %blackboard bold R
\newcommand{\qq}{\mathbb Q}   %blackboard bold Q
\newcommand{\calM}{\mathcal M}  %calligraphic M
\newcommand{\sm}{\setminus} 
\newcommand{\bfa}{\mathbf a}
\newcommand{\bfb}{\mathbf b}
\newcommand{\bfc}{\mathbf c}




%Here are some user-defined operators
\newcommand{\re}{\operatorname {Re}}
\newcommand{\im}{\operatorname {Im}}


%These commands deal with theorem-like environments (i.e., italic)
\theoremstyle{plain}
\newtheorem{theorem}{Theorem}[section]
\newtheorem{corollary}[theorem]{Corollary}
\newtheorem{lemma}[theorem]{Lemma}
\newtheorem{conjecture}[theorem]{Conjecture}

%These deal with definition-like environments (i.e., non-italic)
\theoremstyle{definition}
\newtheorem{definition}[theorem]{Definition}
\newtheorem{example}[theorem]{Example}
\newtheorem{remark}[theorem]{Remark}

%your name and date in the header.
\usepackage[us]{datetime} 
\usepackage{fancyhdr}
\pagestyle{fancy}
\lhead{}
\chead{MATH 3210\\ Homework 1}
\rhead{ Your name \\ \today}
\lfoot{}
\cfoot{}
\rfoot{\thepage}
\renewcommand{\headrulewidth}{0 pt}
\renewcommand{\footrulewidth}{0 pt}
\begin{document}
\begin{enumerate}[1.]
 \item Let $\qq(\sqrt{2})=\{a+b\sqrt{2} : a,b\in \mathbb{Q}\}$.  Note that $\qq(\sqrt{2})$ is field and more specifically it is known as an algebraic number field. The binary operations on $\qq(\sqrt{2})$ are the standard addition and multiplication of numbers. Verify for all $\alpha\neq 0$ in $\qq(\sqrt{2})$ that there exists a $\beta\in \qq(\sqrt{2})$ such that $\alpha \cdot \beta = 1$. \\

\item Is the space of non-negative functions on the interval $[0,1]$ a vector space over the real numbers $\mathbb{R}$? Justify your answer with a proof. 

\item Let $M_{2\times 2}$ be the set of $2\times 2$ matrices with real entries, i.e. 
\[M_{2\times 2}=\left\{ \begin{pmatrix}a & b \\ c & d\end{pmatrix} \middle| \, a,b, c,d\in \mathbb{R}\right\}.\]
$M_{2\times 2}$ is a vector space over the reals with the operations 
\[k \begin{pmatrix}a & b \\ c & d\end{pmatrix}= \begin{pmatrix}ka & kb \\ kc & kd\end{pmatrix} \text{ with }k\in \mathbb{R}\]
\[\begin{pmatrix}a & b \\ c & d\end{pmatrix}+\begin{pmatrix}a^\prime & b^\prime \\ c^\prime & d^\prime\end{pmatrix}= \begin{pmatrix}a+a^\prime & b+b^\prime \\ c+c^\prime & d+d^\prime\end{pmatrix}\]
Identify the additive identity in $M_{2\times 2}$ and justify your answer with a proof. 
\item Are the positive real numbers a field? Justify your answer. 
\item Suppose $a\in \mathbb{F}$, $v\in V$, and $av=0$. Prove $a=0$ or $v=0$. 
\end{enumerate}


\end{document}









\documentclass[12pt,letterpaper]{amsart}
\setlength{\oddsidemargin}{.0in}
\setlength{\evensidemargin}{.0in}
\setlength{\textwidth}{6.5in}
\setlength{\topmargin}{-.3in}
\setlength{\headsep}{.20in}
\setlength{\textheight}{9.in}
\usepackage[leqno]{amsmath}
\usepackage{amsfonts}
\usepackage{amssymb}
\usepackage{amsthm}
\usepackage{amssymb}
\usepackage[all]{xy}
\usepackage{graphicx}
\usepackage{enumerate}


%Here are some user-defined notations
\newcommand{\RR}{\mathbf R}  %bold R
\newcommand{\CC}{\mathbf C}  %bold C
\newcommand{\ZZ}{\mathbf Z}   %bold Z
\newcommand{\QQ}{\mathbf Q}   %bold Q
\newcommand{\rr}{\mathbb R}     %blackboard bold R
\newcommand{\cc}{\mathbb C}    %blackboard bold R
\newcommand{\zz}{\mathbb Z}    %blackboard bold R
\newcommand{\qq}{\mathbb Q}   %blackboard bold Q
\newcommand{\ZZn}[1]{\ZZ/{#1}\ZZ}
\newcommand{\zzn}[1]{\zz/{#1}\zz}
\newcommand{\calM}{\mathcal M}  %calligraphic M
\newcommand{\sm}{\setminus} 
\newcommand{\bfa}{\mathbf a}
\newcommand{\bfb}{\mathbf b}
\newcommand{\bfc}{\mathbf c}


%improving spacing in tables (space above and below characters in a row)
\newcommand{\tfix}{\rule{0pt}{2.6ex}}
\newcommand{\bfix}{\rule[-1.2ex]{0pt}{0pt}}


%Here are commands with variable inputs 
\newcommand{\intf}[1]{\int_a^b{#1}\,dx}
\newcommand{\intfb}[3]{\int_{#1}^{#2}{#3}\,dx}
\newcommand{\pln}[1]{$\sm${\tt #1}}
\newcommand{\bgn}[1]{$\tt {\sm}begin\{#1\}$}
\newcommand{\nd}[1]{$\tt {\sm}end\{#1\}$}
\newcommand{\marginalfootnote}[1]{%
        \footnote{#1}
        \marginpar[\hfill{\sf\thefootnote}]{{\sf\thefootnote}}}
\newcommand{\edit}[1]{\marginalfootnote{#1}}


%Here are some user-defined operators
\newcommand{\Tr}{\operatorname {Tr}}
\newcommand{\GL}{\operatorname {GL}}
\newcommand{\SL}{\operatorname {SL}}
\newcommand{\Prob}{\operatorname {Prob}}
\newcommand{\re}{\operatorname {Re}}   %new definition of "real part" operator
\newcommand{\im}{\operatorname {Im}}   %new definition of "imaginary part" operator


%These commands deal with theorem-like environments (i.e., italic)
\theoremstyle{plain}
\newtheorem{theorem}{Theorem}[section]
\newtheorem{proposition}{Proposition}[section]
\newtheorem{corollary}[theorem]{Corollary}
\newtheorem{lemma}[theorem]{Lemma}
\newtheorem{conjecture}[theorem]{Conjecture}

%These deal with definition-like environments (i.e., non-italic)
\theoremstyle{definition}
\newtheorem{definition}[theorem]{Definition}
\newtheorem{example}[theorem]{Example}
\newtheorem{remark}[theorem]{Remark}

%This numbers equations by section
\numberwithin{equation}{section}


%This is for hypertext references
\usepackage{color}
\usepackage{hyperref}


\begin{document}
\section{Basic Notions}
\begin{definition} A binary operation on a set $S$ is a function $f:S\times S\rightarrow S$. 
\end{definition}
\begin{definition} A \emph{field} is a set $\mathbb{F}$ together with two binary operations $+$, and $\cdot$ called addition and multiplication (respectively) such that 
\ \\
\begin{enumerate}[1.]
\setlength{\itemsep}{5pt}
\item For all $a,b,c\in \mathbb{F}$ we have 
\[a+(b+c)=(a+b)+c\]
and 
\[a\cdot(b\cdot c)=(a\cdot b)\cdot c.\]
\item For all $a,b\in \mathbb{F}$ we have 
\[a+b=b+a\]
and 
\[a\cdot b=b\cdot a.\]
\item There exists an element $0\in \mathbb{F}$, called an additive identity,  such that for all $a\in \mathbb{F}$ we have $a+0=a$. 
\item There exists an element $1\in \mathbb{F}$, called a multiplicative identity, such that for all $a\in \mathbb{F}$ we have $a\cdot 1=a$.
\item For all $a\in \mathbb{F}$ there exists an element $b\in \mathbb{F}$, called an additive inverse, such that $a+b=0.$ 
\item For all $a\in \mathbb{F}$ such that $a\neq 0$ there exists an element $c\in \mathbb{F}$, called a multiplicative inverse, such that $a\cdot c=1$.
\item  For all $a,b,c\in \mathbb{F}$ 
\[a\cdot (b+c)=(a\cdot b)+ (a\cdot c)\]

\end{enumerate}
\end{definition}
\ \\
{\bf \noindent Note:} Fields have a unique additive and multiplicative identity denoted 0 and 1 respectively. Moreover, when the additive and multiplicative inverses exist they are unique.\\
\ \\
{\bf \noindent Some examples:} All of the following examples are with their standard operations. 
\begin{enumerate}[1.]
\setlength{\itemsep}{5pt}
\item $\mathbb{Q}$ (rational numbers)
\item $\mathbb{R}$ (real numbers)
\item $\mathbb{C}$ (complex numbers)
\item $\mathbb{Z}/p\mathbb{Z}$ for $p$ prime (Integers modulo p)
\end{enumerate}
\ \\
{\bf Non example: }$\mathbb{Z}$ is not a field, it lacks multiplicative inverses. 
\ \\
\begin{definition} A \emph{vector space} $V$ over a field $\mathbb{F}$ is a set $V$ with two operations called \emph{vector addition} and \emph{scalar multiplication} where vector addition is a function $+:V\times V\rightarrow V$ and scalar multiplication is a function $\cdot: \mathbb{F}\times V\rightarrow \mathbb{V}$ such that 
\\\
\begin{enumerate}[1.] 
\setlength{\itemsep}{5pt}
\item For all $u,v\in V$ we have 
\[u+v=v+u\]
\item For all $u,v,w\in V$ and for all $a,b\in\mathbb{F}$ we have 
\[(u+v)+w=u+(v+w)\]
and 
\[(ab)\cdot v=a\cdot(b\cdot v)\]
\item There exists a vector $0\in V$, called an additive identity, such that for all $v\in V$ we have
\[v+0=v\]

\item For all $v\in V$ we have a vector $w\in V$, called an additive inverse, such that 
\[v+w=0\]
\item For all $v\in V$ we have 
\[1\cdot v=v\]
\item For all $a,b \in \mathbb{F}$ and for all $u,v\in V$ we have
\[a\cdot(u+v)=a\cdot u+a\cdot v\] 
\end{enumerate}
\end{definition}
\ \\
{\bf \noindent Some examples:} All of the following examples are with their standard operations. 
\begin{enumerate}[1.]
\setlength{\itemsep}{5pt}
\item $\mathbb{F}^n=\left\{\begin{bmatrix}a_1\\ \vdots \\ a_n\end{bmatrix}: a_i\in \mathbb{F}\right\}$ where $\mathbb{F}$ is a field. 
\item Polynomials with coefficients in a field $\mathbb{F}$.
\item Polynomials ( with coefficients in a field $\mathbb{F}$) of degree $\leq n$
\item Continuous functions $f:X\rightarrow Y$, $C(X,Y)$, where $X$ and $Y$ are fields. 
\item Functions from a field $X$ into a field $Y$. 
\item $\mathbb{F}^\infty=\{(a_1, a_2, a_3, \ldots): a_i\in \mathbb{F}\}$. 
\end{enumerate}
\ \\
\begin{proposition}Every vector space $V$ has a unique additive identity. The unique additive identity is denoted $0$. 
\end{proposition}

\begin{proposition}Every element $v\in V$ has a unique additive inverse. For all $v\in V$ its unique additive inverse is denoted $-v$. 
\end{proposition}

\begin{proposition}For all $v\in V$ we have $0\cdot v=0$. 
\end{proposition}

\begin{proposition}For all $a\in \mathbb{F}$ and $0\in V$ we have $a\cdot 0=0$.
\end{proposition}

\begin{proposition}For every $v\in V$ we have $(-1)\cdot v=-v$
\end{proposition}

\end{document}









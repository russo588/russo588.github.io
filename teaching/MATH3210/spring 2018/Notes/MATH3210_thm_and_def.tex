\documentclass[12pt,letterpaper]{amsart}
\setlength{\oddsidemargin}{.0in}
\setlength{\evensidemargin}{.0in}
\setlength{\textwidth}{6.5in}
\setlength{\topmargin}{-.3in}
\setlength{\headsep}{.20in}
\setlength{\textheight}{9.in}
\usepackage[leqno]{amsmath}
\usepackage{amsfonts}
\usepackage{amssymb}
\usepackage{amsthm}
\usepackage{amssymb}
\usepackage[all]{xy}
\usepackage{graphicx}
\usepackage{enumerate}


%Here are some user-defined notations
\newcommand{\RR}{\mathbf R}  %bold R
\newcommand{\CC}{\mathbf C}  %bold C
\newcommand{\ZZ}{\mathbf Z}   %bold Z
\newcommand{\QQ}{\mathbf Q}   %bold Q
\newcommand{\rr}{\mathbb R}     %blackboard bold R
\newcommand{\cc}{\mathbb C}    %blackboard bold R
\newcommand{\zz}{\mathbb Z}    %blackboard bold R
\newcommand{\qq}{\mathbb Q}   %blackboard bold Q
\newcommand{\ZZn}[1]{\ZZ/{#1}\ZZ}
\newcommand{\zzn}[1]{\zz/{#1}\zz}
\newcommand{\calM}{\mathcal M}  %calligraphic M
\newcommand{\sm}{\setminus} 
\newcommand{\bfa}{\mathbf a}
\newcommand{\bfb}{\mathbf b}
\newcommand{\bfc}{\mathbf c}


%improving spacing in tables (space above and below characters in a row)
\newcommand{\tfix}{\rule{0pt}{2.6ex}}
\newcommand{\bfix}{\rule[-1.2ex]{0pt}{0pt}}


%Here are commands with variable inputs 
\newcommand{\intf}[1]{\int_a^b{#1}\,dx}
\newcommand{\intfb}[3]{\int_{#1}^{#2}{#3}\,dx}
\newcommand{\pln}[1]{$\sm${\tt #1}}
\newcommand{\bgn}[1]{$\tt {\sm}begin\{#1\}$}
\newcommand{\nd}[1]{$\tt {\sm}end\{#1\}$}
\newcommand{\marginalfootnote}[1]{%
        \footnote{#1}
        \marginpar[\hfill{\sf\thefootnote}]{{\sf\thefootnote}}}
\newcommand{\edit}[1]{\marginalfootnote{#1}}


%Here are some user-defined operators
\newcommand{\Tr}{\operatorname {Tr}}
\newcommand{\GL}{\operatorname {GL}}
\newcommand{\SL}{\operatorname {SL}}
\newcommand{\Prob}{\operatorname {Prob}}
\newcommand{\re}{\operatorname {Re}}   %new definition of "real part" operator
\newcommand{\im}{\operatorname {Im}}   %new definition of "imaginary part" operator


%These commands deal with theorem-like environments (i.e., italic)
\theoremstyle{plain}
\newtheorem{theorem}{Theorem}[section]
\newtheorem{proposition}{Proposition}[section]
\newtheorem{corollary}[theorem]{Corollary}
\newtheorem{lemma}[theorem]{Lemma}
\newtheorem{conjecture}[theorem]{Conjecture}

%These deal with definition-like environments (i.e., non-italic)
\theoremstyle{definition}
\newtheorem{definition}[theorem]{Definition}
\newtheorem{example}[theorem]{Example}
\newtheorem{remark}[theorem]{Remark}

%This numbers equations by section
\numberwithin{equation}{section}


%This is for hypertext references
\usepackage{color}
\usepackage{hyperref}


\begin{document}
\section{Basic Notions}
\begin{definition} A binary operation on a set $S$ is a function $f:S\times S\rightarrow S$. 
\end{definition}
\begin{definition} A \emph{field} is a set $\mathbb{F}$ together with two binary operations $+$, and $\cdot$ called addition and multiplication (respectively) such that 
\ \\
\begin{enumerate}[1.]
\setlength{\itemsep}{5pt}
\item For all $a,b,c\in \mathbb{F}$ we have 
\[a+(b+c)=(a+b)+c\]
and 
\[a\cdot(b\cdot c)=(a\cdot b)\cdot c.\]
\item For all $a,b\in \mathbb{F}$ we have 
\[a+b=b+a\]
and 
\[a\cdot b=b\cdot a.\]
\item There exists an element $0\in \mathbb{F}$, called an additive identity,  such that for all $a\in \mathbb{F}$ we have $a+0=a$. 
\item There exists an element $1\in \mathbb{F}$, called a multiplicative identity, such that for all $a\in \mathbb{F}$ we have $a\cdot 1=a$.
\item For all $a\in \mathbb{F}$ there exists an element $b\in \mathbb{F}$, called an additive inverse, such that $a+b=0.$ 
\item For all $a\in \mathbb{F}$ such that $a\neq 0$ there exists an element $c\in \mathbb{F}$, called a multiplicative inverse, such that $a\cdot c=1$.
\item  For all $a,b,c\in \mathbb{F}$ 
\[a\cdot (b+c)=(a\cdot b)+ (a\cdot c)\]

\end{enumerate}
\end{definition}
\ \\
{\bf \noindent Note:} Fields have a unique additive and multiplicative identity denoted 0 and 1 respectively. Moreover, when the additive and multiplicative inverses exist they are unique.\\
\ \\
{\bf \noindent Some examples:} All of the following examples are with their standard operations. 
\begin{enumerate}[1.]
\setlength{\itemsep}{5pt}
\item $\mathbb{Q}$ (rational numbers)
\item $\mathbb{R}$ (real numbers)
\item $\mathbb{C}$ (complex numbers)
\item $\mathbb{Z}/p\mathbb{Z}$ for $p$ prime (Integers modulo p)
\end{enumerate}
\ \\
{\bf Non example: }$\mathbb{Z}$ is not a field, it lacks multiplicative inverses. 
\ \\
\begin{definition} A \emph{vector space} $V$ over a field $\mathbb{F}$ is a set $V$ with two operations called \emph{vector addition} and \emph{scalar multiplication} where vector addition is a function $+:V\times V\rightarrow V$ and scalar multiplication is a function $\cdot: \mathbb{F}\times V\rightarrow V$ such that 
\\
\begin{enumerate}[1.] 
\setlength{\itemsep}{5pt}
\item For all $u,v\in V$ we have 
\[u+v=v+u\]
\item For all $u,v,w\in V$ and for all $a,b\in\mathbb{F}$ we have 
\[(u+v)+w=u+(v+w)\]
and 
\[(ab)\cdot v=a\cdot(b\cdot v)\]
\item There exists a vector $0\in V$, called an additive identity, such that for all $v\in V$ we have
\[v+0=v\]

\item For all $v\in V$ we have a vector $w\in V$, called an additive inverse, such that 
\[v+w=0\]
\item For all $v\in V$ we have 
\[1\cdot v=v\]
\item For all $a,b \in \mathbb{F}$ and for all $u,v\in V$ we have
\[a\cdot(u+v)=a\cdot u+a\cdot v\] 
\end{enumerate}
\end{definition}
\ \\
{\bf \noindent Some examples:} All of the following examples are with their standard operations. 
\begin{enumerate}[1.]
\setlength{\itemsep}{5pt}
\item $\mathbb{F}^n=\left\{\begin{bmatrix}a_1\\ \vdots \\ a_n\end{bmatrix}: a_i\in \mathbb{F}\right\}$ where $\mathbb{F}$ is a field. 
\item Polynomials with coefficients in a field $\mathbb{F}$.
\item Polynomials ( with coefficients in a field $\mathbb{F}$) of degree $\leq n$
\item Continuous functions $f:X\rightarrow Y$, $C(X,Y)$, where $X$ and $Y$ are fields. 
\item Functions from a field $X$ into a field $Y$. 
\item $\mathbb{F}^\infty=\{(a_1, a_2, a_3, \ldots): a_i\in \mathbb{F}\}$. 
\end{enumerate}
\ \\
\begin{proposition}Every vector space $V$ has a unique additive identity. The unique additive identity is denoted $0$. 
\end{proposition}

\begin{proposition}Every element $v\in V$ has a unique additive inverse. For all $v\in V$ its unique additive inverse is denoted $-v$. 
\end{proposition}

\begin{proposition}For all $v\in V$ we have $0\cdot v=0$. 
\end{proposition}

\begin{proposition}For all $a\in \mathbb{F}$ and $0\in V$ we have $a\cdot 0=0$.
\end{proposition}

\begin{proposition}For every $v\in V$ we have $(-1)\cdot v=-v$
\end{proposition}
\newpage
\section{Basis for a Vector Space}

\begin{definition} A \emph{linear combination} of a list of vectors $v_1,\ldots, v_m$ in $V$ is a vector of the form 
\[a_1v_1+\ldots+a_mv_m\]
where $a_1,\ldots, a_m\in \mathbb{F}$. 
\end{definition}

\begin{definition} The set of all linear combinations of a list of vectors $v_1, \ldots, v_m$ in $V$ is called the \emph{span} of $v_1, \ldots,v_m$ denoted by 
$\text{span}\{v_1,\ldots,v_m\}$.
\[\text{span}\{v_1,\ldots,v_m\}=\{a_1v_1+\ldots+a_mv_m\mid a_i\in \mathbb{F}\}\]
\end{definition}

\begin{definition} If $V$ is a vector space and $V=\text{span}\{v_1,\ldots,v_m\}$ then we say that $v_1, \ldots, v_m$ span $V$. 
\end{definition}

\begin{definition} We say that a vectors space is \emph{finite dimensional} if there exists a finite list of vectors $v_1,\ldots, v_m$ such that 
\[\text{span}\{v_1,\ldots v_m\}=V\] 
Otherwise we say that $V$ is \emph{infinite dimensional}.
\end{definition}


\begin{definition} A list of vectors $v_1,\ldots, v_m$ in $V$ is called \emph{linearly independent} if the only choice of $a_1, \ldots, a_m\in \mathbb{F}$ such that 
\[a_1v_1+\ldots+a_mv_m=0\] is $a_1=a_2=\ldots =a_m$. A list is called \emph{linearly dependent} if it is not linearly independent.
\end{definition}

\begin{lemma} Suppose that $v_1,\ldots, v_m$ is a linearly dependent list in $V$. There exists a $j\in\{1,\ldots, m\}$ such that \\
\begin{enumerate}[1)]
\item $v_j\in \text{span}\{v_1,\ldots v_{j-1}\}$
\item $\text{span}\{v_1,\ldots, v_{j-1},v_j,v_{j+1},\ldots v_m\}=\text{span}\{v_1,\ldots, v_{j-1},v_{j+1},\ldots v_m\}$
\end{enumerate}
\end{lemma}

\begin{proposition} In a finite dimensional vector space the length of every linearly independent list of vectors is less than or equal to the length of every spanning list of vectors. 
\end{proposition}


\begin{definition} A basis for a vector space $V$ is a list of vectors $\{v_1, \ldots, v_n\}$ such that 
\begin{enumerate}[1.]
\item $\{v_1, \ldots, v_n\}$ is linearly independent
\item $\text{span}\{v_1, \ldots, v_n\}=V$. 
\end{enumerate}
\end{definition}
\begin{proposition} A list of vectors $\{v_1, \ldots, v_n\}$ in $V$ is a basis for $V$ if and only if every vector $v\in V$ can be written \emph{uniquely} in the form
\[v=a_1v_1+\ldots+a_nv_n\]
for some $a_1, \ldots, a_n\in \mathbb{F}$. 
\end{proposition}
\begin{proposition} Every spanning list of vectors in $V$ can be reduced down to a basis. 
\end{proposition}

\begin{proposition} Every linearly independent list of vectors in $V$ can be extended to a basis.  
\end{proposition}

\begin{proposition} Any two basis of a finite dimensional vector space $V$ have the same length. 
\end{proposition}

\begin{definition} The dimension of a finite dimensional vector space is the length of any basis of the vector space. Denoted $\text{dim}(V)$. 
\end{definition}

\begin{proposition} Suppose $V$ is finite dimensional. Every list of linearly independent vectors whose length is equal to the dimension of $V$ is a basis. 
\end{proposition}

\begin{proposition} Suppose $V$ is finite dimensional. Every spanning list vectors whose length is equal to the dimension of $V$ is a basis. 
\end{proposition}

\section{Subspaces}
\begin{definition} A subspace of a vector space $V$ is a subset $H$ such that $H$ is a vector space under the same binary relations and field as $V$. 
\end{definition}

\begin{proposition}[Subspace Test] A subset $H$ is a subspace of $V$ if and only if 
\begin{enumerate}[1.] 
\item $0\in H$. 
\item For all $u,v\in H$ we have $u+v\in H$
\item For all $u\in H$ and $a\in \mathbb{F}$ we have $au\in H$.
\end{enumerate}

\begin{proposition} If $U$ is a subspace of a finite dimensional vector space $V$ then $\text{dim}(U)\leq\text{dim}(V)$. Moreover, $\text{dim}(U)=\text{dim}(V)$ if and only if $V=U$. 
\end{proposition}

\end{proposition}
\begin{definition} Suppose $U_1,\ldots, U_m$ are subsets of $V$. The \emph{sum} of $U_1, \ldots, U_m$ denoted $U_1+\ldots+U_m$ is the set of all possible sums i.e., 
\[U_1+\ldots+U_m=\{u_1+\ldots+u_m\ |\ u_i\in U_i, i=1,\ldots, m\}\]
\end{definition}

\begin{proposition} If $U_1, \ldots, U_m$ are subspaces then so is $U_1+\ldots+U_m$. 
\end{proposition}

\begin{definition} Suppose $U_1,\ldots, U_m$ are subsspaces of $V$. The sum $U_1+\ldots+U_m$ is a \emph{direct sum} if each element of $U_1+\ldots+U_m$ can be written in only one way as a sum $u_1+\ldots +u_m$ where $u_i\in U_i$, $i=1,\ldots,m$. The direct sum is denoted $U_1\oplus \ldots \oplus U_m$. 
\end{definition}

\begin{proposition}  $U_1+\ldots+U_m$ is a direct sum if and only if the only way to write $0$ as a sum is by taking each $u_i$ where $i=1, \ldots, m$ to be $0$. 
\end{proposition}

\begin{proposition}  The sum of two subspaces $U$ and $W$ is a direct sum if and only if\\ $U\cap W=\emptyset$. 
\end{proposition}

\begin{proposition} If $V$ is a finite dimensional vector space and $U$ is a subspace of $V$ then there exists a $W$ which is a subspace of $V$ such that $V=U\oplus W$. 
\end{proposition}

\subsection{Quotient Spaces} 

\begin{definition}Let $V$ be a vectors space and $U$ a subspace. For every $v\in V$ define 
\[v+U=\{v+u\ \mid\  u\in U\}\]
and 
\[V/U=\{v+U\ \mid\  v\in V\}\]
\end{definition}
\begin{proposition}Let $V$ be a vectorspace, $U$ a subspace and $v,w\in V$.  The following are equivalent. 
\begin{enumerate}[(a)]
\item $v-w\in U$
\item $v+U=w+U$
\item $(v+U)\cap (w+U)\neq \emptyset$
\end{enumerate}
\end{proposition}

\begin{proposition} Let $V$ be a vectorspace, $U$ a subspace, $\lambda\in \mathbb{F}$, and $v,w\in V$The set $V/U$ is a vector space with the following operations: 
\[(v+U)+(w+U)=(v+w)+U\]
\[\lambda(v+U)=(\lambda v)+U\]
\end{proposition}


\begin{proposition} Let $V$ be a finite dimensional vector space and $U$ be a subspace. 
\[\text{dim}(V/U)=\text{dim}(V)-\text{dim}(U)\]
\end{proposition}

\section{Linear Maps}
\begin{definition} A linear map from $V(\mathbb{F})$ to $W(\mathbb{F})$ is a function $T:V\rightarrow W$ such that 
\[T(\lambda x+y)=\lambda T(x)+T(y)\]
for every $x,y\in V$ and $\lambda\in \mathbb{F}$. Denote the set of all linear maps from $V$ to $W$ as $\mathcal{L}(V,W)$. 
\end{definition}
\end{document}









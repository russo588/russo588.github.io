\documentclass[12pt,letterpaper]{article}
\usepackage[margin=1in]{geometry}
\usepackage{amsfonts}
\usepackage{amssymb}
\usepackage{amsthm}
\usepackage{amsmath}
\usepackage{enumerate}

%Here are some user-defined notations
\newcommand{\RR}{\mathbf R}  %bold R
\newcommand{\CC}{\mathbf C}  %bold C
\newcommand{\ZZ}{\mathbf Z}   %bold Z
\newcommand{\QQ}{\mathbf Q}   %bold Q
\newcommand{\rr}{\mathbb R}     %blackboard bold R
\newcommand{\cc}{\mathbb C}    %blackboard bold R
\newcommand{\zz}{\mathbb Z}    %blackboard bold R
\newcommand{\qq}{\mathbb Q}   %blackboard bold Q
\newcommand{\calM}{\mathcal M}  %calligraphic M
\newcommand{\sm}{\setminus} 
\newcommand{\bfa}{\mathbf a}
\newcommand{\bfb}{\mathbf b}
\newcommand{\bfc}{\mathbf c}
\newcommand{\lub}{\text{lub}}

\usepackage{tikz}
\usetikzlibrary{positioning}
\usepackage{graphicx}


%Here are some user-defined operators
\newcommand{\re}{\operatorname {Re}}
\newcommand{\im}{\operatorname {Im}}


%These commands deal with theorem-like environments (i.e., italic)
\theoremstyle{plain}
\newtheorem{theorem}{Theorem}[section]
\newtheorem{corollary}[theorem]{Corollary}
\newtheorem{lemma}[theorem]{Lemma}
\newtheorem{conjecture}[theorem]{Conjecture}

%These deal with definition-like environments (i.e., non-italic)
\theoremstyle{definition}
\newtheorem{definition}[theorem]{Definition}
\newtheorem{example}[theorem]{Example}
\newtheorem{remark}[theorem]{Remark}
\newtheorem*{solution}{Solution}
%your name and date in the header.
\usepackage[us]{datetime} 
\usepackage{fancyhdr}
\pagestyle{fancy}
\lhead{}
\chead{MATH 2710\\ Homework \#}
\rhead{ Your name \\ date}
\lfoot{}
\cfoot{}
\rfoot{\thepage}
\renewcommand{\headrulewidth}{0 pt}
\renewcommand{\footrulewidth}{0 pt}
\begin{document}
This is meant to be a template for your homework. In the .tex file you will see how the questions are numbered and how the equations are coded. 
%the code below allows for an automatic numbering. 
\begin{enumerate}[1.]
\item Here is the question for number one. 
%here is the proof environment. 
\begin{solution} Here is the solution for number one. 
\begin{proof} Normally a proof would go here. Instead we will write some nonsense. We can write equations inline like this $x^2+3x+3=0$ or we can do displayed equations like this 
\[2=\sum_{n=0}^\infty \frac{1}{2^n}.\]
Do not forget to use proper punctuation and spelling. 
\end{proof}
\end{solution}
\item Here is the question for number two. 
\begin{solution} Here is the solution for number two. Blah blah blah. 
\begin{proof} Another proof happens here. 
\end{proof}
\end{solution}
\end{enumerate}
\end{document}










\documentclass[11pt]{article}
\usepackage{amsmath, amssymb}
\usepackage{enumerate}
\usepackage{pdfpages}
\usepackage{tabularx}
\usepackage{graphicx}
\usepackage[margin=1.0 in]{geometry}
\usepackage{hyperref}
\newlength\tindent
\setlength{\tindent}{\parindent}
\setlength{\parindent}{0pt}
\renewcommand{\indent}{\hspace*{\tindent}}
\hypersetup{colorlinks=true,  urlcolor=blue }
\begin{document}
\begin{minipage}[]{.7\textwidth}
{\bf Mathematics Department}\\
{\bf  Methods in Operations Research $\mid$ MTH 390 $\mid$ 3 Credits}
\end{minipage}\begin{minipage}[]{.3\textwidth}\begin{flushright}\includegraphics[height=1.5cm]{farmingdalelogo.jpg}\end{flushright}\end{minipage}\\
\ \\ 
\ \\
{\bf Prerequisites/Corequisites:}  MTH 130 or MTH 150.\\

{\bf Type of Instruction: }Lecture\\

{\bf Meeting Days and Times: }\begin{tabular}[t]{rrr}
Monday &10:50 am - 12:05 pm & GLSN 129\\
Wednesday &10:50 am - 12:05 pm & LUPT 233\\
\end{tabular}\\
%%%%%%%%%%%%%%%%%%%%%%%%%%%%%%%%%%%%%%%%%%%%%%%%%%%%%%%%%%%%%%%%%%%%%%%%%%%

{\bf Benjamin Russo}\\
Office: Whitman Hall 180-I\\
Email:  \href{mailto:russobp@farmingdale.edu}{russobp@farmingdale.edu}\\
Website: \href{http://www.benrussomath.com}{www.benrussomath.com}\\

{\bf General Education:}
This class fulfills 3 credits of the mathematics competency area of the General Education Requirements at Farmingdale State College.\\

{\bf Catalog Course Description:}
This course is intended to focus on understanding, formulating and solving deterministic models in operations research. Maximum and Minimum Linear Programming problems will be studied graphically and theoretically. The Simplex Method, Sensitivity Analysis and Duality will be covered and an in-depth analysis of the reasoning on which these topics are based will be given. Instruction in computer software techniques will be presented to solve Linear Programming problems, using the simplex method and sensitivity analysis. Transportation Problems, Integer Programming, or Markov Chains will be covered. In order to enhance quantitative reasoning, the course emphasizes the formulation of mathematical models commonly used by operation research analysts, as well as the theoretical and computer software solutions to these models.\\

{\bf Course Learning Outcomes}: Upon successful completion of the course a student will be able to 
\begin{enumerate}[---]
\item Solve problems using graphical method and using the associated computer software
\item Identify and interpreting basic variables, non-basic variables, basic feasible solutions and reduced costs. 
\item Comprehend sensitivity analysis.
\end{enumerate}

{\bf General Course Requirements:} 
\begin{center}
{\renewcommand{\arraystretch}{1.2}%
\begin{tabular}{|l|r|}
\hline
Participation & 15\%\\
Exam 1& 25\% \\
Exam 2& 25\% \\
Final Exam& 35\% \\
\hline
\end{tabular}}
\end{center}
\subsubsection*{Participation}
Participation will be in the format of frequent inclass quizzes. 
\subsubsection*{Homework}
There will be a list of suggested homework problems. These problems will not be collected but it is imperative that you do them in order to practice for the exams. 
\subsubsection*{Exams}
There will be two mid-terms worth 25\% and a cumulative final worth 35\%. The midterms will be in-class on the dates indicated in the calendar. The date of the final is set by the registrar. 
\subsubsection*{Grade Scale}
\begin{center}
{\renewcommand{\arraystretch}{1.2}%
\begin{tabular}{|l|l|}
\hline
Grade & Minimum \%\\
\hline
 A& 93\\
 A$-$& 90\\
 B$+$& 87\\
  B& 83\\
 B$-$& 80\\
C$+$& 77\\
 C& 73\\
 C$-$& 70\\
 D$+$& 67\\
 D& 60\\
 F& 0\\
\hline
\end{tabular}}
\end{center}
%%%%%%%%%%%%%%%%%%%%%%%%%%%%%%%%%%%%%%%%%%%%%%%%%%%%%%%%%%%%%%%%%%%%%%%%
\subsubsection*{Required Materials}
\begin{enumerate}[--]
 \item Textbook - Introduction to Mathematical Programming by Winston Venkataramanan.
 \item Graphing Calculator
\item MATLAB and Excel
\end{enumerate}
\emph{Note:} Calculators with a computer algebra system (C.A.S) are not allowed. \\

{\bf Makeups:} Make-up exams and quizzes will be given to students who miss exams for valid reasons at the discretion of the instructor. In general, acceptable reasons for absence from class include illness, serious family emergencies, special curricular requirements (e.g., field trips, professional conferences), military obligation, severe weather conditions, religious holidays and participation in official university activities such as music performances, athletic competition or debate. Absences from class for court-imposed legal obligations (e.g., jury duty or subpoena) will be excused. Other reasons also may be approved. In addition, if you are already aware of a conflict with an exam date, then you need to discuss this with your instructor within the first two weeks of class. \\

{\bf Religious Absences:} If you are unable to attend class on certain days due to religious beliefs, please consult with your instructor well in advance of the absence so that appropriate accommodation can be made.\\

{\bf Disability Services Center:} If you have a disability for which you are or may be requesting an accommodation, you are encouraged to contact both your instructor and the Disability Services Center, Roosevelt Hall, Room 151, or call 631-420-2411 as soon as possible this semester.\\

{\bf Temporary Grades:} A grade of ``I" (incomplete) is reported when, for some reason beyond their control, the student misses the final examination or has not completed a portion of the required work for the course. The decision to grant an ``I" is at the sole discretion of the instructor. No achievement points are awarded for an incomplete. All incompletes must be resolved and a change of grade submitted no later than 30 days after the beginning of the next semester (fall to spring, winter intersession to spring, spring to fall, summer session to fall). An instructor may grant an extension of an incomplete grade until the end of the semester by documenting and filling the approved form with the Registrar prior to the conclusion of the 30 day period. Any incomplete grade not finalized or extended by the instructor within the 30 day time period mentioned above will automatically be changed to an ``F". An incomplete does not constitute successful completion of a prerequisite.  \\

{\bf Academic Honesty:} This course expects all students to act in accordance with the \href{https://www.farmingdale.edu/administration/provost/pdf/academic-integrity.pdf}{Academic Integrity Policy} at the Farmingdale State College.  If you work on the homework with your classmates, you must write your own solutions individually. There should be no help given or received on midterms or the final exam. Academic misconduct includes, but is not limited to, providing or receiving assistance in a manner not authorized by the instructor in the creation of work to be submitted for academic evaluation (e.g. papers, projects, examinations and assessments - whether online or in class); presenting, as one's own, the ideas, words or calculations of another for academic evaluation; doing unauthorized academic work for which another person will receive credit or be evaluated; using unauthorized aids in preparing work for evaluation (e.g. unauthorized formula sheets, unauthorized calculators, unauthorized programs or formulas loaded into your calculator, etc.); and presenting the same or substantially the same papers or projects in two or more courses without the explicit permission of the instructors involved. A student who knowingly assists another student in committing an act of academic misconduct shall be equally accountable for the violation, and shall be subject to the sanctions. Such sanctions include failing the assignment in question and depending on the severity of the incident failing the course and/or other remedies. \\

{\bf Copyright Statement:} Course material accessed from Blackboard or the Farmingdale website is for the exclusive use of students who are currently enrolled in the course. Content from these systems cannot be reused or distributed without written permission of the instructor and/or the copyright holder. Duplication of materials protected by copyright, without permission of the copyright holder, is a violation of the Federal copyright law, as well as a violation of SUNY copyright policy.\\

{\bf Cancellation of Classes:} Weather and other campus-wide cancellations will be listed on the home page, Facebook and Twitter and you can also sign up for RAVE and SUNY Alert. Go to the Rave web page and use your Farmingdale user ID and password to enter the site. For SUNY-Alert, please visit the University Police web page. You may also be notified via email of class cancellations.\\

{\bf Electronic Devices Policy: }Please silence all electronics while in class. Use of electronics during exams is prohibited.\\

{\bf Attendance Policy: } For absences on exam days please see the make-up policy above. \\

{\bf Use of Email: }It is Farmingdale State College policy that instructors and students use the Farmingdale email system or the Blackboard email system to contact one another. \\

{\bf Disclaimer:} The instructor reserves the right to adjust the syllabus. By taking this course, you acknowledge that you have read this syllabus and abide
to it and any such changes.
%\includepdf[page=1]{weekbyweekMTH151.pdf} 
\end{document}
%%%%%%%%%%%%%%%%%%%%%%%%%%%%%%%%%%%%%%%%%%%%%%%%%%%%%%%%%%%%%%%%%%%%%%%%%%%

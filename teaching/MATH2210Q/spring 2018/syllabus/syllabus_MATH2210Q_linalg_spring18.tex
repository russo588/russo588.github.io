\documentclass[11pt]{article}
\usepackage{amsmath, amssymb}
\usepackage{enumerate}
\usepackage[margin=1.0 in]{geometry}
\usepackage{hyperref}
\hypersetup{colorlinks=true,  urlcolor=blue }
\begin{document}
\begin{center}
{\LARGE Math 2210Q}\\
Applied Linear Algebra\\
\end{center}

\section*{\underline{Instructor Information}}
Ben Russo

\noindent 
Office: MONT 125\\
Email:  \href{mailto:benjamin.russo@uconn.edu}{benjamin.russo@uconn.edu}\\
Website: \href{http://www.benrussomath.com}{www.benrussomath.com}
\section*{\underline{Course Description}}
Systems of equations, matrices, determinants, linear transformations on vector spaces, characteristic values and vectors, from a computational point of view. The course is an introduction to the techniques of linear algebra with elementary applications.
\subsection*{Pre-requisites}
MATH 1132, 1152, or 2142. Recommended Preparation: a grade of C- or better in MATH 1132. Not open for credit to students who have passed MATH 3210.
\subsection*{Required Materials}
Linear Algebra And Its Applications, 5th ed. by David C. Lay, Addison-Wesley \\

\noindent A graphing calculator may be used on the quizzes and exams. However, calculators with a computer algebra system (C.A.S) are not allowed. 

%\subsection*{Canvas} Canvas is located at 
%\url{http://lss.at.ufl.edu}. Use your Gatorlink name and password to login.  Your instructor will post your grades for the course on Canvas. It is your responsibility to check your grade is posted correctly and any discrepancy must be addressed within one week of posting. 

\section*{\underline{Course Breakdown}}
\begin{center}
{\renewcommand{\arraystretch}{1.2}%
\begin{tabular}{|l|r|}
\hline
Quiz 1& 5 points\\
Quiz 2& 5 points \\
Quiz 3& 5 points\\
Exam 1& 25 points\\
Exam 2& 25 points \\
Final Exam& 35 points\\
\hline 
Total & 100 points\\
\hline
\end{tabular}}
\end{center}
\subsection*{Homework}
There will be a list of suggested homework problems. These problems will not be collected but it is imperative that you do them in order to practice for the exams. 
\subsection*{In-Class Quizzes}
There will be 3 in-class quizzes on the dates indicated on the calendar, each taking approximately 10-15 minutes. They are worth 5 points each. 
\subsection*{Tests}
There will be two mid-terms worth 25 points and a cumulative final worth 35 points. The midterms will be in-class on the dates indicated in the calendar. The date of the final is set by the registrar. 
%They will be in-class exams and occur on the dates indicated in the calendar and below.
%\begin{center}
%{\renewcommand{\arraystretch}{1.2}
%\begin{tabular}{|l|l|}
%\hline
%Exam 1 & May 20th \\
%Exam 2 & June 3rd\\
%Final & June 17th\\
%\hline 
%\end{tabular}}
%\end{center}
\section*{\underline{Grade Scale}}
\begin{center}
{\renewcommand{\arraystretch}{1.2}%
\begin{tabular}{|l|l|}
\hline
Points & Grade\\
\hline
93 -- 100  & A\\
90 -- 92 & A$-$\\
87 -- 89 & B$+$\\
83 -- 86 & B\\
80 -- 82 & B$-$\\
77 -- 79 &C$+$\\
73 -- 76 & C\\
70 -- 72& C$-$\\
67 -- 69& D$+$\\
63 -- 66& D\\
60 -- 62& D$-$\\
0 -- 59 & E\\
\hline
\end{tabular}}
\end{center}
\section*{\underline{Make-ups}}
Make-up exams and quizzes will be given to students who miss exams for valid reasons at the discretion of the instructor. In general, acceptable reasons for absence from class include illness, serious family emergencies, special curricular requirements (e.g., field trips, professional conferences), military obligation, severe weather conditions, religious holidays and participation in official university activities such as music performances, athletic competition or debate. Absences from class for court-imposed legal obligations (e.g., jury duty or subpoena) will be excused. Other reasons also may be approved. In addition, if you are already aware of a conflict with an exam date, then you need to discuss this with your instructor within the first two weeks of class. 
\section*{\underline{Temporary Grades}}
{\bf\noindent I Grade:} The instructor reports an I if the completed work is passing and the instructor decides that, due to unusual circumstances, the student cannot complete the course assignments. If the student completes the work by the end of the third week of the next, registered semester, the instructor will send the Registrar a grade for the course. Otherwise, the Registrar will convert the I to IF. Effective with spring 2004 classes, upon successful completion of a course, the I on the academic record is replaced by the permanent letter grade. If the instructor does not submit a grade the Registrar will change the grade to IF or IU.\\

{\bf\noindent N Grade:} An N grade is recorded when no grade is reported for a student who has been registered in a course section. It usually indicates a registration problem. To resolve this problem you must see the instructor who may submit a letter stating that you never attended the class. The Dean of the school in which you are enrolled must approve a late drop. N grades are replaced on the academic record by the actual grade when submitted by the instructor. An N mark which remains unresolved will become “NF” and be computed as an “F” at the end of the third week of the next semester of registration.\\

{\bf\noindent X Grade:} The instructor reports an X only when a student missed the final examination and when passing it with a high mark could have given the student a passing grade for the course. If the student would have failed the course regardless of the grade on the final examination, the student will receive an F. If the instructor reports an X and the Dean of Students Office excuses the absence, the instructor will give the student another opportunity to take the examination . The absence must be due to sickness or other unavoidable causes. The instructor must give the examination before the end of the third week of the next, registered semester. If by the end of the third week of the next, registered semester the instructor does not send a grade to the Registrar, the Registrar will change the X to X F or X U. In exceptional instances, after consulting the instructor, the Dean of Students Office may extend the time for completing courses marked I or X.
\section*{\underline{Students with Disabilities}}
The Center for Students with Disabilities (CSD) at UConn provides accommodations and services for qualified students with disabilities. If you have a documented disability for which you wish to request academic accommodations and have not contacted the CSD, please do so as soon as possible. The CSD is located in Wilbur Cross, Room 204 and can be reached at (860) 486-2020 or at \href{mailto:csd@uconn.edu}{csd@uconn.edu}. Detailed information regarding the accommodations process is also available on their website at \url{www.csd.uconn.edu}.

\section*{\underline{Academic Honesty}}
This course expects all students to act in accordance with the \href{http://community.uconn.edu/the-student-code-appendix-a/}{Guidelines for Academic Integrity} at the University of Connecticut.  If you work on the homework with your classmates, you must write your own solutions individually. There should be no help given or received on midterms or the final exam. Academic misconduct includes, but is not limited to, providing or receiving assistance in a manner not authorized by the instructor in the creation of work to be submitted for academic evaluation (e.g. papers, projects, examinations and assessments - whether online or in class); presenting, as one's own, the ideas, words or calculations of another for academic evaluation; doing unauthorized academic work for which another person will receive credit or be evaluated; using unauthorized aids in preparing work for evaluation (e.g. unauthorized formula sheets, unauthorized calculators, unauthorized programs or formulas loaded into your calculator, etc.); and presenting the same or substantially the same papers or projects in two or more courses without the explicit permission of the instructors involved. A student who knowingly assists another student in committing an act of academic misconduct shall be equally accountable for the violation, and shall be subject to the sanctions and other remedies described in Appendix A of the Student Code. 
\newpage
\section*{\underline{Disclaimer}}
This syllabus is not set in stone and I may adjust it as I deem necessary throughout
the semester. By taking this course, you acknowledge that you have read this syllabus and abide
to it and any such changes.

\end{document}
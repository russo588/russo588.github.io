\documentclass[12 pt]{article}
%Margins and packages
\usepackage[margin=1.0in]{geometry}
\usepackage{amsmath, amsthm, amssymb, enumerate}
\usepackage{xcolor}
\usepackage{graphicx}
\usepackage{enumerate}
\usepackage{lipsum}
%These commands deal with theorem-like environments (i.e., italic)
%\newtheorem{name}[counter]{display}
\theoremstyle{plain}
\newtheorem{theorem}{Theorem}[section]
\newtheorem{corollary}[theorem]{Corollary}
\newtheorem{lemma}[theorem]{Lemma}
\newtheorem{conjecture}[theorem]{Conjecture}

%These deal with definition-like environments (i.e., non-italic)
%\newtheorem{name}[counter]{display}
\theoremstyle{definition}
\newtheorem{definition}[theorem]{Definition}
\newtheorem{example}[theorem]{Example}
\newtheorem{remark}[theorem]{Remark}
\newtheorem{claim}[theorem]{Claim}

%User defined commands
\newcommand{\ip}[2]{\langle #1, #2\rangle}
\newcommand{\ran}{\text{ran}}

% Beginning of the document here
\begin{document}
%Title page starts 
\begin{titlepage}
\newgeometry{left=7.5cm} %defines the geometry for the titlepage
\noindent
\makebox[0pt][l]{\rule{1.3\textwidth}{1pt}}
\par
\noindent
\textbf{\textsf{University of Connecticut}}\ \ \ \  
{\textsf{Benjamin Russo}}
\vfill
\noindent
{\huge \textsf{Analysis I}}\\
\ \\
{\large \textsf{Title of Paper}}
\vskip\baselineskip
\noindent
\textsf{December 2017}
\end{titlepage}
%Title page ends here
\restoregeometry % restores the geometry
\section{Introduction}
\lipsum[1-2]% exposition
Here is a citation \cite{Paulsen}. This was taken from MATHSCINET.
\begin{definition} 
This is a definition.This is a definition.This is a definition.This is a definition.
\end{definition}
\section{Math Stuff Goes here}
\lipsum[3-4]
\begin{theorem}
This is a theorem.This is a theorem.This is a theorem.This is a theorem.
\end{theorem}
\lipsum[4]%Some exposition
 A displayed equation is to follow. Here is some more random text. 
\[e^{in\theta}=\cos(\theta)+\i \sin(\theta)\]
\lipsum[5]
\begin{proof}
A proof is here!A proof is here!A proof is here!A proof is here!A proof is here!
A proof is here!A proof is here!A proof is here!A proof is here!
A proof is here!A proof is here!A proof is here!A proof is here!A proof is here!
\end{proof}
\lipsum[5-9]
\begin{example}
Here is an example. Here is an example. Here is an example. Here is an example. Here is an example. 
\end{example}
Blah blah blah blah another citation \cite{Dritschel}, I got these from MATHSCINET. 
\lipsum[4-5]
\section{Another Section}
References after this section.  \lipsum[6]. 
\newpage
\bibliographystyle{plain}% sets style 
\bibliography{bibliography} % calls your bib file. 
\end{document} 
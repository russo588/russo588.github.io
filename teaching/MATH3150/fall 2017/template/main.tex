\documentclass[12 pt]{article}
%Margins and packages
\usepackage[margin=1.0in]{geometry}
\usepackage{amsmath, amsthm, amssymb, enumerate}
\usepackage{xcolor}
\usepackage{graphicx}
\usepackage{enumerate}
%These commands deal with theorem-like environments (i.e., italic)
%\newtheorem{name}[counter]{display}
\theoremstyle{plain}
\newtheorem{theorem}{Theorem}[section]
\newtheorem{corollary}[theorem]{Corollary}
\newtheorem{lemma}[theorem]{Lemma}
\newtheorem{conjecture}[theorem]{Conjecture}

%These deal with definition-like environments (i.e., non-italic)
%\newtheorem{name}[counter]{display}
\theoremstyle{definition}
\newtheorem{definition}[theorem]{Definition}
\newtheorem{example}[theorem]{Example}
\newtheorem{remark}[theorem]{Remark}
\newtheorem{claim}[theorem]{Claim}

%User defined commands
\newcommand{\ip}[2]{\langle #1, #2\rangle}
\newcommand{\ran}{\text{ran}}

% Beginning of the document here
\begin{document}
%Title page starts 
\begin{titlepage}
\newgeometry{left=7.5cm} %defines the geometry for the titlepage
\noindent
\makebox[0pt][l]{\rule{1.3\textwidth}{1pt}}
\par
\noindent
\textbf{\textsf{University of Connecticut}}\ \ \ \  
{\textsf{Benjamin Russo}}
\vfill
\noindent
{\huge \textsf{Analysis I}}\\
\ \\
{\large \textsf{The Fej\'{e}r-Riesz Theorem}}
\vskip\baselineskip
\noindent
\textsf{December 2017}
\end{titlepage}
%Title page ends here
\restoregeometry % restores the geometry
\section{Introduction}
The Fej\'{e}r-Riesz theorem gives the form of a Laurent polynomial that is nonnegative on the unit circle. We let $\mathbb{D}$ be the open unit disc and let $\mathbb{T}$ be the unit circle in the complex plane.
\begin{theorem}[Fej\'{e}r-Riesz]
A Laurent polynomial $q(z)=\sum_{k=-m}^m q_k z^k$ which has complex coefficients and satisfies $q(\zeta)\geq 0$ for all $\zeta \in \mathbb{T}$  can be written 
\[q(\zeta)=|p(\zeta)|^2, \ \zeta\in \mathbb{T}\]
for some polynomial $p(z)=p_0+p_1z+ \cdots +p_m z^m$, and $p(z)$ can be chosen to have no zeros in $\mathbb{D}$. 
\end{theorem}
The Fej\'{e}r-Riesz Theorem shows up naturally in spectral theory, the theory of orthogonal polynomials, and control theory.
In this essay we will discuss the rather elementary proof of the Fej\'{e}r-Riesz theorem and then show an application in the proof of Von Neumann's inequality \cite{Paulsen}. There is also a natural generalization to an operator theoretic version whose proof in full generality was due to Rosenblum. We will also discuss its proof as presented in \cite{Dritschel}. The theorem even has generalizations to a multivariable version and a noncommutative function theory version. The non-commutative version will be discussed but its proof will be absent. 
\section{The Scalar Fej\'{e}r-Riesz Theorem}We present the proof of Fej\'{e}r Riesz as it appears in \cite{Paulsen}. 
\begin{proof}%[Scalar Fej\'{e}r Riesz]
First note that since $q$ is real valued, then $q_{-k}=\bar{q}_k$ and $q_0$ is real (easy to see if you set $q=\bar{q}$). We may assume that $q_{-m}\neq 0$. We set $g(z)=\sum_{k=-m}^mq_kz^{k+m}$, so that $g$ is a polynomial of degree $2m$ with $g(0)\neq 0 $. We have that $g(e^{i\theta})=q(e^{i\theta})\cdot e^{im\theta}\neq0$.
Notice that 
\[\overline{g(1/\bar{z})}=z^{-2m}g(z).\]
This means the $2m$ zeros of $g$ may be written as $z_1,\ldots ,z_m$ and $1/\bar{z}_{1},\ldots ,1/\bar{z}_{m}.$ Let \[w(z)=(z-z_1)\cdot\ldots\cdot(z-z_m)\text{ and }h(z)=(z-1/\bar{z}_1)\cdot\ldots\cdot(z-1/\bar{z}_m).\] We have that 
\[g(z)=q_mw(z)h(z),\]
with 
\[\overline{h(z)}=\frac{(-1)^m\bar{z}^mw(1/\bar{z})}{z_1\cdot\ldots\cdot z_m}.\]
Thus
\begin{align*}
q(e^{i\theta})&=e^{-im\theta}g(e^{i\theta})=|g(e^{i\theta})|=|q_m|\cdot |w(e^{i\theta})|\cdot|\overline{h(e^{i\theta})}|\\
&=\left|\frac{q_m}{z_1\ldots z_m}\right| \cdot |w(e^{i\theta})|^2,
\end{align*}
so that $q(e^{i\theta})=|p(e^{i\theta})|^2$, where $p(z)=\left|\frac{q_m}{z_1\ldots z_m}\right|^{1/2} \cdot w(z)$.
\end{proof}
If we do not bog ourselves down with details we can give a quick application of the Fej\'{e}r-Riesz theorem to the theory of positive maps and operator algebras. A {\bf C$^*$-algebra} $A$ is a Banach algebra with a map $*:A\rightarrow A$ called an {\bf involution} with the following properties
\begin{enumerate}[1)]
\item $x^{**}=(x^*)^*=x$
\item $x^*+y^*=(x+y)^*$
\item$(xy)^*=y^*x^*$
\item $\|x^*x\|=\|x\|^2 $ \ \  (C$^*$ identity).
\end{enumerate}
Examples include bounded operators on a Hilbert space and complex functions on the unit circle. Let $A$ be a unital C$^*$ algebra and let $S$ be a subset, then we set 
\[S^*=\{a: a^*\in S\}.\]
We call $S$ self adjoint when $S=S^*$. If $S$ also contains the identity, then we call $S$ an operator system. Let $\phi:S\rightarrow B$ where $B$ is a C$^*$ algebra; we call $\phi$ positive if it sends positive elements to positive elements. For instance, we can take both $S$ and $B$ to be the bounded operators. We say that an operator $T$ in a Hilbert space $H$ is {\bf positive} if $\ip{Tx}{x}\geq 0$ for all $x\in H$. For future reference we note that operators of the form $T=P^*P$ are positive operators.
We can apply Fej\'{e}r-Riesz in proving the following theorem.
\begin{theorem}
Let $T$ be an operator on a Hilbert space $H$ with $\|T\|\leq 1$, and let $S\subset C(\mathbb{T})$ be the operator system defined by 
\[S=\{p(e^{i\theta})+\overline{q(e^{i\theta})}: p, q polynomials\}.\]
Then the map $\phi : S\rightarrow B(H)$ defined by $\phi(p+\bar{q})=p(T)+q(T)^*$ is positive. 
\end{theorem}
We note that $S\subset C(\mathbb{T})$ is dense. It is a standard theorem that positive maps on operator systems are bounded and so the map $\phi$ can be extended (by continuity) to $\hat{\phi}:C(\mathbb{T})\rightarrow B(H)$; $\hat{\phi}$ is also a positive map itself. We state that positive maps on $C(\mathbb{T})$ i.e. $\hat{\phi}$ automatically have the property $\|\hat{\phi}\|=\|\hat{\phi}(1)\|$. Keeping all of this in mind, the proof of the next theorem is more or less obvious.
\begin{theorem}
Let $T$ be an operator on a Hilbert space with $\|T\|\leq 1$. Then for any polynomial $p$ 
\[\|p(T)\|\leq \|p\|,\]
where $\|p\|=\sup_\theta |p(e^{i\theta})|$.
\end{theorem}
\section{The Operator Fej\'{e}r-Riesz Theorem}
We now state the extension of Fej\'{e}r-Riesz to operators. 
\begin{theorem}
Let $Q(z)=\sum_{k=-m}^m Q_kz^k$ be a Laurent polynomial with coefficients in $B(G)$ for some Hilbert space $G$. If $Q(\zeta)\geq 0$ for all $\zeta\in \mathbb{T}$, then 
\[Q(\zeta)=P(\zeta)^* P(\zeta), \ \ \ \zeta\in \mathbb{T}\]
for some polynomial $P(z)=P_0+P_1z+\cdots +P_mz^m$ with coefficients in $B(G)$. The polynomial $P(z)$ can be chosen to be outer. 
\end{theorem}
Here the inner-product for the space is given by
\[\int_{\mathbb{T}}\ip{Q(\zeta)f(\zeta)}{g(\zeta)}_G\ d\sigma(\zeta),\]
where $\sigma$ is the standard normalized Lebesgue measure on $\mathbb{T}$.
The proof relies heavily on the use of Toeplitz, analytic and shift operators. We will need a few definitions to get us started. 
\begin{definition}
An operator $S$ in $B(H)$ is a shift operator if $S$ is an isometry and $S^{*n}\rightarrow 0$ strongly, that is $\|S^{*n}f\|\rightarrow 0$ for all $f\in H$. 
\end{definition}
We also have another characterization of shift operators. 
\begin{definition}
A shift operator is an isometry $S$ on $H$ such that the unitary component of its Wold decomposition is trivial. 
\end{definition}
This seems even more abstract than the previous definition. So without delving into Wold decompositions, we will say that with some natural identifications we can write $H=G\oplus G\oplus G\oplus G\cdots $ for some Hilbert space $G$, and 
\[S(h_0,h_1,h_2,\ldots)=(0,h_0,h_1,h_2,\ldots)\]
when the elements of $h$ are written in sequence form.
\begin{definition}
We say that $T$ is an S-{\bf Toeplitz} operator or Toeplitz if $S^*TS=T$. 
\end{definition}
\begin{definition}
We say that $A$ is S-{\bf analytic} or analytic if $AS=SA$. An analytic operator $A$ is said to be {\bf outer} if $\overline{\ran}(A)$ is a subspace of $H$ of the form $F\oplus F\oplus F\oplus \cdots $ for some closed subspace $F$ of $G$. 
\end{definition}
To be a bit more concrete we can consider the block matrix form of the operators 
\begin{equation}\label{eq:blockmatrix}T=\begin{pmatrix}T_0 &T_{-1} &T_{-2 }& \cdots \\  T_1 & T_0 & T_{-1}& \ddots \\ T_2 & T_1 & T_0 & \ddots \\ \vdots & \ddots & \ddots  & \ddots\end{pmatrix}
\ \ \ A=\begin{pmatrix}A_0 &0 &0& \cdots \\  A_1 & A_0 & 0& \ddots \\ A_2 & A_1 & A_0 & \ddots \\ \vdots & \ddots & \ddots  & \ddots\end{pmatrix}.\end{equation}
Here we define 
\begin{equation}\label{eq:Toeplitzenty}T_j=\left\{ \begin{array}{lr} E^*_0 S^{*j}TE_0|_{G}& j\geq 0\\  E^*_0 TS^{|j|}E_0|_{G}& j<0\end{array}\right.;\ \ \ \ \ \ A_j=E_0S^{*j}AE_0\end{equation} 
and 
\[E_0g=(g,0, 0, \ldots ).\]
Given the polynomials  $Q(z)=\sum_{k=-m}^m Q_kz^k$ and $P(z)=P_0+P_1z+\cdots +P_mz^m$  with bounded operator coefficients, consider the following block matrices. Set $Q_j=0$ for $|j|>m$ and $P_j=0$ for $j>m$ and let
\begin{equation}\label{eq:QPmats}
T_Q=\begin{pmatrix}Q_0 &Q_{-1} &Q_{-2 }& \cdots \\  Q_1 & Q_0 & Q_{-1}& \ddots \\ Q_2 & Q_1 & Q_0 & \ddots \\ \vdots & \ddots & \ddots  & \ddots\end{pmatrix}
\ \ \ T_P=\begin{pmatrix}P_0 &0 &0& \cdots \\  P_1 & P_0 & 0& \ddots \\ P_2 & P_1 & P_0 & \ddots \\ \vdots & \ddots & \ddots  & \ddots\end{pmatrix}.
\end{equation} 
We have the following identity,
\begin{equation}\label{eq: innerprod}\int_{\mathbb{T}}\ip{Q(\zeta)f(\zeta)}{g(\zeta)}\ d\sigma(\zeta)=\sum_{j,k=0}^m\ip{Q_{j-k}f_k}{g_j}_G,\end{equation}
where $f(\zeta)=f_0+f_1\zeta +f_2\zeta^2+\cdots$  and $g(\zeta)=g_0+g_1\zeta +g_2\zeta^2+\cdots$ with coefficients in $G$, and all but finitely many are non-zero.
The operators $T_Q$ and $T_P$ are clearly Toeplitz and analytic respectively. They are bounded by use of (\ref{eq: innerprod}) and a density argument. More importantly, we have $Q(\zeta)\geq0$ for all $\zeta\in \mathbb{T}$ if and only if $T_Q\geq 0$ and $Q(\zeta)=P^*(\zeta)P(\zeta)$ for all $\zeta\in \mathbb{T}$ if and only if $T_Q=T_P^*T_P$. Harking back to the statement of the the theorem we will say $P(z)$ is {\bf outer} if and only if $A=T_P$ is outer. So we see that we can encode our problem into a problem about Toeplitz and analytic block matrix operators. More specifically, if given a positive Toeplitz operator we want to write it in the form $T=A^*A$ for $A$ analytic. The other half of the statement of the theorem is if $T=T_Q$, then we want to show we can pick an $A=T_P$ such that $A$ is outer i.e. $P(z)$ is outer. We reduce this problem further by using a couple of lemmas. 
\begin{lemma}[Lowdenslager's Criterion]\label{Lowdenslagers_Crit}
Let $H$ be a Hilbert space and let $S\in B(H)$ be a shift operator. Let $T\in B(H)$ be S-Toeplitz and suppose that $T\geq 0$. Let $H_T$ be the closure of the range of $T^{1/2}$ in the inner product of $H$. Then there is an isometry $S_T$ mapping $H_T$ into itself such that 
\[S_TT^{1/2}f=T^{1/2}Sf \ \ \ \ f\in H.\]
In order for $T=A^*A$ for some operator $A\in B(H)$, it is necessary and sufficient that $S_T$ be a shift operator. In this case $A$ can be chosen to be outer. 
\end{lemma}
The existence of the isometry $S_T$ is not difficult to see. Given the identity $S^*TS=T,$ we can see that 
\[T=S^*TS=S^*T^{*1/2}T^{1/2}S=(T^{1/2}S)^*(T^{1/2}S),\]
and thus
\[\|T^{1/2}\|^2=\|T^{*1/2}T^{1/2}\|=\|(T^{1/2}S)^*(T^{1/2}S)\|=\|T^{1/2}S\|^2.\]
So $T^{1/2}S$ and $T^{1/2}$ have the same norm. The rest of the proof is fairly simple (pardoning some perhaps difficult details). If $S_T$ is a shift operator it allows you to define some isometry $W:H_T\rightarrow H$ such and we defined $Af=WT^{1/2}f$. Conversely, if $T=A^*A$, then we define the isometry by again letting $WT^{1/2}f=Af$.  We can then show $WS_T=SW$ and thus $S_T^{*n}=W^*S^{*n}W$ converges strongly to 0 since $S$ is a shift operator. The next lemma shows that if our Toeplits operator is of the form $T_Q$ for some Laurent polynomial $Q$, then our needed operator $A$ is of the form $T_P$ for some polynomial $P$. 
\begin{lemma}
In Lemma(\ref{Lowdenslagers_Crit}), let $T=T_Q$ be given by (\ref{eq:QPmats}) for a Laurent polynomial of degree $m$. If $T=A^*A$, where $A\in B(H)$ is analytic and outer, then $A=T_P$ for some outer analytic polynomial $P(z)$ of degree $m$.
\end{lemma}
\noindent This proof consists mostly of showing the lower triangular entries have the correct form. \\
We are now prepared to prove the main theorem. The previous two lemmas reduce the problem to showing that $S_T$ is a shift operator. We make the following claim.
\begin{claim}
If $f=T^{1/2}fh$ where $h\in H$ has the form $h=(h_0, h_1, \ldots , h_r, 0 ,\ldots)$, then $S_T^{*n}$ converges strongly to $0$. 
\end{claim}
For if $u\in H$ and $n$ is any positive integer, then 
\[\ip{S_T^{*n}f}{T^{1/2}u}_{H_T}=\ip{f}{S_T^{n}T^{1/2}u}_{H_T}=\ip{T^{1/2}h}{T^{1/2}S^nu}_H=\ip{Th}{S^nu}_H.\]
By definition of $T=T_Q$, $Th$ has only a finite number of nonzero entries (depending on $m$ and $r$), and the first $n$ entries of $S^nu$ are zero (independent of $u$). The claim then follows by the arbitrariness of $u$. By the claim, $\|S_T^{*n}f\|\rightarrow 0$ for all $f$ in a dense subset of $H_T=\overline{\ran}(T^{1/2})$, and thus holds on the entire space.
\section{Non-Commutative Fej\'{e}r-Riesz} We will discuss the statement of the non-commutative version of Fej\'{e}r-Riesz, a theorem due to Scott McCullough as stated in \cite{Dritschel}. Let $S$ be the free semigroup with generators $a_1, \ldots a_d$ with the binary operation of concatonation. Let $G$ be a free group over a finite number of generators, i.e. words in $a_1,\ldots,a_d$ and $a_1^{-1},\ldots, a_d^{-1}$ with the binary operation of concatenation. Words in $G$ of the form $h=v^{-1}w$  with $v, w\in S$ will be called {\bf hereditary words}. Notice that a hereditary word $h$ can have many representations $h=v^{-1}w$ with $v,w\in S$. Let $S_m$ be the words of length at most $m$, and let $H_m$ be the set of hereditary words $h$ which have at least one representation in the form $h=v^{-1}w$ with $v,w\in S_m$. In a formal algebraic manner, we build our non-commutative analogues of Laurent polynomials and analytic polynomials. A hereditary polynomial is a formal expression of the form
\[Q=\sum_{h\in H_m}h\otimes Q_h,\]
where $Q_h\in B(G)$, i.e. is a bounded operator over a Hilbert space $G$ for all $h$. Analytic polynomials are hereditary polynomials of the form
\[P=\sum_{w\in H_m}w\otimes P_w,\]
where $P_w\in B(G)$ for all of $w$. The identity 
\[Q=P^*P\] 
is defined to mean 
\[Q_h=\sum_{\substack{v,w\in S_m\\ h=v^{-1}w}}P_v^*P_w.\]
Let $U=(U_1, \ldots,U_d)$ be a $d$-tuple of unitary operators on a Hilbert space $H$. Define $U^w$ for any $w\in S$ by writting $w$ in the form 
\[w=a_{j_1}\cdots a_{j_k},\ \ \ \ \ j_1,\ldots ,j_k\in \{1,\ldots ,d\},\ \ \ \ \  k=0,1,2,\ldots\] and setting 
\[U^w=U_{j_1}\cdots U_{j_k},\]
where for the empty word $e$ we set 
\[U^e=I.\]
If $h$ is a hereditary word, set 
\[U^h=(U^v)^*U^w\] for any representation $h=v^{-1}w$ with $v,w\in S$ (this does not depend on choice of representation). Finally define $Q(U)$ and $P(U) \in B(H\otimes G)$ by
\[Q(U)=\sum_{h\in H_m}U^h\otimes Q_h\ \text{and}\ P(U)=\sum_{w\in S_m}U^w\otimes P_w .\]
\begin{theorem}
Let 
\[Q=\sum_{h\in H_m}h\otimes Q_h\]
be a hereditary polynomial with coefficients in $B(G)$ for some Hilbert space $G$ such that $Q(U)\geq 0$ for every tuple $U=(U_1,\ldots , U_d)$ of unitary operators  in some Hilbert space $H$. Then for some $\ell\leq \ell_m$ there exists analytic polynomials 
\[P_j=\sum_{w\in S_m}w\otimes P_{j,w}, \ \ \ \ \ j=1,\ldots \ell,\]
with coefficients in $B(G)$ such that 
\[Q=P_1^*P_1+\ldots +P_\ell^*P_\ell.\]
Moreover, for any tuple $U=(U_1, \ldots U_d$) of unitary operators on $H$, 
\[Q(U)=P_1(U)^*P_1(U)+\ldots +P_{\ell}(U)^*P_{\ell}(U).\]
In these statements when $G$ is infinite dimensional, we can choose $\ell=1$. 
\end{theorem}
When $d=1$, from this theorem we can deduce (after a bit of work) the original version of the operator Fej\'{e}r-Riesz.


\newpage
\bibliographystyle{plain}
\bibliography{bibilography}
\end{document} 
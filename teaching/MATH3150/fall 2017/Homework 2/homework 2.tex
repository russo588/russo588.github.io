\documentclass[12pt,letterpaper]{article}
\usepackage[margin=1in]{geometry}
\usepackage{amsfonts}
\usepackage{amssymb}
\usepackage{amsthm}
\usepackage{amsmath}
\usepackage{enumerate}

%Here are some user-defined notations
\newcommand{\RR}{\mathbf R}  %bold R
\newcommand{\CC}{\mathbf C}  %bold C
\newcommand{\ZZ}{\mathbf Z}   %bold Z
\newcommand{\QQ}{\mathbf Q}   %bold Q
\newcommand{\rr}{\mathbb R}     %blackboard bold R
\newcommand{\cc}{\mathbb C}    %blackboard bold R
\newcommand{\zz}{\mathbb Z}    %blackboard bold R
\newcommand{\qq}{\mathbb Q}   %blackboard bold Q
\newcommand{\calM}{\mathcal M}  %calligraphic M
\newcommand{\sm}{\setminus} 
\newcommand{\bfa}{\mathbf a}
\newcommand{\bfb}{\mathbf b}
\newcommand{\bfc}{\mathbf c}




%Here are some user-defined operators
\newcommand{\re}{\operatorname {Re}}
\newcommand{\im}{\operatorname {Im}}


%These commands deal with theorem-like environments (i.e., italic)
\theoremstyle{plain}
\newtheorem{theorem}{Theorem}[section]
\newtheorem{corollary}[theorem]{Corollary}
\newtheorem{lemma}[theorem]{Lemma}
\newtheorem{conjecture}[theorem]{Conjecture}

%These deal with definition-like environments (i.e., non-italic)
\theoremstyle{definition}
\newtheorem{definition}[theorem]{Definition}
\newtheorem{example}[theorem]{Example}
\newtheorem{remark}[theorem]{Remark}

%your name and date in the header.
\usepackage[us]{datetime} 
\usepackage{fancyhdr}
\pagestyle{fancy}
\lhead{}
\chead{MATH 3150\\ Homework 2}
\rhead{ Your name \\ \today}
\lfoot{}
\cfoot{}
\rfoot{\thepage}
\renewcommand{\headrulewidth}{0 pt}
\renewcommand{\footrulewidth}{0 pt}
\begin{document}
\begin{enumerate}[1.]
\item Suppose that $(X, d_X)$ and $(Y,d_Y)$ are metric spaces. Define $d:(X\times Y)\times (X\times Y)\rightarrow \mathbb{R}$ by 
\[d((x,y),(a,b))=d_X(x,a)+d_Y(y,b).\]
Prove $(X\times Y, d)$ is a metric space. 
\item Let $X$ be a set with the following metric: 
\[\rho(x,x)=0\]
\[\rho(x,y)=1, \quad x\neq y\]
Show that in $(X,\rho)$ every subset is open. 
\item Let $a_1=\sqrt{2}$,  and $a_{n+1}=\sqrt{2a_n}$ for $n\geq 1$. Show that this sequence converges. \\
\ \\
{\bf Hint: } Show that this sequence is bounded above by 2 and increasing via induction. 
\item Find the limits and show by arguing directly from the definitions that the following sequences converge.
\begin{enumerate}[a)]
\item $a_n=\dfrac{2n-3}{n+5}$, $n\geq 0$. 
\item $b_n=\dfrac{n+5}{n^2-n-1}$, $n\geq 2$. 
\end{enumerate}
\item Suppose $(a_n)$, $(b_n)$ and $(c_n)$ are sequences of real numbers. Show if $a_n\leq b_n\leq c_n$ for all $n$ and both $(a_n)$ and $(c_n)$ converge to $L$ then $(b_n)$ converges to $L$.
\item Prove that a set is closed if and only if $S$ contains all its limit points. As a reminder:
 \begin{definition} Let $S$ be a subset of a metric space $X$. A point $y\in X$ is a limit point of $S$ if and only if for every $\varepsilon>0$ there exists a point $s\in S$ such that $s\neq y$ and $d(s,y)<\varepsilon$ (i.e. $N_\varepsilon(y)\cap (S\setminus \{y\})\neq \emptyset$).
\end{definition}
\end{enumerate}
\end{document}









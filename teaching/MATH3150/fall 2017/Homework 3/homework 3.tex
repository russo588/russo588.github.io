\documentclass[12pt,letterpaper]{article}
\usepackage[margin=1in]{geometry}
\usepackage{amsfonts}
\usepackage{amssymb}
\usepackage{amsthm}
\usepackage{amsmath}
\usepackage{enumerate}

%Here are some user-defined notations
\newcommand{\RR}{\mathbf R}  %bold R
\newcommand{\CC}{\mathbf C}  %bold C
\newcommand{\ZZ}{\mathbf Z}   %bold Z
\newcommand{\QQ}{\mathbf Q}   %bold Q
\newcommand{\rr}{\mathbb R}     %blackboard bold R
\newcommand{\cc}{\mathbb C}    %blackboard bold R
\newcommand{\zz}{\mathbb Z}    %blackboard bold R
\newcommand{\qq}{\mathbb Q}   %blackboard bold Q
\newcommand{\calM}{\mathcal M}  %calligraphic M
\newcommand{\sm}{\setminus} 
\newcommand{\bfa}{\mathbf a}
\newcommand{\bfb}{\mathbf b}
\newcommand{\bfc}{\mathbf c}




%Here are some user-defined operators
\newcommand{\re}{\operatorname {Re}}
\newcommand{\im}{\operatorname {Im}}


%These commands deal with theorem-like environments (i.e., italic)
\theoremstyle{plain}
\newtheorem{theorem}{Theorem}[section]
\newtheorem{corollary}[theorem]{Corollary}
\newtheorem{lemma}[theorem]{Lemma}
\newtheorem{conjecture}[theorem]{Conjecture}

%These deal with definition-like environments (i.e., non-italic)
\theoremstyle{definition}
\newtheorem{definition}[theorem]{Definition}
\newtheorem{example}[theorem]{Example}
\newtheorem{remark}[theorem]{Remark}

%your name and date in the header.
\usepackage[us]{datetime} 
\usepackage{fancyhdr}
\pagestyle{fancy}
\lhead{}
\chead{MATH 3150\\ Homework 3}
\rhead{ Your name \\ \today}
\lfoot{}
\cfoot{}
\rfoot{\thepage}
\renewcommand{\headrulewidth}{0 pt}
\renewcommand{\footrulewidth}{0 pt}
\begin{document}
\begin{enumerate}[1.]
\item Assuming the elementary properties of the trigonometric functions show on the interval $(0,\pi/2)$ that the function $\tan(x)-x$ is strictly increasing and $\frac{\sin(x)}{x}$ is strictly decreasing. 
\item We first define limits at infinity.  
\begin{definition}
Given a metric space $Y$, a point $L\in Y$ and $f:[0,\infty)\rightarrow Y$ \emph{has limit }$L\in Y$ \emph{at infinity}, written 
\[\lim_{x\rightarrow \infty}f(x)=L,\]
if for every $\varepsilon>0$ there is a $C>0$ such that if $x>C$ then $d_Y(f(x),L)<\varepsilon$. 
\end{definition}
{\bf Warning:} This is now a definition you will be expected to know\\

Show that if $f:[0,\infty)\rightarrow Y$ is continuous and has a limit at infinity then $f$ is uniformly continuous.  

\item Let $f:[0,1]\rightarrow [0,1]$ be a continuous function. Show that $f$ has a fixed point, i.e. there is a point $x\in [0,1]$ such that $f(x)=x$. 
\item Formulate and prove a squeeze theorem for functions. 
\item We start with the following definition
\begin{definition}Let $X$ and $Y$ be metric spaces.  We call a function $f:X\rightarrow Y $ \emph{Lipschitz continuous} if there exists a $K>0$ such that 
\[d_{Y}(f(p),f(q))\leq K d_X(p,q)\]
 for all $p,q\in X$. 
\end{definition}
 Let $U$ be an open interval of $\mathbb{R}$. Prove that if $f$ is differentiable and $f\prime :U\rightarrow \mathbb{R}$ is bounded, then $f$ is Lipschitz continuous. 
\end{enumerate}
\end{document}









\documentclass[12pt,letterpaper]{article}
\usepackage[margin=1in]{geometry}
\usepackage{amsfonts}
\usepackage{amssymb}
\usepackage{amsthm}
\usepackage{amsmath}
\usepackage{enumerate}

%Here are some user-defined notations
\newcommand{\RR}{\mathbf R}  %bold R
\newcommand{\CC}{\mathbf C}  %bold C
\newcommand{\ZZ}{\mathbf Z}   %bold Z
\newcommand{\QQ}{\mathbf Q}   %bold Q
\newcommand{\rr}{\mathbb R}     %blackboard bold R
\newcommand{\cc}{\mathbb C}    %blackboard bold R
\newcommand{\zz}{\mathbb Z}    %blackboard bold R
\newcommand{\qq}{\mathbb Q}   %blackboard bold Q
\newcommand{\calM}{\mathcal M}  %calligraphic M
\newcommand{\sm}{\setminus} 
\newcommand{\bfa}{\mathbf a}
\newcommand{\bfb}{\mathbf b}
\newcommand{\bfc}{\mathbf c}




%Here are some user-defined operators
\newcommand{\re}{\operatorname {Re}}
\newcommand{\im}{\operatorname {Im}}


%These commands deal with theorem-like environments (i.e., italic)
\theoremstyle{plain}
\newtheorem{theorem}{Theorem}[section]
\newtheorem{corollary}[theorem]{Corollary}
\newtheorem{lemma}[theorem]{Lemma}
\newtheorem{conjecture}[theorem]{Conjecture}

%These deal with definition-like environments (i.e., non-italic)
\theoremstyle{definition}
\newtheorem{definition}[theorem]{Definition}
\newtheorem{example}[theorem]{Example}
\newtheorem{remark}[theorem]{Remark}

%your name and date in the header.
\usepackage[us]{datetime} 
\usepackage{fancyhdr}
\pagestyle{fancy}
\lhead{}
\chead{MATH 3150\\ Homework 1}
\rhead{ Your name \\ \today}
\lfoot{}
\cfoot{}
\rfoot{\thepage}
\renewcommand{\headrulewidth}{0 pt}
\renewcommand{\footrulewidth}{0 pt}
\begin{document}
\begin{enumerate}[1.]
 \item Let $\mathbb{C}$ denote the complex numbers with the standard addition and multiplication. Show that there is no relation $>$ such that $\mathbb{C}$ is an ordered field. 
 \item Let $A$ and $B$ be two subsets of $\mathbb{R}$ which are bounded below. 
 Show \[\inf(A\cup B)=\text{min}\{\inf(A),\inf(A)\}.\]
%The above \text{g.l.b} is a not proper Latex, it is much better to define an operator in the preamble 
 \item Find the supremum and infimum of the following set: $S=\{1,\frac{1}{2}, \frac{1}{3},\frac{1}{4},\ldots\}$. Prove your claim. 
 \item Let $a\in \mathbb{R}$ such that $a>1$. Show that $A=\{a, a^2, a^3,\ldots \}$ is not bounded above.\\
 \ \\
 {\bf Hint:} Do not forget about induction!
 \begin{enumerate}[(i)]
 \item Show that there exists an integer $n_0$ such that $a>1+\frac{1}{n_0}$. 
 \item Show that $\left(1+\frac{1}{n}\right)^n\geq 2$ for all natural numbers $n$.
 \item Show $2^n>n$ for all natural numbers $n$. 
 \item Since $a>1+\frac{1}{n_0}$ then $a^{n_0}>\left(1+\frac{1}{n_0}\right)^{n_0}>2$. Use (i) and (ii) to reduce the problem to showing $\{2, 2^2, 2^3,\ldots \}$ is unbounded and then show this is true by item (iii)
 \end{enumerate}
 \end{enumerate}


\end{document}









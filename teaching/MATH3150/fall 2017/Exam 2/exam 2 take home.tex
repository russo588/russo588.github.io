\documentclass[12pt,letterpaper]{article}
\usepackage[margin=1in]{geometry}
\usepackage{amsfonts}
\usepackage{amssymb}
\usepackage{amsthm}
\usepackage{amsmath}
\usepackage{enumerate}

%Here are some user-defined notations
\newcommand{\RR}{\mathbf R}  %bold R
\newcommand{\CC}{\mathbf C}  %bold C
\newcommand{\ZZ}{\mathbf Z}   %bold Z
\newcommand{\QQ}{\mathbf Q}   %bold Q
\newcommand{\rr}{\mathbb R}     %blackboard bold R
\newcommand{\cc}{\mathbb C}    %blackboard bold R
\newcommand{\zz}{\mathbb Z}    %blackboard bold R
\newcommand{\qq}{\mathbb Q}   %blackboard bold Q
\newcommand{\calM}{\mathcal M}  %calligraphic M
\newcommand{\sm}{\setminus} 
\newcommand{\bfa}{\mathbf a}
\newcommand{\bfb}{\mathbf b}
\newcommand{\bfc}{\mathbf c}




%Here are some user-defined operators
\newcommand{\re}{\operatorname {Re}}
\newcommand{\im}{\operatorname {Im}}


%These commands deal with theorem-like environments (i.e., italic)
\theoremstyle{plain}
\newtheorem{theorem}{Theorem}[section]
\newtheorem{corollary}[theorem]{Corollary}
\newtheorem{lemma}[theorem]{Lemma}
\newtheorem{conjecture}[theorem]{Conjecture}

%These deal with definition-like environments (i.e., non-italic)
\theoremstyle{definition}
\newtheorem{definition}[theorem]{Definition}
\newtheorem{example}[theorem]{Example}
\newtheorem{remark}[theorem]{Remark}

%your name and date in the header.
\usepackage[us]{datetime} 
\usepackage{fancyhdr}
\pagestyle{fancy}
\lhead{}
\chead{MATH 3150\\ Exam 2}
\rhead{ Your name \\ \today}
\lfoot{}
\cfoot{}
\rfoot{\thepage}
\renewcommand{\headrulewidth}{0 pt}
\renewcommand{\footrulewidth}{0 pt}
\begin{document}
\begin{enumerate}[1.]
\item We define the following:
\begin{definition}
Let $I=(c,d)$ be an interval and suppose $a\in I$. Let $E$ denote either $I$ or $I\setminus\{a\}$ and suppose $f:E\rightarrow \mathbb{R}$. We say $f$ has a limit as $x$ approaches $a$ from the right if the function $f|_{(a,d)}:(a,d)\rightarrow \mathbb{R}$ has a limit at $a$. The limit if it exists is denoted 
\[\lim_{x\rightarrow a^+}f(x)=\lim_{a<x\rightarrow a}f(x).\] The limit from the left is defined similarly. 
\end{definition}
Show $f$ has a limit at $a$ if and only if both the limit from the right and the limit from the left exist and are equal. 
\item We will start with a definition. 
\begin{definition}
 Let $(f_n)$ be a sequence of functions in $C(X)$ where $X$ is a compact set. We say the sequence $(f_n)$ is \emph{equicontinuous} if given an $\varepsilon>0$ there is a $\delta>0$ such that, for every $n$, if $d(x,y)<\delta$ we have that $d(f_n(x), f_n(y))<\varepsilon$. 
 \end{definition}
Show that if $(f_n)$ converges to $f$ (in the sense of the uniform metric) then $(f_n)$ is equicontinuous. 
\item A simple pendulum consists of a mass $m$ hanging from a string of length $L$. When displaced to an initial angle and released the pendulum will have the following equation of motion:
\[\dfrac{d^2\theta}{dt^2}+\dfrac{g}{L}\sin \theta=0,\]
where $\theta$ is the displacement angle and $g$ is the acceleration due to gravity. To solve this equation a physicist will employ the small angle approximation
\[\sin \theta \approx \theta \] for small $\theta$. The equation of motion then becomes simplified:
\[\dfrac{d^2\theta}{dt^2}+\dfrac{g}{L}\theta=0.\]
Show this step is justified by showing that 
\[|\sin \theta -\theta|\leq \frac{1}{6}|\theta|^{3}\]
via Taylor's theorem. 
\item Show that if $f:\mathbb{R}\rightarrow \mathbb{R}$ is differentiable then between any two zeros of $f$ there is a zero of $f^\prime$. Use this result to show via induction that any polynomial of degree $n$ can have \emph{at most} $n$ distinct real roots. (You may not use the Fundamental Theorem of Algebra to do this problem.)
\item Let $f(x)=\log(x)$. Given $a>0$, let $g(x)=f(ax)$. Prove $g^\prime(x)=f^\prime(x)$ and thus there exists a $c$ such that $g(x)=f(x)+c$. Prove, $c=\log(a)$ and thus $\log(ax)=\log(x)+\log(a)$. (Use the integral definition of logarithms.)
\end{enumerate}
\end{document}









\documentclass[12pt,letterpaper]{article}
\usepackage[margin=1in]{geometry}
\usepackage{amsfonts}
\usepackage{amssymb}
\usepackage{amsthm}
\usepackage{amsmath}
\usepackage{enumerate}

%Here are some user-defined notations
\newcommand{\RR}{\mathbf R}  %bold R
\newcommand{\CC}{\mathbf C}  %bold C
\newcommand{\ZZ}{\mathbf Z}   %bold Z
\newcommand{\QQ}{\mathbf Q}   %bold Q
\newcommand{\rr}{\mathbb R}     %blackboard bold R
\newcommand{\cc}{\mathbb C}    %blackboard bold R
\newcommand{\zz}{\mathbb Z}    %blackboard bold R
\newcommand{\qq}{\mathbb Q}   %blackboard bold Q
\newcommand{\calM}{\mathcal M}  %calligraphic M
\newcommand{\sm}{\setminus} 
\newcommand{\bfa}{\mathbf a}
\newcommand{\bfb}{\mathbf b}
\newcommand{\bfc}{\mathbf c}




%Here are some user-defined operators
\newcommand{\re}{\operatorname {Re}}
\newcommand{\im}{\operatorname {Im}}


%These commands deal with theorem-like environments (i.e., italic)
\theoremstyle{plain}
\newtheorem{theorem}{Theorem}[section]
\newtheorem{corollary}[theorem]{Corollary}
\newtheorem{lemma}[theorem]{Lemma}
\newtheorem{conjecture}[theorem]{Conjecture}

%These deal with definition-like environments (i.e., non-italic)
\theoremstyle{definition}
\newtheorem{definition}[theorem]{Definition}
\newtheorem{example}[theorem]{Example}
\newtheorem{remark}[theorem]{Remark}

%your name and date in the header.
\usepackage[us]{datetime} 
\usepackage{fancyhdr}
\pagestyle{fancy}
\lhead{}
\chead{MATH 3150\\ Homework 1}
\rhead{ Your name \\ \today}
\lfoot{}
\cfoot{}
\rfoot{\thepage}
\renewcommand{\headrulewidth}{0 pt}
\renewcommand{\footrulewidth}{0 pt}
\begin{document}
\begin{enumerate}[1.]
 \item Let $\mathbb{C}$ denote the complex numbers with the standard addition and multiplication. Show that there is no order relation $>$ such that $\mathbb{C}$ is an ordered field. As a reminder:
 \begin{definition}\rm
   An {\it ordered field} $\mathbb{F}=(\mathbb{F},+,\cdot,<)$ consists
   of a field $(\mathbb{F},+,\cdot)$ together with a relation $<$ on $\mathbb{F}$, called
  an \emph{order}, satisfying
  \begin{enumerate}[(i)]
  \item (\emph{trichotomy}) for each $x,y\in S$, exactly one of the following hold,
 $$
   x<y, \ \ \ y<x, \ \ \ x=y;
 $$
  \item (\emph{transitivity}) for $x,y,z\in S$, 
   if $x<y$ and $y<z$, then $x<z$.
   \item if $x,y,z\in \mathbb{F}$ and $x<y$, then $x+z<y+z$;
   \item if $x,y\in \mathbb{F}$ and $x,y>0$, then $xy>0$.
  \end{enumerate} 
\end{definition} 
 \item Find the supremum and infimum of the following set: $S=\{1,\frac{1}{2}, \frac{1}{3},\frac{1}{4},\ldots\}$. Prove your claim. 
\item Show if $A\subset B$ are subsets of $\mathbb{R}$ where $B$ is bounded above, then $A$ and $B$ have least upper bounds and 
\[\sup(A)\leq \sup(B).\]
Find an example where $\sup(A)=\sup(B)$.  
 \item Let $A$ and $B$ be two non-empty subsets of $\mathbb{R}$ which are bounded below. 
 Show \[\inf(A\cup B)=\min\{\inf(A),\inf(B)\}.\]
\item Suppose that $A$ and $B$ are non-empty subsets of $\mathbb{R}$ that are bounded above. Let 
\[A+B=\{a+b\ | a\in A, b\in B\}.\]
Show $A+B$ has a supremum and that $\sup(A+B)=\sup(A)+\sup(B)$. 


 \end{enumerate}


\end{document}









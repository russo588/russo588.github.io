\documentclass[12pt,letterpaper]{article}
\usepackage[margin=1in]{geometry}
\usepackage{amsfonts}
\usepackage{amssymb}
\usepackage{amsthm}
\usepackage{amsmath}
\usepackage{enumerate}

%Here are some user-defined notations
\newcommand{\RR}{\mathbf R}  %bold R
\newcommand{\CC}{\mathbf C}  %bold C
\newcommand{\ZZ}{\mathbf Z}   %bold Z
\newcommand{\QQ}{\mathbf Q}   %bold Q
\newcommand{\rr}{\mathbb R}     %blackboard bold R
\newcommand{\cc}{\mathbb C}    %blackboard bold R
\newcommand{\zz}{\mathbb Z}    %blackboard bold R
\newcommand{\qq}{\mathbb Q}   %blackboard bold Q
\newcommand{\calM}{\mathcal M}  %calligraphic M
\newcommand{\sm}{\setminus} 
\newcommand{\bfa}{\mathbf a}
\newcommand{\bfb}{\mathbf b}
\newcommand{\bfc}{\mathbf c}




%Here are some user-defined operators
\newcommand{\re}{\operatorname {Re}}
\newcommand{\im}{\operatorname {Im}}


%These commands deal with theorem-like environments (i.e., italic)
\theoremstyle{plain}
\newtheorem{theorem}{Theorem}[section]
\newtheorem{corollary}[theorem]{Corollary}
\newtheorem{lemma}[theorem]{Lemma}
\newtheorem{conjecture}[theorem]{Conjecture}

%These deal with definition-like environments (i.e., non-italic)
\theoremstyle{definition}
\newtheorem{definition}[theorem]{Definition}
\newtheorem{example}[theorem]{Example}
\newtheorem{remark}[theorem]{Remark}

%your name and date in the header.
\usepackage[us]{datetime} 
\usepackage{fancyhdr}
\pagestyle{fancy}
\lhead{}
\chead{MATH 3150\\ Exam 1}
\rhead{ Your name \\ \today}
\lfoot{}
\cfoot{}
\rfoot{\thepage}
\renewcommand{\headrulewidth}{0 pt}
\renewcommand{\footrulewidth}{0 pt}
\begin{document}
\begin{enumerate}[1.]
 \item Let $p_n=a_m n^m + a_{m-1}n^{m-1}+\ldots +a_1 n +a_0$ and $q_n=b_m n^m + b_{m-1}n^{m-1}+\ldots +b_1 n +b_0$ for $n>0$. Prove that 
 \[\lim_{n\rightarrow \infty} \frac{p_n}{q_n}= \frac{a_m}{b_m}.\]
 \item Suppose that $y$ is a limit point of a metric space $X$. Show that $Y=X\setminus\{y\}$ is not complete. 
 \item Let $(X,d_X)$ and $(Y,d_Y)$ be complete metric spaces. Let $(X\times Y, d)$ be the metric space defined by the metric 
 \[d:(X\times Y)\times (X\times Y)\rightarrow \mathbb{R};\quad \quad d((x,y),(a,b))=d_X(x,a)+d_Y(y,b)\]
 is a complete metric space. 
 \item Let $S$ be a bounded subset of $\mathbb{R}$. Show that \[\inf(S)=-\sup(-S)\] where $-S=\{-s\ :\ s\in S\}$. 
 \item A  metric space $(X,d)$ is called sequentially compact if every sequence has a convergent subsequence. Show that $X$ is sequentially compact if and only if every infinite subset has a limit point in $X$. 
\end{enumerate}
\end{document}









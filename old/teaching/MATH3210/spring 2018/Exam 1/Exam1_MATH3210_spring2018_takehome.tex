\documentclass[12pt,letterpaper]{article}
\usepackage[margin=1in]{geometry}
\usepackage{amsfonts}
\usepackage{amssymb}
\usepackage{amsthm}
\usepackage{amsmath}
\usepackage{enumerate}

%Here are some user-defined notations
\newcommand{\RR}{\mathbf R}  %bold R
\newcommand{\CC}{\mathbf C}  %bold C
\newcommand{\ZZ}{\mathbf Z}   %bold Z
\newcommand{\QQ}{\mathbf Q}   %bold Q
\newcommand{\rr}{\mathbb R}     %blackboard bold R
\newcommand{\cc}{\mathbb C}    %blackboard bold R
\newcommand{\zz}{\mathbb Z}    %blackboard bold R
\newcommand{\qq}{\mathbb Q}   %blackboard bold Q
\newcommand{\calM}{\mathcal M}  %calligraphic M
\newcommand{\sm}{\setminus} 
\newcommand{\bfa}{\mathbf a}
\newcommand{\bfb}{\mathbf b}
\newcommand{\bfc}{\mathbf c}





%Here are some user-defined operators
\newcommand{\re}{\operatorname {Re}}
\newcommand{\im}{\operatorname {Im}}


%These commands deal with theorem-like environments (i.e., italic)
\theoremstyle{plain}
\newtheorem{theorem}{Theorem}[section]
\newtheorem{corollary}[theorem]{Corollary}
\newtheorem{lemma}[theorem]{Lemma}
\newtheorem{conjecture}[theorem]{Conjecture}

%These deal with definition-like environments (i.e., non-italic)
\theoremstyle{definition}
\newtheorem{definition}[theorem]{Definition}
\newtheorem{example}[theorem]{Example}
\newtheorem{remark}[theorem]{Remark}

%your name and date in the header.
\usepackage[us]{datetime} 
\usepackage{fancyhdr}
\pagestyle{fancy}
\lhead{}
\chead{MATH 3210\\ Exam 1}
\rhead{}
\lfoot{}
\cfoot{}
\rfoot{\thepage}
\renewcommand{\headrulewidth}{0 pt}
\renewcommand{\footrulewidth}{0 pt}
\begin{document}
\begin{enumerate}[{\bf1.}]
\item Let $\mathcal{F}:\mathbb{C}^N\rightarrow \mathbb{C}^N$ be the operator defined by 
\[\begin{bmatrix}x_0\\ \vdots\\ x_{N-1}\end{bmatrix} \mapsto \begin{bmatrix}X_0\\ \vdots\\ X_{N-1}\end{bmatrix}\]
where 
\[X_k=\sum_{n=0}^{N-1}x_ne^{-i2\pi kn/ N}.\]
This is the \emph{discrete Fourier transform}. Define the map $\mathcal{D}:\mathbb{C}^N\rightarrow \mathbb{C}^N$ given by
\[\begin{bmatrix}X_0\\ \vdots\\ X_{N-1}\end{bmatrix}\mapsto \begin{bmatrix}x_0\\ \vdots\\ x_{N-1}\end{bmatrix}\]
where 
\[x_n=\frac{1}{N}\sum_{k=0}^{N-1}X_ke^{i2\pi kn/ N}.\]
Construct the matrices with respect to the standard basis (for both the domain and codomain) for both $\mathcal{D}$ and $\mathcal{F}$ on $\mathbb{C}^4$ and use these matrices to show $\mathcal{F}$ is invertible and its inverse is $\mathcal{D}$. \\

{\bf \noindent Hint: }$e^{i\theta}=\cos(\theta)+i\sin(\theta)$.\\

\item Let $\mathcal{A}(\mathbb{R})$ be the space of ``formal" power-series over the reals i.e. 
\[\mathcal{A}(\mathbb{R})=\left\{f(x)=\sum_{n=0}^\infty a_nx^n\ \middle| a_i\in\mathbb{R}\right\} \]
with the usual operations of addition and scalar multiplication on powerseries. 
Let $\frac{d}{dx}~:~\mathcal{A}(\mathbb{R})\rightarrow \mathcal{A}(\mathbb{R})$ be the linear map of  ``differentiation", i.e. 
\[\frac{d}{dx}(f(x))=\sum_{n=1}^\infty na_nx^{n-1}.\]
Let $\mathcal{A}_{n.c}(\mathbb{R})$ be the space of formal power series without a constant term, i.e. 
\[\mathcal{A}_{n.c}(\mathbb{R})=\left\{f(x)=\sum_{n=1}^\infty a_nx^n\ \middle| a_i\in\mathbb{R}\right\} \]
Construct an explicit isomorphism $T:\mathcal{A}_{n.c}(\mathbb{R})\rightarrow\mathcal{A}(\mathbb{R})/\text{null}(\frac{d}{dx})$.\\

\item Determine the dimension of $U=\left\{[a_1,\ldots,a_n]^\top \middle| \sum_{i=1}^n a_i=0\right\}$ as a subspace of $\mathbb{R}^n$. \\

{\bf \noindent Hint: }Consider the linear map $S:\mathbb{R}^n\rightarrow \mathbb{R}$ given by $S([a_1,\ldots, a_n]^\top)=\sum_{i=1}^na_i$. \\

\item Let $P_{n}(x)=\{p(x)=a_nx^n+\ldots+a_1x+a_0\ |\ a_i\in \mathbb{R},\ p:[0,1]\rightarrow \mathbb{R}\}$ be the space of polynomials of degree $\leq n$. Let $P_{\text{per}}(x)$ be the subspace of polynomials in $P_{n}(x)$ with periodic boundary conditions, i.e. 
\[P_{\text{per}}(x)=\{p\in P_{n}(x)\ \mid\  p(0)=p(1)\}.\]
Determine the dimension of $P_{\text{per}}(x)$ as a subspace of $P_n(x)$. \\

{\bf \noindent Hint: }Try to construct a basis for $P_{\text{per}}(x)$ as a subspace of $P_{2}(x)$ and then generalize the argument for an arbitrary $n$. Alternatively, reduce to the above problem. 
\end{enumerate}
\end{document}









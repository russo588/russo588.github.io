\documentclass[12pt,letterpaper]{article}
\usepackage[margin=1in]{geometry}
\usepackage{amsfonts}
\usepackage{amssymb}
\usepackage{amsthm}
\usepackage{amsmath}
\usepackage{enumerate}

%Here are some user-defined notations
\newcommand{\RR}{\mathbf R}  %bold R
\newcommand{\CC}{\mathbf C}  %bold C
\newcommand{\ZZ}{\mathbf Z}   %bold Z
\newcommand{\QQ}{\mathbf Q}   %bold Q
\newcommand{\rr}{\mathbb R}     %blackboard bold R
\newcommand{\cc}{\mathbb C}    %blackboard bold R
\newcommand{\zz}{\mathbb Z}    %blackboard bold R
\newcommand{\qq}{\mathbb Q}   %blackboard bold Q
\newcommand{\calM}{\mathcal M}  %calligraphic M
\newcommand{\sm}{\setminus} 
\newcommand{\bfa}{\mathbf a}
\newcommand{\bfb}{\mathbf b}
\newcommand{\bfc}{\mathbf c}




%Here are some user-defined operators
\newcommand{\re}{\operatorname {Re}}
\newcommand{\im}{\operatorname {Im}}


%These commands deal with theorem-like environments (i.e., italic)
\theoremstyle{plain}
\newtheorem{theorem}{Theorem}[section]
\newtheorem{corollary}[theorem]{Corollary}
\newtheorem{lemma}[theorem]{Lemma}
\newtheorem{conjecture}[theorem]{Conjecture}

%These deal with definition-like environments (i.e., non-italic)
\theoremstyle{definition}
\newtheorem{definition}[theorem]{Definition}
\newtheorem{example}[theorem]{Example}
\newtheorem{remark}[theorem]{Remark}

%your name and date in the header.
\usepackage[us]{datetime} 
\usepackage{fancyhdr}
\pagestyle{fancy}
\lhead{}
\chead{MATH 3210\\ Homework 3}
\rhead{ Your name \\ date}
\lfoot{}
\cfoot{}
\rfoot{\thepage}
\renewcommand{\headrulewidth}{0 pt}
\renewcommand{\footrulewidth}{0 pt}
\begin{document}
\begin{enumerate}[1.]
\item (\S 3.A \#7) Show that every linear map from a 1-dimensional vector space to itself is multiplication by some scalar. More precisely, prove that if $\text{dim}(V)=1$ and $T\in \mathcal{L}(V)$, then there exists a $\lambda \in \mathbb{F}$ such that $Tv=\lambda v$ for all $v\in V$.
\begin{proof}
Let $V$ be a one dimensional vector space, $\beta=\{v\}$ be its basis, and $T\in \mathcal{L}(V)$. If $u\in V$ then $u=\alpha v$ for some $\alpha\in \mathbb{F}$. Since $T:V\rightarrow V$, if $u\in V$ then 
\[Tv=\lambda v \quad\text{ for some }\quad\lambda\in \mathbb{F}\]
and
\[Tu=T(\alpha v)=\alpha T(v)=\alpha (\lambda v).\]
\end{proof}
\ \\
\item (\S 3.B \# 9) Suppose that $T\in \mathcal{L}(V,W)$ is injective and $v_1, \ldots ,v_n$ is linearly independent in $V$. Prove that $Tv_1,\ldots, Tv_n$ is linearly independent in $W$. 
\begin{proof}Suppose for the sake of contradiction that $Tv_1, \ldots , Tv_n$ is linearly dependent. Hence there exists $a_1, \ldots, a_n$ not all zero such that 
\[0=a_1Tv_1+\ldots + a_nTv_n=T(a_1v_1+\ldots +a_nv_n).\]
Since $T$ is injective, we have that $\text{null}(T)=\{0\}$ and 
\[a_1v_1+\ldots +a_nv_n=0\] and we have a contradiction.
\end{proof}
\ \\
\item (\S 3.B \# 10) Suppose that $v_1, \ldots, v_n$ spans $V$ and $T\in \mathcal{L}(V,W)$. Prove that $Tv_1, \ldots ,Tv_n$ spans $\text{ran}(T)$. 
\begin{proof}Since $v_1,\ldots, v_n$ spans $V$, if $v\in V$ then 
\[v=b_1v_1+\ldots +b_nv_n\quad \text{ for some }\quad b_1, \ldots, b_n\in \mathbb{F}.\]
Hence, if $w\in \text{ran}(T)$ then $w=T\hat{v}$ for some $\hat{v}\in V$ and 
\[w=T\hat{v}=a_1Tv_1+\ldots +a_nTv_n\] for some $a_1, \ldots, a_n\in \mathbb{F}$. Thus $\{Tv_1, \ldots, Tv_n\}$ spans $\text{ran}(T)$.
\end{proof}
\ \\
\item (\S 3.B \#12) Suppose that $V$ is finite dimensional and that $T\in \mathcal{L}(V,W)$. Prove that there exists a subspace $U$ of $V$ such that $U\cap \text{null}(T)=\{0\}$ and $\text{ran}(T)=\{Tu : u\in U\}$.
\begin{proof} Since $\text{null}(T)$ is a subspace of $V$ then it has some basis $\beta=\{u_1, \ldots, u_p\}$. Expand the basis $\beta$ to a basis $\gamma=\{v_1, \ldots, v_p,u_1,\ldots u_n\}$ of $V$. Let $U=\text{span}\{u_1,\ldots, u_n\}$. Since $\gamma$ is linearly independent we have that $U\cap \text{null}(T)=\{0\}$. Suppose that $v\in V$, since $\gamma$ spans $V$ we have that 
\[v=a_1v_1+\ldots+a_pv_p+b_1u_1+\ldots+b_nu_n.\]
Thus 
\[Tv=a_1Tv_1+\ldots+a_pTv_p+b_1Tu_1+\ldots +b_n Tu_n=0+b_1Tu_1+\ldots +b_n Tu_n=T(b_1u_1+\ldots b_nu_n).\]
Clearly, $\text{ran}(T)=\{Tu\ :\ u\in U\}$.
\end{proof}
\ \\
\item (\S 3.D \# 8) Suppose $V$ is finite dimensional and $T:V\rightarrow W$ is a surjective linear map of $V$ onto $W$. Prove that there is a subspace $U$ of $V$ such that $T\mid_U$ is an isomorphism of $U$ onto $W$. (Here $T\mid_U$ means the function $T$ restricted to $U$. In other words, $T\mid_U$ is the function whose domain is $U$, with $T\mid_U$ defined by $T\mid_U(u)=T(u)$ for every $u\in U$.)
\begin{proof} By the above problem there exists a subspace $U$ such that $U\cap \text{null}(T)=\{0\}$ and that $\text{ran}(T)=\{Tu\ : u\in U\}$. By surjectivity of $T$ we have that $\text{ran}(T)=W$ and thus $W=\{Tu\ :\ u\in U\}$. Clearly $T\mid_U:U\rightarrow W$ is surjective. We must check injectivity of $T\mid_U$. Suppose that $T\mid_U$ is not injective, then there exists a $u\in U$ such that $T\mid_U(u)=0$. But 
\[T\mid_U(u)=T(u)=0\]
and therefore $u\in \text{null}(T)$. This is a contradiction on the definition of $U$.  Hence $T\mid_U$ is injective and surjective from $U$ onto $W$ and thus is a linear isomorphism from $U$ onto $W$. 
\end{proof}
\ \\
\item (\S 3.D \# 18) Show that $V$ and $\mathcal{L}(\mathbb{F},V)$ are isomorphic vector spaces. 
\begin{proof} We note that if $V$ was finite dimensional we would have that 
\[\text{dim}(V)=\text{dim}(V)\cdot 1 =\text{dim}(V)\cdot \text{dim}(\mathbb{F})=\text{dim}(\mathcal{L}(\mathbb{F},V)).\]
The result is then true since two finite dimensional vector spaces are isomorphic if and only if they are the same dimension.
However, we will do a proof without using an argument which relies on finite dimensionality. Since $\{1\}$ is a basis for $\mathbb{F}$ over itself then $T$ is completely determined by what happens to $1$. That is, 
\[T(\lambda \cdot 1)=\lambda T(1)=\lambda v.\] For every $v\in V$ let $T_v\in \mathcal{L}(\mathbb{F},V)$ be the linear transformation such that $T_v(1)=v$. We define the following linear isomorphism 
\[\Phi: V\rightarrow \mathcal{L}(\mathbb{F},V),\]
\[\Phi(v)=T_v\in V.\]
Note that $\Phi$ is linear since 
\[\Phi(\lambda v+u)=T_{\lambda v+u}=\lambda T_v+T_u=\lambda \Phi(v)+\Phi(u).\]
Suppose that $v_1\neq v_2$, then $T_{v_1}\neq T_{v_2}$ since 
\[T_{v_1}(1)=v_1\neq v_2=T_{v_2}(1).\]
Thus $\Phi$ is injective. Let $T\in \mathcal{L}(\mathbb{F},V)$. Let $\hat{v}=T(1)$. Then 
\[\Phi(\hat{v})=T\] by definition of $\Phi$. Thus, $\Phi$ is surjective. Therefore, $\Phi$ is a linear isomorphism. 
\end{proof}
\end{enumerate}


\end{document}









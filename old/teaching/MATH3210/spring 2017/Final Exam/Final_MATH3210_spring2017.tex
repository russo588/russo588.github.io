\documentclass[11pt]{exam}
\RequirePackage{amssymb, amsfonts, amsmath, latexsym, verbatim, xspace, setspace}
\RequirePackage{tikz, pgflibraryplotmarks}
\usepackage[margin=1in]{geometry}
\usepackage{amsmath, amsthm, amssymb}
\newtheorem*{thm}{{\bf Theorem}}
\newtheorem{lemma}{{\bf Lemma}}
\newcommand{\A}{\mathfrak{A}}
\theoremstyle{definition}
\newtheorem{define}{Definition}
\newtheorem{claim}{Claim}
\newtheorem*{method}{Method}
\newtheorem{ex}{Example}
\newcommand{\dydx}{\dfrac{dy}{dx}}
\newcommand{\dydt}{\dfrac{dy}{dt}}
\newcommand{\dxdt}{\dfrac{dx}{dt}}
\newcommand{\dxdy}{\dfrac{dx}{dy}}
\newcommand{\pp}{\prime\prime}
\newcommand{\p}{\prime}
\renewcommand{\d}[2]{\dfrac{d#1}{d#2}}
\newcommand{\dd}[2]{\dfrac{d^2#1}{d#2^2}}
\newcommand{\ypp}{y^{\prime\prime}}
\newcommand{\yp}{y^{\prime}}
\newcommand{\tr}{\text{tr}}
\renewcommand\thesection{2.5}
\usepackage{xcolor}
\usepackage{graphicx}
\usepackage{lipsum}% Used for dummy text.
\usepackage{enumerate}
\DeclareMathOperator{\dimension}{dim}
\DeclareMathOperator{\ran}{ran}
\DeclareMathOperator{\nullspc}{null}

% Here's where you edit the Class, Exam, Date, etc.
\newcommand{\class}{MATH3210}
\newcommand{\term}{Spring 2017}
\newcommand{\examnum}{Final Exam}
\newcommand{\examdate}{Due: 5/1/17}


% For an exam, single spacing is most appropriate
\singlespacing
% \onehalfspacing
% \doublespacing

% For an exam, we generally want to turn off paragraph indentation
\parindent 0ex

\begin{document} 

% These commands set up the running header on the top of the exam pages
\pagestyle{head}
\firstpageheader{}{}{}
\runningheader{\class}{\examnum\ - Page \thepage\ of \numpages}{\examdate}
\runningheadrule
 {\bf \class} \\
\ \\
 {\bf \examnum} \\\

\rule[1ex]{\textwidth}{.1pt}
\ \\
The following rules apply:\\

\begin{itemize}
\item \textbf{Exam must be typed}. Please organize your proofs in a reasonably neat and coherent way. Write in complete sentences.  

\item \textbf{Mysterious or unsupported claims will not receive full
credit}.  Unreasonably large gaps in logic or an argument will receive little credit. You may quote theorems from class or the book.

\item \textbf{Your solutions must be your own.} You may use outside sources but your submitted solution must be in your own words. 
\end{itemize}

\newpage % End of cover page


\begin{questions}
\question Let $\mathcal{A}(\mathbb{R})$ be the set of real analytic functions over $\mathbb{R}$, this is a real vector space with the usual operations of function addition and scalar multiplication.  
\[\mathcal{A}(\mathbb{R})=\left\{f:\mathbb{R}\rightarrow \mathbb{R} \ \left|\  f(x)=\sum_{n=0}^\infty a_nx^n\right.\right\}\]
Consider the following operator on $\mathcal{A}(\mathbb{R})$; 
\[B:\mathcal{A}(\mathbb{R})\rightarrow \mathcal{A}(\mathbb{R})\]
\[Bf(x)=xf(x).\]
Show that this operator has no eigenvalues. Note that two analytic functions $f(x)=\sum_{n=0}^\infty a_nx^n$ and $g(x)=\sum_{n=0}^\infty b_nx^n$ are equal if and only if $a_n=b_n$ for all $n$. 
\vfill
\question Let $V$ be a finite dimensional complex vector space.  A linear map $\Gamma:\mathcal{L}(V)\rightarrow \mathcal{L}(V)$ is called positive if it takes positive operators to positive operators i.e. if $A$ is positive then $\Gamma(A)$ is positive. Let, $\Psi:\mathcal{L}(V)\rightarrow \mathcal{L}(V)$ denote the following map, 
\[\Psi(A)=\sum_{i=0}^n B_i^* A B_i\]
where each $B_i\in \mathcal{L}(V)$. Show $\Psi$ is a positive map. 
\vfill
\question Suppose $T$ is a positive operator on $V$. Prove that $T$ is invertible if and only if 
\[\langle Tv,v\rangle >0 \]
for every $v\in V$ with $v\neq 0$. 
\vfill
\item Suppose $V$ and $W$ are finite dimensional and $U$ is a subspace of $V$. Prove there exists a $T\in \mathcal{L}(V,W)$ such that $\text{null}(T)=U$ if and only if $\text{dim}(U)\geq \text{dim}(V)-\text{dim}(W)$.
\vfill
\question Suppose $V$ is a complex finite dimensional vector space and $T\in \mathcal{L}(V)$. Let $p\in \mathcal{P}(\mathbb{C})$ be a polynomial and $\alpha\in \mathbb{C}$. Prove that $\alpha$ is an eigenvalue of $p(T)$ if and only if $\alpha=p(\lambda)$ for some eigenvalue $\lambda$ of $T$.
\vfill
\question Show that every self adjoint operator on a complex vector space has a cube root, i.e. if $T$ is self adjoint then there exists an operator $S$ such that $S^3=T$. 
\vfill
\question Let $V$ and $W$ be vector spaces over some field $\mathbb{F}$ and let
\[V\times W=\{(v,w)\ |\ v\in V, w\in W\}.\]
$V\times W$ is a vector space with the following operations 
\[(v_1,w_1)+(v_2, w_2)=(v_1+v_2, w_1 +w_2)\]
and 
\[c(v,w)=(cv, cw)\]
where $c\in \mathbb{F}$, $v, v_1, v_2\in V$ and $w, w_1, w_2\in W$. 
Let $T:V\rightarrow W$ be a map. The graph of $T$ is the subset of $V\times W$ defined by 
\[\text{graph}(T)=\{(v,Tv)\in V\times W : v\in V\}.\] Show that $T$ is a linear map if and only if $\text{graph}(T)$ is a subspace of $V\times W$. 

\end{questions}

\end{document}


%%%%%%%%%%%%%%%%%%%%%%%%%%%%%%%%%%%%%%%%%%%%%%%
% Basic question
\addpoints
\question[10] Differentiate $f(x)=x^2$ with respect to $x$.

% Question with parts
\newpage
\addpoints
\question Consider the function $f(x)=x^2$.
\begin{parts}
\part[5] Find $f'(x)$ using the limit definition of derivative.
\vfill
\part[5] Find the line tangent to the graph of $y=f(x)$ at the point where $x=2$.
\vfill
\end{parts}

% If you want the total number of points for a question displayed at the top,
% as well as the number of points for each part, then you must turn off the point-counter
% or they will be double counted.
\newpage
\addpoints
\question[10] Consider the function $f(x)=x^3$.
\noaddpoints % If you remove this line, the grading table will show 20 points for this problem.
\begin{parts}
\part[5] Find $f'(x)$ using the limit definition of derivative.
\vspace{4.5in}
\part[5] Find the line tangent to the graph of $y=f(x)$ at the point where $x=2$.
\end{parts}
\documentclass[12pt,letterpaper]{article}
\usepackage[margin=1in]{geometry}
\usepackage{amsfonts}
\usepackage{amssymb}
\usepackage{amsthm}
\usepackage{amsmath}
\usepackage{enumerate}

%Here are some user-defined notations
\newcommand{\RR}{\mathbf R}  %bold R
\newcommand{\CC}{\mathbf C}  %bold C
\newcommand{\ZZ}{\mathbf Z}   %bold Z
\newcommand{\QQ}{\mathbf Q}   %bold Q
\newcommand{\rr}{\mathbb R}     %blackboard bold R
\newcommand{\cc}{\mathbb C}    %blackboard bold R
\newcommand{\zz}{\mathbb Z}    %blackboard bold R
\newcommand{\qq}{\mathbb Q}   %blackboard bold Q
\newcommand{\calM}{\mathcal M}  %calligraphic M
\newcommand{\sm}{\setminus} 
\newcommand{\bfa}{\mathbf a}
\newcommand{\bfb}{\mathbf b}
\newcommand{\bfc}{\mathbf c}




%Here are some user-defined operators
\newcommand{\re}{\operatorname {Re}}
\newcommand{\im}{\operatorname {Im}}


%These commands deal with theorem-like environments (i.e., italic)
\theoremstyle{plain}
\newtheorem{theorem}{Theorem}[section]
\newtheorem{corollary}[theorem]{Corollary}
\newtheorem{lemma}[theorem]{Lemma}
\newtheorem{conjecture}[theorem]{Conjecture}

%These deal with definition-like environments (i.e., non-italic)
\theoremstyle{definition}
\newtheorem{definition}[theorem]{Definition}
\newtheorem{example}[theorem]{Example}
\newtheorem{remark}[theorem]{Remark}

%your name and date in the header.
\usepackage[us]{datetime} 
\usepackage{fancyhdr}
\pagestyle{fancy}
\lhead{}
\chead{MATH 3210\\ Homework 5}
\rhead{ Your name \\ date}
\lfoot{}
\cfoot{}
\rfoot{\thepage}
\renewcommand{\headrulewidth}{0 pt}
\renewcommand{\footrulewidth}{0 pt}
\begin{document}
\begin{enumerate}[1.]
\item (\S 5.A \#3) Suppose $S,T\in \mathcal{L}(V)$ are such that $ST=TS$. Prove that $\text{ran}(S)$ is invariant under $T$. 
\begin{proof} If $v\in \text{ran}(S)$ then $v=S(u)$ for some $u\in V$. We apply $T$ to $v$. Hence we have that 
\[T(v)=TS(u)=ST(u).\]
Thus $T(v)=S(T(u))$ and is in the range of $S$. 
\end{proof}
\item (\S5.B \#1) Suppose that $T\in \mathcal{L}(V)$ and there exists a positive integer $n$ such that $T^n=0$. Prove that $(I-T)$ is invertible and that 
\[(I-T)^{-1}=I+T+\cdots +T^{n-1}\]
\begin{proof} We proceed by computation. 
\[(I-T)(I+T+\cdots+T^{n-1})=I+T+\cdots+T^{n-1}-(T+T^2+\cdots+T^n)=I+T^n=I.\]
Since $I$ commutes with $T$ and $T$ commutes with itself we have that 
\[(I+T+\cdots+T^{n-1})(I-T)=I.\]
\end{proof}
\item Suppose that $S,T\in \mathcal{L}(V)$ and $S$ is invertible. Suppose that $p\in \mathcal{P}(\mathbb{F})$ is a polynomial. Prove that 
\[p(STS^{-1})=Sp(T)S^{-1}.\]
\begin{lemma}
\[(STS^{-1})^n=ST^nS^{-1}\]
for all $n\in \mathbb{N}$.
\end{lemma}
\begin{proof} We proceed by induction on $n$\\
\ \\
{\bf Base Case: }$n=1$\\
\ \\
This case is clear. \\
\ \\
{\bf Induction Hypothesis: }For $n=k-1$ we have $(STS^{-1})^{k-1}=ST^{k-1}S^{-1}$.\\
\ \\
Now for $n=k$ we have that 
\[(STS^{-1})^k=(STS^{-1})^{k-1}(STS^{-1})=ST^{k-1}S^{-1}STS^{-1}=ST^{k}S^{-1}\]
by our induction hypothesis.
\end{proof} 
We now prove our main result.
\begin{proof}Suppose $p(z)=a_0+a_1z+\ldots+a_nz^n$. We then have that 
\begin{align*}
p(STS^{-1})&=a_0I+a_1(STS^{-1})+a_2(STS^{-1})^2+\ldots +a_n(STS^{-1})^n\\
&=a_0SS^{-1}+a_1STS^{-1}+a_2ST^2S^{-1}+\ldots +a_nST^nS^{-1}\\
&=Sp(T)S^{-1}
\end{align*}
by a repeated application of our lemma.
\end{proof}
\item (\S 5.C \# 16) The Fibonacci sequence $F_1, F_2, \ldots$ is defined by 
\[F_1=1, F_2=1, \quad \text{ and }\quad F_n=F_{n-2}+F_{n-1} \text{ for }n\geq 3\]
Define $T\in \mathcal{L}(\mathbb{R}^2)$ by 
\[T \left( \left[\begin{array}{c}x\\y\end{array}\right]\right)=\left[\begin{array}{cc}y\\ x+y\end{array}\right].\]
\begin{enumerate}[(a)]
\item Show that $T^n\left(\left[\begin{array}{c}0\\1\end{array}\right]\right)=\left[\begin{array}{c}F_n\\F_{n+1}\end{array}\right]$
\item Find the eigenvalues of $T$.
\item Find a basis of $\mathbb{R}^2$ consisting of eigenvectors of $T$. 
\item Use the solution to part (c) to compute $T^n\left(\left[\begin{array}{c}0\\1\end{array}\right]\right)$. Conclude that 
\[F_n=\frac{1}{\sqrt{5}}\left[\left(\frac{1+\sqrt{5}}{2}\right)^n-\left(\frac{1-\sqrt{5}}{2}\right)^n\right]\]
for each positive integer $n$. 
\end{enumerate}
\begin{proof}
\ \\
\begin{enumerate}[(a)]
\item We proceed by induction on $n$. \\
\ \\
{\bf Base case: }n=1\\
 \ \\
 Note that $T\left(\left[\begin{array}{c}0\\1\end{array}\right]\right)=\left[\begin{array}{c}F_1\\F_{2}\end{array}\right]$.\\
\ \\
 {\bf Induction Hypothesis: }Suppose for $n=k-1$ we have that 
 \[T^{k-1}\left(\left[\begin{array}{c}0\\1\end{array}\right]\right)=\left[\begin{array}{c}F_{k-1}\\F_{k}\end{array}\right]\]
 Now we apply $T$ to $T^{k-1}$ and we have that 
 \[T^{k}\left(\left[\begin{array}{c}0\\1\end{array}\right]\right)=T\left(\left[\begin{array}{c}F_{k-1}\\F_{k}\end{array}\right]\right)=\left[\begin{array}{c}F_{k}\\F_{k-1}+F_k\end{array}\right]=\left[\begin{array}{c}F_{k}\\F_{k+1}\end{array}\right]\]
by application of the recurrence relation and induction hypothesis.
\item The eigenvector equation 
\[T \left( \left[\begin{array}{c}x\\y\end{array}\right]\right)=\lambda\left[\begin{array}{cc}x\\ y\end{array}\right]\]
is equivalent to the system 
\[y=\lambda x \quad \text{ and }\quad x+y=\lambda y.\]
By substitution we have that 
\[x+\lambda x=\lambda^2x.\]
We note that $x\neq 0$ since this would imply $y=0$ by the system of equations and the zero vector is not a candidate for an eigenvector. Hence we can dived both sides by $x$ and get that 
\[\lambda^2-\lambda -1=0.\]
The only solutions to this equation are 
\[\lambda=\frac{1\pm \sqrt{5}}{2}.\]

\item We find the eigenvectors corresponding to the above eigenvectors. Substituting $\lambda=\frac{1\pm \sqrt{5}}{2}$ into the above system and solving for $x$ and $y$ shows that the eigenvectors are 
\[\left[\begin{array}{cc}1\\ \frac{1+\sqrt{5}}{2}\end{array}\right].\quad\text{ and }\quad\left[\begin{array}{cc}1\\ \frac{1-\sqrt{5}}{2}\end{array}\right].\] 
These vectors are clearly linearly independent and thus form a basis.
\item Note that 
\[\left[\begin{array}{c}0\\ 1\end{array}\right]=\frac{1}{\sqrt{5}}\left[\begin{array}{cc}1\\ \frac{1+\sqrt{5}}{2}\end{array}\right]-\frac{1}{\sqrt{5}}\left[\begin{array}{c}1\\ \frac{1-\sqrt{5}}{2}\end{array}\right].\]
Hence,
\[T^n\left(\left[\begin{array}{c}0\\ 1\end{array}\right]\right)=\frac{1}{\sqrt{5}}T^n\left(\left[\begin{array}{c}1\\ \frac{1+\sqrt{5}}{2}\end{array}\right]\right)-\frac{1}{\sqrt{5}}T^n\left(\left[\begin{array}{c}1\\ \frac{1-\sqrt{5}}{2}\end{array}\right]\right).\]
By our eigenvalue relation ship we have that 
\[T^n\left(\left[\begin{array}{c}0\\ 1\end{array}\right]\right)=\frac{1}{\sqrt{5}}\left(\frac{1+\sqrt{5}}{2}\right)^n\left[\begin{array}{c}1\\ \frac{1+\sqrt{5}}{2}\end{array}\right]-\frac{1}{\sqrt{5}}\left(\frac{1-\sqrt{5}}{2}\right)^n\left[\begin{array}{c}1\\ \frac{1-\sqrt{5}}{2}\end{array}\right].\]
By part (a) we have 
\[F_n=\frac{1}{\sqrt{5}}\left[\left(\frac{1+\sqrt{5}}{2}\right)^n-\left(\frac{1-\sqrt{5}}{2}\right)^n\right]\]
for each positive integer $n$. 

\end{enumerate}

\end{proof}
\end{enumerate}


\end{document}








